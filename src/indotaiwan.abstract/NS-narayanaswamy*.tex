%%%%%%%%%%%%%%%%%%%%%%%%%%%%%%%%%%%%%%%%%%%%%%%%%%%%%%%%%%%%%%%
\documentclass[twoside,12pt]{article}                        %%
%\usepackage{CJK}      %%% CJK package (in Big 5 encoding).  %%
\usepackage{amssymb}   %%% uncomment some or all of          %%
\usepackage{amsmath}   %%% these lines in case you           %%
\usepackage{latexsym}  %%% want to load the packages         %%
%%%%%%%%%%%%%%%%%%%%%%%%%%%%%%%%%%%%%%%%%%%%%%%%%%%%%%%%%%%%%%%
\topmargin 0.05truein                % do not modify          %
\oddsidemargin 0.4truein             % do not modify          %
\evensidemargin -.19truein           % do not modify          %
\textheight 9.0truein                % do not modify          %
\textwidth 6.0truein                 % do not modify          %
\headsep 0.5truein                   % do not modify          %
%%%%%%%%%%%%%%%%%%%%%%%%%%%%%%%%%%%%%%%%%%%%%%%%%%%%%%%%%%%%%%%
%%%%%%%%%%%%%%%%%%%%%%%%%%%%%%%%%%%%%%%%%%%%%%%%%%%%%%%%%%%%%%%

\pagestyle{myheadings}               % do not modify %
\def\a{2$^{nd}$ Indo-Taiwan Conference in Discrete Mathematics, Sep
  8-11, 2011, Amrita, India}         %
\def\h{\small\underline{\hbox{\a}}}  %
\markboth{\h}{\h}                    % do not modify %
%%%%%%%%%%%%%%%%%%%%%%%%%%%%%%%%%%%%%%%%%%%%%%%%%%%%%%%%%%%%%%%
%%%%%%%%%%%%%%%%%%%%%%%%%%%%%%%%%%%%%%%%%%%%%%%%%%%%%%%%%%%%%%%

%% font defs
\DeclareMathAlphabet{\mathpzc}{OT1}{pzc}{m}{it}
\DeclareMathAlphabet{\mathcalligra}{T1}{calligra}{m}{n}

%% string defs
\def\cA{{\cal A}} \def\cB{{\cal B}} \def\cC{{\cal C}} \def\cD{{\cal
    D}} \def\cE{{\cal E}} \def\cF{{\cal F}} \def\cG{{\cal G}}
\def\cH{{\cal H}} \def\cI{{\cal I}} \def\cJ{{\cal J}} \def\cK{{\cal
    K}} \def\cL{{\cal L}} \def\cM{{\cal M}} \def\cN{{\cal N}}
\def\cO{{\cal O}} \def\cP{{\cal P}} \def\cQ{{\cal Q}} \def\cR{{\cal
    R}} \def\cS{{\cal S}} \def\cT{{\cal T}} \def\cU{{\cal U}}
\def\cV{{\cal V}} \def\cW{{\cal W}} \def\cX{{\cal X}} \def\cY{{\cal
    Y}} \def\cZ{{\cal Z}} \def\hA{{\hat A}} \def\hB{{\hat B}}
\def\hC{{\hat C}} \def\hD{{\hat D}} \def\hE{{\hat E}} \def\hF{{\hat
    F}} \def\hG{{\hat G}} \def\hH{{\hat H}} \def\hI{{\hat I}}
\def\hJ{{\hat J}} \def\hK{{\hat K}} \def\hL{{\hat L}} \def\hP{{\hat
    P}} \def\hQ{{\hat Q}} \def\hR{{\hat R}} \def\hS{{\hat S}}
\def\hT{{\hat T}} \def\hX{{\hat X}} \def\hY{{\hat Y}} \def\hZ{{\hat
    Z}} \def\eps{\epsilon} \def\C{{\mathcal C}} \def\F{{\mathcal F}}
\def\A{{\mathcal A}} \def\H{{\mathcal H}} \def\bI{\mathbb I}
\def\bO{\mathbb O} \def\cl{\mathpzc{l}} \def\overlap{\between}

\begin{document}                     % do not modify          %
%\begin{CJK*}{Bg5}{bsmi}             % do not modify          %
\parskip 6pt                         % do not modify          %
\baselineskip 16pt                   % do not modify          %
                                     
%%%%%%%%%%%%%%%%%%%%%%%%%%%%%%%%%%%%%%%%%%%%%%%%%%%%%%%%%%%%%%%%%
%%%%%%%%%%%%%%%%%%%%%%%%%%%%%%%%%%%%%%%%%%%%%%%%%%%%%%%%%%%%%%%%%
                                     % Fill in the title of your
                                     % contribution.          %
\noindent {\bf\Large                 % do not modify          %
Tree Path Labeling of Path Hypergraphs: \\ 
A Generalization of Consecutive Ones Property}

%% do not modify
%%%%%%%%%%%%%%%%%%%%%%%%%%%%%%%%%%%%%%%%%%%%%%%%%%%%%%%%%%%%%%%
%%%%%%%%%%%%%%%%%%%%%%%%%%%%%%%%%%%%%%%%%%%%%%%%%%%%%%%%%%%%%%%
%% N.S. Narayanaswamy, Anju Srinivasan
%% ------------------
\vspace{0.5cm}
{
  \noindent N.S. Narayanaswamy$^{1}${~~\small {\tt
      swamy@cse.iitm.ernet.in}} \\
  \noindent {Anju Srinivasan}$^{*1}${~~\small {\tt
      asz@cse.iitm.ac.in}} 
}


%% do not modify                                          
%%%%%%%%%%%%%%%%%%%%%%%%%%%%%%%%%%%%%%%%%%%%%%%%%%%%%%%%%%%%%                               
%%%%%%%%%%%%%%%%%%%%%%%%%%%%%%%%%%%%%%%%%%%%%%%%%%%%%%%%%%%%%
%% Fill in your affiliation(s).                             
\noindent{\small \sl                 % do not modify %
$^{1}$Indian Institute of Technology Madras,  Chennai}
                                     % do not modify %

\vspace{0.5cm}                       % do not modify %

%% Give at least two keywords
\noindent{\textbf{Keywords:} Consecutive ones property, Hypergraph
  isomorphism, Interval labeling, Interval graphs, Path graphs}

%%Give 2000 MR subject Classification
\noindent{\textbf{2000 MR Subject Classification:} [05C85] Graph
  algorithms, [68R10] Graph theory, [68W40] Analysis of algorithms

\vspace{0.5cm}                       % do not modify %


%%%%%%%%%%%%%%%%%%%%%%%%%%%%%%%%%%%%%%%%%%%%%%%%%%%%%%%%%%%%%%                              
%%%%%%%%%%%%%%%%%%%%%%%%%%%%%%%%%%%%%%%%%%%%%%%%%%%%%%%%%%%%%%
%% Fill in the abstract/summary of your contribution/invited talk.

\noindent
We consider the following constraint satisfaction problem. Given (i) a
set system $\F \subseteq (powerset(U) \setminus \O)$ of a finite set
$U$ of cardinality $n$, (ii) a tree $T$ of size $n$ and (iii) a
bijection $\cl$, defined as {\em tree path labeling}, mapping the sets
in $\cF$ to paths in $T$, does there exist at least one bijection
$\phi:U \rightarrow V(T)$ such that for each $S \in \cF$, $\{\phi(x)
\mid x \in S\} = \cl(S)$?  A tree path labeling of a set system is
called {\em feasible} if there exists such a bijection $\phi$.  In
this paper, we characterize feasible tree path labeling of a given set
system to a tree.  This result is a natural generalization of results
on matrices with the Consecutive Ones Property. Moreover, we pose
some interesting algorithmic questions which extend from this work.


\noindent
Consecutive ones property (COP) of binary matrices is its property
of rearrangment rows (columns) in such a way that
every column (row) has its $1$s occuring consecutively.
% It
% has several applications \cite{hl06, k77, abh98, mcg04, ht02, hl06}.
% Recognition of COP is polynomial time solvable by several
% algorithms - PQ trees \cite{bl76}, variations of PQ
% trees \cite{mm09, wlh01, wlh02, mcc04}, ICPIA \cite{nsnrs09}.
The problem of COP testing is also a constraint satisfaction problem
of a set system as follows. In a binary matrix, if every column is
represented as a set of indices of the rows with $1$s in that column,
then if the matrix has the COP, a reordering of its rows will result
in sets that are intervals. The COP is equivalent to the problem of
finding interval assignments to a given set system \cite{nsnrs09} with
a single permutation of the universe which permutes each set to its
interval.  Clearly COP is a special instance of tree path labeling
problem described above when $T$ is a path.  The result in
\cite{nsnrs09} characterize interval assignments to the sets which can
be obtained from a single permutation of the rows - the cardinality of
the interval assigned to it must be same as the cardinality of the
set, and the intersection cardinality of any two sets must be same as
the interesction cardinality of the corresponding intervals -
Intersection Cardinality Preserving Interval Assignment (ICPIA).  This
is obviously necessary and was discovered to be sufficient.

\noindent
We focus on the question of generalizing the notion of an ICPIA
\cite{nsnrs09} to characterize feasible path assignments.  We show
that for a given set system $\cF$, a tree $T$, and an assignment of
paths from $T$ to the sets, there is a bijection $\phi$ between $U$
and $V(T)$ if and only if all intersection cardinalities among any
three sets (not necessarily distinct) is same as the intersection
cardinality of the paths assigned to them and the input successfully
executes our filtering algorithm (described in this paper) without
prematurely exiting.  Aside from finding the bijection $\phi$
mentioned above for a given path labeling, we also present a
characterization of set systems which have a feasible tree path
labeling on a given tree $T$ and present an algorithm to find a path
labeling if it exists when $T$ is a {\em $k$-subdivided star}. A 
  $k$-subdivided star is a star with all its rays subdivided exactly
$k$ times. The path from the center to a leaf is called a ray of a
$k$-subdivided star and they are all of length $k+2$. A {\em star}
graph is a complete bipartite graph $K_{1,l}$ which is clearly a tree
and $l$ is the number of leaves. The vertex with maximum degree is
called the {\em center} of the star and the edges are called {\em
  rays} of the star.

\noindent
The following questions are extensions to this work.
\begin{enumerate}
\item 
% It is an interesting fact that for a matrix with the COP, the
%   intersection graph of the corresponding set system is an interval
%   graph.  A similar connection to a subclass of chordal graphs and a
%   superclass of interval graphs exists for the generalization of COP.
%   In this case, 
  The intersection graph of a set system with a feasible tree path
  labeling from a tree $T$ must be a {\em path graph} which is a
  subclass of chordal graphs. This can be checked efficiently because
  path graph recognition is polynomial time solvable\cite{gav78,
    aas93}. However, this is only a necessary condition.  It is
  possible to have a pair of set system and tree $(\cF, T)$, such that
  the intersection graph of $\cF$ is a path graph, but there is no
  feasible tree path labeling to $T$. Therefore, the
  following questions.
  \begin{enumerate}
  \item What is the maximal set system $\cF' \subseteq \cF$ such that
    $(\cF', T)$ has a feasible tree path labeling?
  \item What is the maximal subtree $T' \subseteq T$ such that $(\cF,
    T)$ has a feasible tree path labeling?
  \item Path graph isomorphism is known be
    isomorphism-complete\cite{kklv10}. An interesting area of research
    would be to see what this result tells us about the complexity of
    the tree path labeling problem.
  \end{enumerate}

\item A set system $\cF$ can be alternatively represented by a {\em
    hypergraph} $\H_\cF$ whose vertex set is $supp(\cF)$ and
  hyperedges are the sets in $\cF$. This is a known representation for
  interval systems in literature \cite{kklv10}.  We extend this
  definition here to path systems.  Two hypergraphs $\cH$, $\cK$ are
  said to be isomorphic to each other, denoted by $\cH \cong \cK$, iff
  there exists a bijection $\phi: supp(\cH) \rightarrow supp(\cK)$
  such that for all sets $H \subseteq supp(\cH)$, $H$ is a hyperedge
  in $\cH$ iff $K$ is a hyperedge in $\cK$ where $K = \{\phi(x) \mid x
  \in H\}$.  If $\H_\cF \cong \H_\cP$ where $\cP$ is a path system,
  then $\H_\cF$ is called a {\em path hypergraph} and $\cP$ is called
  {\em path representation} of $\H_\cF$. If isomorphism is $\phi:
  supp(\H_\cF) \rightarrow supp(\H_\cP)$, then it is clear that there
  is an induced path labeling $l_\phi: \cF \rightarrow \cP$ to the set
  system. So our problem of finding if a given path labeling is a
  feasible path labeling is a path hypergraph isomorphism problem.
\end{enumerate}


%%%%%%%%%%%%%%%%%%%%%%%%%%%%%%%%%%%%%%%%%%%%%%%%%%%%%%%%%%%%%
%%% If you want, you may list some references here.
%
\frenchspacing                       % do not modify %


\begin{thebibliography}{99}          % do not modify %

  \itemsep 0pt                       % do not modify %

\bibitem{gav78} F.~Gavril.  \newblock A recognition algorithm for the
  intersection graphs of paths in trees.  \newblock {\em Discrete
    Mathematics}, 23(3):211 -- 227, 1978.

\bibitem{kklv10} J.~K{\"o}bler, S.~Kuhnert, B.~Laubner, and
  O.~Verbitsky.  \newblock Interval graphs: Canonical representation
  in logspace.  \newblock {\em Electronic Colloquium on Computational
    Complexity (ECCC)}, 17:43, 2010.

\bibitem{nsnrs09} N.~S. Narayanaswamy and R.~Subashini.  \newblock A
  new characterization of matrices with the consecutive ones property.
  \newblock {\em Discrete Applied Mathematics}, 157(18):3721--3727,
  2009.

\bibitem{aas93} A.~A. Schaffer.  \newblock A faster algorithm to
  recognize undirected path graphs.  \newblock {\em Discrete Applied
    Mathematics}, 43:261--295, 1993.

\end{thebibliography}                % do not modify %
%%%%%%%%%%%%%%%%%%%%%%%%%%%%%%%%%%%%%%%%%%%%%%%%%%%%%%%%%%%%%%%
%%%%%%%%%%%%%%%%%%%%%%%%%%%%%%%%%%%%%%%%%%%%%%%%%%%%%%%%%%%%%%% 
%\end{CJK*}                           % do not modify  %
\end{document}                        % do not modify  %
%%%%%%%%%%%%%%%%%%%%%%%%%%%%%%%%%%%%%%%%%%%%%%%%%%%%%%%%%%%%%%% 