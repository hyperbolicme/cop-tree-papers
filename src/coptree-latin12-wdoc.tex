%%
% Author: N S Narayanaswamy and Anju Srinivasan
%%

\documentclass[11pt,
               envcountsect,
               envcountsame]
               {../lib/llncs2e/llncs}

\usepackage{fullpage} %% llncs
\usepackage{latexsym}
\usepackage{amssymb}
\usepackage{amsfonts}
\usepackage{amsmath}
%\usepackage{comment}
%\usepackage{epsfig}
%\usepackage{graphicx}
%\usepackage{epstopdf}
\usepackage{algorithm}
\usepackage{algorithmic}
\usepackage{enumerate}
\usepackage{textcomp}
%\usepackage{pifont}

%%%%%%%%%%%%%%%%%%%%
%   TrackChanges   %
%
% Toggle these declarations for Review or Final version
% 
\usepackage[inline]{../lib/trackchanges}     % uncomment to see review comments
%\usepackage[finalnew]{../lib/trackchanges}   % uncomment to see no review notes
\addeditor {n}
\addeditor {a}

%
%
% finalold
%   Ignore all of the edits. 
%   The document will look as if the edits had not been added.
% finalnew
%   Accept all of the edits. 
%   Notes will not be shown in the final output.
% footnotes
%   Added text will be shown inline. Removed text and notes will be shown as footnotes. 
%   This is the default option.
% margins
%   Added text will be shown inline. Removed text and notes will be
%   shown in the margin. Margin notes will be aligned with the edits when possible.
% inline
%   All changes and notes will be shown inline.
% End TrackChanges %
%%%%%%%%%%%%%%%%%%%%

\DeclareMathAlphabet{\mathpzc}{OT1}{pzc}{m}{it}
\DeclareMathAlphabet{\mathcalligra}{T1}{calligra}{m}{n}


%%% string defs
\def\cA{{\cal A}}
\def\cB{{\cal B}}
\def\cC{{\cal C}}
\def\cD{{\cal D}}
\def\cE{{\cal E}}
\def\cF{{\cal F}}
\def\cG{{\cal G}}
\def\cH{{\cal H}}
\def\cI{{\cal I}}
\def\cJ{{\cal J}}
\def\cK{{\cal K}}
\def\cL{{\cal L}}
\def\cM{{\cal M}}
\def\cN{{\cal N}}
\def\cO{{\cal O}}
\def\cP{{\cal P}}
\def\cQ{{\cal Q}}
\def\cR{{\cal R}}
\def\cS{{\cal S}}
\def\cT{{\cal T}}
\def\cU{{\cal U}}
\def\cV{{\cal V}}
\def\cW{{\cal W}}
\def\cX{{\cal X}}
\def\cY{{\cal Y}}
\def\cZ{{\cal Z}}
\def\hA{{\hat A}}
\def\hB{{\hat B}}
\def\hC{{\hat C}}
\def\hD{{\hat D}}
\def\hE{{\hat E}}
\def\hF{{\hat F}}
\def\hG{{\hat G}}
\def\hH{{\hat H}}
\def\hI{{\hat I}}
\def\hJ{{\hat J}}
\def\hK{{\hat K}}
\def\hL{{\hat L}}
\def\hP{{\hat P}}
\def\hQ{{\hat Q}}
\def\hR{{\hat R}}
\def\hS{{\hat S}}
\def\hT{{\hat T}}
\def\hX{{\hat X}}
\def\hY{{\hat Y}}
\def\hZ{{\hat Z}}
\def\eps{\epsilon}
\def\C{{\mathcal C}}
\def\F{{\mathcal F}}
\def\A{{\mathcal A}}
\def\H{{\mathcal H}}
\def\bI{\mathbb I}
\def\bO{\mathbb O}
\def\cl{\mathpzc{l}}
\def\cg{\mathpzc{g}}
\def\ccT{\mathpzc{T}}
\def\overlap{\between}
\def\icppl{\maltese} 
\def\invb{\textreferencemark}
\def\lndisplay{1}
\def\marginal{{super-marginal }}
\def\icpplpr{Property}

\def\commentboxsize {7cm} %% llncs
\def\xnoindent{\noindent} %% llncs
\def\topshrink{0mm} %% llncs
\def\assign{\leftarrow}
\def\prelimspace{2mm}

%%% new/renew commands
% Format of comments in algorithmic package
\renewcommand{\algorithmiccomment}[1]
{ 
  \vspace {0.5mm}
  \hfill
  {\small
  \begin{tabular}{|r}
   \parbox[right]{\commentboxsize}{ \space \tt{ #1 }}\\  % {\tt /* #1 */}    \hspace{2mm}
  \end{tabular}
  }
}
% \renewcommand{\algorithmiccomment}[1]
% { 
% %  \hfill
% %  \parbox[right]{\commentboxsize} 
% {{\small \tt /* #1 */}}  % {\tt /* #1 */}    \hspace{2mm}
% }


% commands for theorems etc. 
\newtheorem{observation}{Observation}
\newcommand{\seq}[1]{\left\langle #1 \right\rangle}
\newcommand{\set}[1]{\left\{ #1\right\}}
\newcommand{\funct}[1]{(\left #1\right)}
% \newcommand{\Eqr}[1]{Eq.~(\ref{#1})}
% \newcommand{\diff}{\ne}
% \newcommand{\OO}[1]{O\left( #1\right)}
% \newcommand{\OM}[1]{\Omega\left( #1 \right)}
% \newcommand{\Prob}[1]{\Pr\left\{ #1 \right\}}
% \newcommand{\Set}[1]{\left\{ #1 \right\}}
% \newcommand{\Range}[1]{\left\{1,\ldots, #1 \right\}}
% \newcommand{\ceil}[1]{\left\lceil #1 \right\rceil}
% \newcommand{\floor}[1]{\left\lfloor #1 \right\rfloor}
% \newcommand{\ignore}[1]{}
% \newcommand{\eq}{\equiv}
% \newcommand{\abs}[1]{\left| #1\right|}
% \newcommand{\itoj}{{i \rightarrow j}}
% \newcommand{\view}{\mbox{$COMM$}}
% \newcommand{\pview}{\mbox{$PView$}}
% \newcommand{\vx}{\mbox{${\vec x}$}}
% \newcommand{\vy}{\mbox{${\vec y}$}}
% \newcommand{\vv}{\mbox{${\vec v}$}}
% \newcommand{\vw}{\mbox{${\vec w}$}}
% \newcommand{\vb}{\mbox{${\vec b}$}}
% \newcommand{\basic}{\mbox{\sc Basic}}
% \newcommand{\WR}{\mbox{$\lfloor wr \rfloor$}}
% \newcommand{\guarantee}{\mbox{\sc BoundedDT}}
% \newcommand{\sq}{{\Delta}}
% \newcommand{\Smin}{{S_{0}}}
% \newcommand{\outt}{{D^{^+}}}
% \newcommand{\outtp}{{\overline{D^{^+}}}}
% \newcommand{\inn}{{D^{^-}}}
% \newcommand{\innp}{{\overline{D^{^-}}}}
% \newcommand{\indexx}{{\gamma}}
% \newcommand{\D}{{D}}
% \newenvironment{denselist}{
%   \begin{list}{(\arabic{enumi})}{\usecounter{enumi}
%       \setlength{\topsep}{0pt} \setlength{\partopsep}{0pt}
%       \setlength{\itemsep}{0pt} }}{\end{list}}
% \newenvironment{denseitemize}{
%   \begin{list}{$\bullet$}{ \setlength{\topsep}{0pt}
%       \setlength{\partopsep}{0pt} \setlength{\itemsep}{0pt}
%     }}{\end{list}}
% \newenvironment{subdenselist}{
%   \begin{list}{(\arabic{enumi}.\arabic{enumii})}{ \usecounter{enumii}
%       \setlength{\topsep}{0pt} \setlength{\partopsep}{0pt}
%       \setlength{\itemsep}{0pt} }}{\end{list}}

% D O C U M E N T
\begin{document}
\mainmatter              % start of the contributions
\title{Tree Path Labeling of Path Hypergraphs --\\
  A Generalization of
  the Consecutive Ones Property} 
\titlerunning{Tree Path Labeling of Path Hypergraphs} % abbreviated title (for running
%                                      head) also used for the TOC
%                                      unless \toctitle is used 
\author{N.\,S.\,Narayanaswamy$^1$ \and Anju Srinivasan$^{2}$}
\authorrunning{N.S. Narayanaswamy Anju
  Srinivasan} % abbreviated author list (for running head)

\institute{ Indian Institute of Technology
  Madras, Chennai - 600036.\\
  \email{$^1$swamy@cse.iitm.ernet.in, $^2$asz@cse.iitm.ac.in}}
\maketitle
\begin{abstract}
  In this paper, we explore a natural generalization of results on
  binary matrices with the {\em consecutive ones property}.  We
  consider the following constraint satisfaction problem. Given (i) a
  set system $\F \subseteq$ $(2^{U} \setminus \emptyset)$ of a finite
  set $U$ of cardinality $n$, (ii) a tree $T$ of size $n$ and (iii) a
  bijection called {\em tree path labeling}, $\cl$ mapping the sets in
  $\cF$ to paths in $T$, does there exist at least one bijection
  $\phi:U \rightarrow V(T)$ such that for each $S \in \cF$, $\{\phi(x)
  \mid x \in S\} = \cl(S)$?  A tree path labeling of a set system is
  called {\em feasible} if there exists such a bijection $\phi$.  We
  present an algorithmic characterization of feasible tree path
  labeling. COP is a special instance of tree path labeling problem
  when $T$ is a path.  We conclude with a polynomial time algorithm to
  find a test for a feasible tree path labeling of a given set system when $T$ is
  a {\em $k$-subdivided star} when all the sets are of size at most $k+2$.

  \keywords{consecutive ones property, algorithmic graph theory, hypergraph
    isomorphism, interval labeling}
\end{abstract}
\section{Introduction}
Consecutive ones property (COP) of binary matrices is a widely studied
combinatorial problem. The problem is to rearrange rows (columns) of a
binary matrix in such a way that every column (row) has its $1$s occur
consecutively. If this is possible the matrix is said to have the COP.
This problem has several practical applications in diverse fields
including scheduling \cite{hl06}, information retrieval \cite{k77} and
computational biology \cite{abh98}.  Further, it is a tool in graph
theory \cite{mcg04} for interval graph recognition, characterization
of Hamiltonian graphs, and in integer linear programming
\cite{ht02,hl06}.  Recognition of COP is polynomial time solvable by
several algorithms. PQ trees \cite{bl76}, variations of PQ trees
\cite{mm09,wlh01,wlh02,mcc04}, ICPIA \cite{nsnrs09} are the main ones.

% \noindent
The problem of COP testing can be easily seen as a constraint
satisfaction problem involving a system of sets from a universe. Every
column of the binary matrix can be converted into a set of integers
which are the indices of the rows with $1$s in that column. When
observed in this context, if the matrix has the COP, a reordering of
its rows will result in sets that have only consecutive integers. In
other words, the sets after reordering are intervals. Indeed the
problem now becomes finding interval assignments to the given set
system \cite{nsnrs09} with a single permutation of the universe (set
of row indices) which permutes each set to its interval. The result in
\cite{nsnrs09} characterizes interval assignments to the sets which
can be obtained from a single permutation of the rows.  They show that
for each set, the cardinality of the interval assigned to it must be
same as the cardinality of the set, and the intersection cardinality
of any two sets must be same as the intersection cardinality of the
corresponding intervals.  While this is naturally a necessary
condition, \cite{nsnrs09} shows this is indeed sufficient.  Such an
interval assignment is called an Intersection Cardinality Preserving
Interval Assignment (ICPIA).  Finally, the idea of decomposing a given
0-1 matrix into prime matrices to check for COP is adopted from
\cite{wlh02} to test if an ICPIA exists for a given set system.

\noindent {\bf Our Work.}  A natural generalization of the interval
assignment problem is feasible tree path labeling problem of a set
system. The problem is defined as follows -- given a set system $\cF$
from a universe $U$ and a tree $T$, does there exist a bijection from
$U$ to the vertices of $T$ such that each set in the system maps to a
path in $T$.  We refer to this as the {\em tree path labeling problem}
for an input set system, target tree pair -- $(\cF,T)$. As a special
case if the tree $T$ is a path, the problem becomes the interval
assignment problem.  We focus on the question of generalizing the
notion of an ICPIA \cite{nsnrs09} to characterize feasible path
assignments.  We show that for a given set system $\cF$, a tree $T$,
and an assignment of paths from $T$ to the sets, there is a feasible
bijection between $U$ and $V(T)$ if and only if all intersection
cardinalities among any three sets (not necessarily distinct) is same
as the intersection cardinality of the paths assigned to them and the
input passes a filtering algorithm (described in this paper)
successfully.  This characterization is proved constructively and it
gives a natural data structure that stores all the relevant feasible
bijections between $U$ and $V(T)$.  Further, the filtering algorithm
is also an efficient algorithm to test if a tree path labeling to the
set system is feasible.  This generalizes the result in
\cite{nsnrs09}.

\noindent 
Our results also have close connection to recognition of {\em path
  graphs} and connection to path graph isomorphism.  The reason is
that given a hypergraph $\cF$, it can be viewed as paths in a tree, if
and only if the intersection graph of $\cF$ is {\em path graph}.  Path
graphs are a subclass of chordal graphs which are combinatorially
characterized as the intersection graphs of subtrees of a tree.  These
results are well studied in the literature in
\cite{plr70,gav78,bp93,mcg04}.  Chordal graphs which can be
represented as the intersection graph of paths in a tree are called
path graph \cite{mcg04}.  Checking if a graph is a path graph can be
done in polynomial time by the results of \cite{gav78,aas93}.
However, this is only a necessary condition for our question.  Indeed,
our work provides a sufficient condition for a related problem: that
is whether a path assignment to be feasible?  On the other hand, path
graph isomorphism is known be isomorphism-complete, see for example
\cite{kklv10}. Therefore, it is unlikely that we can solve the for a
feasible path labeling $\cl$ for a given $\cF$ and tree $T$.  It is
definitely intersecting to classify the kinds of trees and hypergraphs
for which feasible path labelings can be found efficiently.  These
results would form a natural generalization of COP testing and
interval graph isomorphism, culminating in Graph Isomorphism itself.


\noindent 
{\bf Our Results:}
\begin{enumerate}
\item Given a path labeling $\cl$ to $\cF$ from a tree $T$, we give a
  necessary and sufficient condition for it to be a feasible path
  labeling.  This necessary and sufficient condition can be testing in
  polynomial time.  The most interesting consequence is that in our
  constructive procedure, it is sufficient to iteratively check if
  three-way intersection cardinalities are preserved.  In other words,
  in each iteration, it is sufficient to check if the intersection any
  three sets is of the same cardinality as the intersection of the
  corresponding paths.  This generalizes the well studied question of
  the case when the given tree $T$ is a path \cite{wlh02,nsnrs09}.
\item In the Section 4, we initiate an exploration of finding feasible
  path labeling of set systems in a special kind of tree which we call
  the $k$-subdivided star.  This question is an attempt to generalize
  the problem of testing if a matrix the consecutive ones property.
  However, we restrict the hypergraph $\cF$ to be such that all
  hyperedges have at most $k+2$ elements.  In spite of this restricted
  case we consider, we believe that our results are of significant
  interest in understanding the nature of Graph Isomorphism which is
  polynomial time solvable in interval graphs and is hard on path
  graphs.
\end{enumerate}
% In the later part of this paper, we decompose our search for a
% bijection between $U$ and $V(T)$ into subproblems.  Each subproblem
% is on a set subsystem in which for each set, there is another set in
% the set subsystem with which the intersection is {\em strict}, i.e.,
% there is a non-empty intersection, but neither is contained in the
% other.  This is in the spirit of results in \cite{wlh02,nsnrs09}
% where to test for the COP in a given matrix, the COP problem is
% solved on an equivalent set of prime matrices.
\noindent {\bf Roadmap.} The necessary preliminaries with definitions
etc. are presented in
Section~\ref{sec:prelims}. Section~\ref{sec:feasible} documents the
characterization of a feasible path labeling and finally,
Section~\ref{sec:ksubdivstar} describes a polynomial time algorithm to
find the tree path labeling of a given set system from a given
$k$-subdivided tree.

\section{Preliminaries} 
\label{sec:prelims} 

{\bf Hypergraph Preliminaries:} The set $\F \subseteq (2^{U} \setminus
\emptyset)$ is a {\em set system} of a universe $U$ with $|U| = n$.
The {\em support} of a set system $\F$ denoted by $supp(\cF)$ is the
union of all the sets in $\F$; $supp(\F) = \bigcup_{S \in \F}S$. For
the purposes of this paper, a set system is required to ``cover'' the
universe; $ supp(\cF) = U$.  A set system $\cF$ can also be visualized
as a {\em hypergraph}\, vertex set is $supp(\cF)$ and hyperedges are
the sets in $\cF$.  The {\em intersection graph}\, $\bI(\cF)$ of a
hypergraph $\cF$ is a graph such that its vertex set has a bijection
to $\cF$ and there exists an edge between two vertices iff their
corresponding hyperedges have a non-empty intersection \cite{mcg04}.
Two hypergraphs $\cF'$, $\cF''$ are said to be {\em isomorphic} to
each other, denoted by $\cF' \cong \cF''$, iff there exists a
bijection $\phi: supp(\cF') \rightarrow supp(\cF'')$ such that for all
sets $A \subseteq supp(\cF')$, $A$ is a hyperedge in $\cF'$ iff $B$ is
a hyperedge in $\cF''$ where $B = \{\phi(x) \mid x \in A\}$
\cite{kklv10}, written as $B=\phi(A)$.
% This is called {\em hypergraph isomorphism}.

% \vspace{\prelimspace}
\noindent
% A set system $\cF$ can be alternatively represented by a {\em
%   hypergraph} $\H_\cF$ whose vertex set is $supp(\cF)$ and
% hyperedges are the sets in $\cF$. This is a known representation for
% interval systems in literature \cite{bls99,kklv10}.  We extend this
% definition here to path systems.
{\bf Path Hypergraph from a Tree:} We consider trees $T$ such that
$|V(T)|=|U|=n$.  A {\em path system}\, $\cP$ is a set system of paths
from $T$; $\cP \subseteq \{P \mid P \subseteq V, \text{ } T[P] \text{
  is a path} \}$.  This generalizes the fact, from the literature
\cite{bls99,kklv10}, that intervals can be viewed as sub-paths of a
path.
% Due to the equivalence of set system and hypergraph in the scope of
% this paper, we drop the subscript $_H$ in the notation and refer to
% both the structures by $\cF$.
% \vspace{\prelimspace}%\tnote[a]{also extend $\phi$ to hyperedges -- see if
% required } \vspace{\prelimspace} \vspace{\prelimspace}
% \noindent
If the intersection graphs of $\cF$ and $\cP$( a path system) are
isomorphic, $\bI(\cF) \cong \bI(\cP)$, then the associated bijection
$\cl: \cF \rightarrow \cP$ due to this isomorphism is called a {\em
  path labeling} of the hypergraph $\cF$.  Note that there are two
kinds of isomorphisms here.  We are concerned about the isomrophisms
intersection graphs on $\cF$ and $\cP$, and also the isomorphism
between the hypergraph $\cF$ and $\cP$.
% To illustrate further, let $\cg: V(\cF) \rightarrow V(\cP)$ be the
% above mentioned isomorphism where $V(\cF)$ and $V(\cP)$ are the
% vertex sets that represent the hyperedges for each hypergraph
% respectively, $V(\cF) = \{ v_S \mid S \in \cF\}$ and $V(\cP) = \{
% v_P \mid P \in \cP\}$. Then the path labeling $\cl$ is defined as
% follows: $\cl(S_1) = P_1$ iff $\cg (v_{S_1}) = v_{P_1}$.  Just to
% emphasize, for a path labeling $\cl$ of $\cF$ with $\cP$ as the path
% system, $\cF^\cl$ is same as $\cP$.  The path system $\cP$ may be
% alternatively denoted in terms of $\cF$ and $\cl$ as $\cF^\cl$. In
% most scenarios in this paper, what is given are the pair $(\cF,
% \cl)$ and the target tree $T$; hence this notation will be used more
% often.
If $\cF \cong \cP$ where $\cP$ is a path system, then $\cF$ is called
a {\em path hypergraph} and $\cP$ is called {\em path representation}
of $\cF$. If this isomorphism is $\phi: supp(\cF) \rightarrow V(T)$,
then it is clear that there is a path labeling $\cl_\phi: \cF
\rightarrow \cP$ to the set system; $\cl_\phi(S) = \set{y \mid y =
  \phi(x), x \in S}$ for all $S \in \cF$. In other words, if $\cF
\cong \cP$, we get a path labeling.  Recall that $supp(\cP) = V(T)$.
% \vspace{\prelimspace} A graph $G$ is a {\em path graph} if it is
% isomorphic to the intersection graph $\bI(\cP)$ of a path system
% $\cP$.  This isomorphism gives a bijection $\cl': V(G) \rightarrow
% \cP$. Moreover, for the purposes of this paper, we require that in a
% path labeling, $supp(\cP) = V(T)$.  If graph $G$ is also isomorphic
% to $\bI(\cF)$ for some hypergraph $\cF$, then clearly there is a
% bijection $\cl: \cF \rightarrow \cP$ such that $\cl(S) = \cl'(v_S)$
% where $v_S$ is the vertex corresponding to set $S$ in $\bI(\cF)$ for
% any $S \in \cF$. This bijection $\cl$ is called the {\em path
%   labeling} of the hypergraph $\cF$ and the path system $\cP$ may be
% alternatively denoted as $\cF^\cl$.  \vspace{\prelimspace}
In this work, we are given as input $\cF$ and a tree $T$, and the
question is whether there is a path labeling $\cl$ to a set of paths
in $T$.  We refer to such a solution path system by $\cF^\cl$.  A path
labeling $(\cF, \cl)$ is defined to be {\em feasible} if
% $\cF \cong \cF^\cl$ and this
there is a hypergraph isomorphism $\phi: supp(\cF) \rightarrow
supp(\cF^\cl)=V(T)$ induces a path labeling $\cl_\phi: \cF \rightarrow
\cF^\cl$ such that $\cl_\phi = \cl$.
% \vspace{\prelimspace}

\xnoindent {\bf Overlap Graphs and Marginal Hyperedges:} An {\em
  overlap graph}\, $\bO(\cF)$ of a hypergraph $\cF$ is a graph such
that its vertex set has a bijection to $\cF$ and there exists an edge
between two of its vertices iff their corresponding hyperedges
overlap. Two hyperedges $S$ and $S'$ are said to {\em overlap},
denoted by $S \overlap S'$, if they have a non-empty intersection and
neither is contained in the other; $S \overlap S' \text{ iff } S \cap
S' \ne \emptyset, S \nsubseteq S', S' \nsubseteq S$. Thus $\bO(\cF)$
is a spanning subgraph of $\bI(\cF)$ and not necessarily
connected. Each connected component of $\bO(\cF)$ is called an {\em
  overlap component}.
% If there are $d$ overlap components in $\bO(\cF)$, the set
% subsystems are denoted by $\cO_1, \cO_2, \ldots \cO_d$. Clearly
% $\cO_i \subseteq \F, i \in [d]$. For any $i, j \in [d]$, it can be
% verified that one of the following is true.
% \begin{enumerate}[a) ]
% \item $supp(\cO_i)$ and $supp(\cO_j)$ are disjoint
% \item $supp(\cO_i)$ is a subset of a set in $\cO_j$
% \item $supp(\cO_j)$ is a subset of a set in $\cO_i$
% \end{enumerate}
% \vspace{\prelimspace} \xnoindent
A hyperedge $S \in \cF$ is called {\em marginal} if for all $S'
\overlap S$, the overlaps $S \cap S'$ form a single inclusion chain
\cite{kklv10}. Additionally, if $S$ is such that it is contained in no
other marginal hyperedge in $\cF$, then it is called {\em
  super-marginal}.
% i.e., it is inclusion maximal then it is called {\em
%   super-marginal}.

\xnoindent {\bf $k$-subdivided star -- a special tree} A {\em star} graph
is a complete bipartite graph $K_{1,p}$ which is clearly a tree and
$p$ is the number of leaves. The vertex with maximum degree is called
the {\em center} of the star and the edges are called {\em rays} of
the star. A {\em $k$-subdivided star} is a star with all its rays
subdivided exactly $k$ times. The definition of a {\em ray of a
  $k$-subdivided star} is extended to the path from the center to a
leaf. It is clear that all rays are of length $k+2$.


\section{Characterization of Feasible Tree Path  Labelings} \label{sec:feasible} 
In this section we give an algorithmic characterization of a
feasibility of tree path labeling.
% Consider a path labeling $\cl: \cF \rightarrow \cP$ for set system $\cF$
% and path system $\cP$ on the given tree $T$. We
% call $\cl$ an {\em Intersection Cardinality Preserving Path Labeling
%   (ICPPL)} if it has the following properties.
Consider a path labeling $(\cF, \cl)$ on the
given tree $T$. We call $(\cF, \cl)$ an {\em Intersection Cardinality
  Preserving Path Labeling (ICPPL)} if it has the following
properties.

%\vspace{\topshrink}

\begin{enumerate}[{(\icpplpr\ }i) \ \ \ ]
\item \label{pr:i} $|S| = |\cl(S)|$ \ \ \ \ \ \ \ \ \ \ \ \ \ \ \ \ \ \ \ \ \ \ \ \
  \ \ \ \ \ \ \ \ \ \ \ \ \ \ \ for all $S \in \cF$
%  \vspace{\topshrink}
\item \label{pr:ii}$|S_1 \cap S_2| = |\cl(S_1) \cap \cl(S_2)|$ \ \ \ \ \ \ \ \ \ \
  \ \ \ \ \ \ \ \ \ \ for all distinct
  $S_1, S_2 \in \cF$
%  \vspace{\topshrink}
\item \label{pr:iii}$|S_1 \cap S_2 \cap S_3| = |\cl(S_1) \cap \cl(S_2) \cap
  \cl(S_3)|$ \ \ \ for all distinct $S_1, S_2, S_3 \in
  \cF$
\end{enumerate}


\noindent
The following lemma is useful in subsequent arguments. 
\begin{lemma}
  \label{lem:setminuscard}
  If $\cl$ is an ICPPL, and $S_1, S_2, S_3 \in \cF$, then $|S_1 \cap
  (S_2 \setminus S_3)| = |\cl(S_1) \cap (\cl(S_2) \setminus
  \cl(S_3))|$.
\end{lemma}
\begin{proof}%[Proof of Lemma~\ref{lem:setminuscard}]
  Let $P_i = \cl(S_i)$, for all $1 \le i \le  3$.
  $|S_1 \cap (S_2 \setminus S_3)| = |(S_1 \cap S_2) \setminus S_3| =
  |S_1 \cap S_2| - |S_1 \cap S_2 \cap S_3|$. Due to properties (ii)
  and (iii) of ICPPL, $|S_1 \cap S_2| - |S_1 \cap S_2 \cap S_3| = |P_1
  \cap P_2| - |P_1 \cap P_2 \cap P_3| = |(P_1 \cap P_2) \setminus P_3|
  = |P_1 \cap (P_2 \setminus P_3)|$. Thus lemma is proved. \qed
\end{proof}

\def \xnoindent  {}

\xnoindent In the remaining part of this section we show that $(\cF,
\cl)$ is feasible if and only if it is an ICPPL and
Algorithm~\ref{al:icppl-find-isomorph} returns a non-empty
function. Algorithm~\ref{al:icppl-find-isomorph} recursively does two levels of
filtering of $(\cF, \cl)$ to make it simpler while retaining the set
of isomorphisms, if any, between $\cF$ and $\cF^\cl$.
% One direction of this claim isclear: that if a path
% labeling is feasible, then all intersection cardinalities are
% preserved, i.e. the path labeling is an ICPPL. Algorithm~\ref{perms}
% \annote[a]{has no premature exit condition hence any input will go
%   through it}{Prove that the filtered sets has ICPPL iff input PL
%   has ICPPL?}. Algorithm~\ref{leafasgn} has an exit condition at
% line~\ref{xempty}. It can be easily verified that $X$ cannot be
% empty if $\cl$ is a feasible path labeling. The reason is that a
% feasible path labeling has an associated bijection between
% $supp(\cF)$ and $V(T)$ \remove[a]{i.e. $supp(\cF^{\cl})$} such that
% the sets map to paths, ``preserving'' the path labeling.  The rest
% of the section is devoted to constructively proving that it is
% sufficient for a path labeling to be an ICPPL and pass the two
% filtering algorithms.  To describe in brief, the constructive
% approaches refine an ICPPL iteratively, such that at the end of each
% iteration we have a ``filtered'' path labeling, and finally we have
% a path labeling that defines a family of bijections from $supp(\cF)$
% to $V(T)$\remove[a]{ i.e. $supp(\cF^{\cl})$}.
First, we present Algorithm~\ref{perms} or {\tt filter\_1}, and prove
its correctness.  This algorithm refines the path labeling by
processing pairs of paths in $\cF^\cl$ that share a leaf until no two
paths in the new path labeling share any leaf.

\begin{algorithm}[h]
  \caption{Refine ICPPL {\tt filter\_1($\cF, \cl, T$)}}
  \label{perms}
  \begin{algorithmic}[\lndisplay]
    \STATE $\cF_0 \assign \cF$, $\cl_0(S) \assign \cl(S)$ for all $S \in \cF_0$\\
    \STATE $j \assign 1$\\
    \WHILE {there is $S_1, S_2 \in \cF_{j-1}$ such that
      $\cl_{j-1}(S_1)$ and $\cl_{j-1}(S_2)$ have a common leaf in
      $T$}\label{shareleaf} \STATE $\cF_j \assign (\cF_{j-1} \setminus
    \{S_1, S_2\})
    \cup \{S_1 \cap S_2, S_1 \setminus S_2, S_2 \setminus S_1 \}$ \label{setbreak} 
    \COMMENT {Remove $S_1$, $S_2$ and add the ``filtered'' sets}
    \STATE {\bf for} every $S \in \cF_{j-1}$ s.t. $S \ne S_1$ and $S \ne
    S_2$ {\bf do} $\cl_j(S) \assign \cl_{j-1}(S)$ {\bf end for}\\

    \STATE $\cl_j(S_1 \cap S_2) \assign \cl_{j-1}(S_1) \cap
    \cl_{j-1}(S_2)$
    \COMMENT {Carry forward the path labeling for all existing sets other than
      $S_1$, $S_2$}
    \STATE $\cl_j(S_1 \setminus S_2) \assign \cl_{j-1}(S_1) \setminus
    \cl_{j-1}(S_2)$ 
    \COMMENT {Define path labeling for new sets}
    \STATE $\cl_j(S_2 \setminus S_1) \assign \cl_{j-1}(S_2) \setminus
    \cl_{j-1}(S_1)$

    \IF{$(\cF_j, \cl_j)$ does not satisfy (\icpplpr~\ref{pr:iii}) of ICPPL}
    \label{ln:3waycheck}
    \STATE {\bf exit} \label{ln:exit1} \\
    \ENDIF

    \STATE $j \assign j+1$\\
    \ENDWHILE
    \STATE $\cF' \assign \cF_j$, $\cl' \assign \cl_j$\\
    \RETURN $(\cF', \cl')$
  \end{algorithmic}
\end{algorithm}

\begin{lemma} 
 \label{lem:feasible} 
 In Algorithm~\ref{perms}, if input $(\cF, \cl)$ is a feasible path
 assignment then at the end of $j$th iteration of the {\bf while}
 loop, $j \ge 0$, $(\cF_j, \cl_j)$ is a feasible path assignment.
\end{lemma}
\begin{proof}%[Proof of Lemma~\ref{lem:feasible}]
  We will prove this by mathematical induction on the number of
  iterations. The base case $(\cF_0, \cl_0)$ is feasible since it is
  the input itself which is given to be feasible. Assume the lemma is
  true till $j-1$th iteration. i.e. every hypergraph isomorphism
  $\phi: supp\left(\cF_{j-1}\right) \rightarrow V\left(T \right)$ that
  defines $(\cF, \cl)$'s feasibility, is such that the induced path
  labeling on $\cF_{j-1}$, $\cl_{\phi[{\cF_{j-1}}]}$ is equal to
  $\cl_{j-1}$. We will prove that $\phi$ is also the bijection that
  makes $(\cF_j, \cl_j)$ feasible. Note that $supp(\cF_{j-1}) =
  supp(\cF_{j})$ since the new sets in $\cF_j$ are created from basic
  set operations to the sets in $\cF_{j-1}$. For the same reason and
  $\phi$ being a bijection, it is clear that when applying the $\phi$
  induced path labeling on $\cF_j$, $ \cl_{\phi[{\cF_{j}}]}(S_1
  \setminus S_2) = \cl_{\phi[{\cF_{j-1}}]}(S_1) \setminus
  \cl_{\phi[{\cF_{j-1}}]}(S_2)$. Now observe that $ \cl_j(S_1
  \setminus S_2) = \cl_{j-1}(S_1) \setminus \cl_{j-1}(S_2) =
  \cl_{\phi[{\cF_{j-1}}]}(S_1) \setminus
  \cl_{\phi[{\cF_{j-1}}]}(S_2)$. Thus the induced path labeling
  $\cl_{\phi[{\cF_{j}}]} = \cl_{j}$. Therefore lemma is proved.  \qed
\end{proof}

\begin{lemma}
  \label{lem:invar1} In Algorithm~\ref{perms}, at the end of $j$th
  iteration, $j \ge 0$, of the {\bf while} loop, the following
  invariants are maintained.
  \begin{enumerate}[I {\ }] %\vspace{\topshrink}
  \item $\cl_j(R)$ is a path in $T$, \ \ \ \ \ \ \ \ \ \ \ \ \ \ \ \ \
    \ \ \ \ \ \ \ \ \ \ for all $R \in \cF_j$%\vspace{\topshrink}
  \item $|R| = |\cl_j(R)|$, \ \ \ \ \ \ \ \ \ \ \ \ \ \ \ \ \ \ \ \ \
    \ \ \ \ \ \ \ \ \ \ \ \ \ \ \ for all $R \in
    \cF_j$%\vspace{\topshrink}
  \item $|R \cap R'| = |\cl_j(R) \cap \cl_j(R')|$, \ \ \ \ \ \ \ \ \ \
    \ \ \ \ \ \ \ \ \ \ for all $R, R' \in \cF_j$%\vspace{\topshrink}
  \item $|R \cap R' \cap R''|=|\cl_j(R) \cap \cl_j(R') \cap
    \cl_j(R'')|$, \ \ \ for all $R, R', R'' \in \cF_j$
  \end{enumerate}
\end{lemma}

\begin{proof}
 Proof is by induction on the number of iterations, $j$. In this
  proof, the term ``new sets'' will refer to the sets added to $\cF_j$
  in $j$th iteration in line~\ref{setbreak} of Algorithm~\ref{perms},
  $S_1 \cap S_2, S_1 \setminus S_2, S_2 \setminus S_1$ and its
  images in $\cl_j$ where $\cl_{j-1}(S_1)$
  and $\cl_{j-1}(S_2)$ intersect and share a leaf.\\
  The invariants are true in the base case $(\cF_0, \cl_0)$, since it
  is the input ICPPL.  Assume the lemma is true till the $j-1$th
  iteration. Let us consider the possible cases for each of the above invariants for
  the $j$th iteration.

  \xnoindent
 \begin{enumerate}[\textreferencemark]
  \item {\em Invariant} I/II
    \begin{enumerate}[{I/II}a $|$] % \textbullet 
    \item {\em $R$ is not a new set.} It is in $\cF_{j-1}$. Thus
      trivially true by induction hypothesis.
    \item {\em $R$ is a new set.} If $R$ is in $\cF_{j}$ and not in
      $\cF_{j-1}$, then it must be one of the new sets added in
      $\cF_j$. In this case, it is clear that for each new set, the
      image under $\cl_j$ is a path since by definition the chosen
      sets $S_1$, $S_2$ are from $\cF_{j-1}$ and due to the while loop
      condition, $\cl_{j-1}(S_1)$, $\cl_{j-1}(S_2)$ have a
      common leaf. Thus invariant I is proved.\\
      Moreover, due to induction hypothesis of invariant III and the
      definition of $l_j$ in terms of $l_{j-1}$, invariant II is
      indeed true in the $j$th iteration for any of the new sets.  If
      $R = S_1 \cap S_2$, $|R| = |S_1 \cap S_2| = |\cl_{j-1}(S_1) \cap
      \cl_{j-1}(S_2)| = |\cl_j(S_1 \cap S_2)| = |\cl_j(R)|$.
      If $R = S_1 \setminus S_2$, $|R| = |S_1 \setminus S_2| = |S_1| -
      |S_1 \cap S_2| = |\cl_{j-1}(S_1)| - |\cl_{j-1}(S_1) \cap
      \cl_{j-1}(S_2)| = |\cl_{j-1}(S_1) \setminus \cl_{j-1}(S_2)| =
      |\cl_j(S_1 \setminus S_2)|
      = |\cl_j(R)|$. Similarly if $R = S_2 \setminus S_1$.\\
    \end{enumerate}
  \item {\em Invariant} III
    \begin{enumerate}[{III}a $|$]
    \item {\em $R$ and $R'$ are not new sets.} It is in
      $\cF_{j-1}$. Thus trivially true by induction hypothesis.
    \item {\em Only one, say $R$, is a new set.} Due to invariant IV
      induction hypothesis, Lemma~\ref{lem:setminuscard} and
      definition of $\cl_j$, it follows that invariant III is true no
      matter which of the new sets $R$ is equal to. If $R = S_1 \cap
      S_2$, $|R \cap R'| = |S_1 \cap S_2 \cap R'| = |\cl_{j-1}(S_1)
      \cap \cl_{j-1}(S_2) \cap \cl_{j-1}(R')| = |\cl_j(S_1 \cap S_2)
      \cap \cl_j(R')| = |\cl_j(R) \cap \cl_j(R')|$.  If $R = S_1
      \setminus S_2$, $|R \cap R'| = |(S_1 \setminus S_2) \cap R'| =
      |(\cl_{j-1}(S_1) \setminus \cl_{j-1}(S_2)) \cap \cl_{j-1}(R')| =
      |\cl_{j}(S_1 \cap S_2) \cap \cl_{j}(R')| = |\cl_{j}(R) \cap
      \cl_{j}(R')|$. Similarly, if $R = S_2 \setminus
      S_1$. Note $R'$ is not a new set.\\

    \item {\em $R$ and $R'$ are new sets.} By definition, the new
      sets and their path images in path label $\cl_j$ are disjoint so
      $|R \cap R'| = |\cl_j(R) \cap \cl_j(R)| = 0$. Thus case proved.
    \end{enumerate}
  \item {\em Invariant} IV
    
    Due to the condition in line~\ref{ln:3waycheck}, this invariant is
    ensured at the end of every iteration.
%     \begin{enumerate} [{Case 3.}1:]
%     \item {\em $R$, $R'$ and $R''$ are not new sets.} Trivially
%       true by induction hypothesis.
%     \item {\em Only one, say $R$, is a new set.}
%       If $R = S_1 \cap S_2$,  from Lemma~\ref{lem:fourpaths} and
%       invariant III hypothesis,  this case is proven. Similarly if $R$
%       is any of the other new  sets, the case is proven by also using
%       Lemma ~\ref{lem:setminuscard}.
%     \item {\em At least two of $R, R', R''$ are new sets.}
%       The new sets are disjoint hence this case is vacuously true.
%     \end{enumerate}
  \end{enumerate} \qed
%\vspace{-6mm} 

\end{proof}

\begin{lemma}
  \label{lem:noexit1}
  If the input ICPPL $(\cF, \cl)$ to Algorithm~\ref{perms} is
  feasible, then the set of hypergraph isomorphism functions that
  defines $(\cF, \cl)$'s feasibility is the same as the set that
  defines $(\cF_j, \cl_j)$'s feasibility, if any.  Secondly, for any
  iteration $j > 0$ of the {\em \bf while} loop, the {\em \bf exit}
  statement in line~\ref{ln:exit1} will not execute.
\end{lemma}
\begin{proof}
  Since $(\cF,\cl)$ is feasible, by Lemma~\ref{lem:feasible}
  $(\cF_j,\cl_j)$ for every iteration $j > 0$ is feasible.  % Therefore,
%   every hypergraph isomorphism $\phi: supp(\cF) \rightarrow V(T)$ that
%   induces $\cl$ on $\cF$ also induces $\cl_{j-1}$ and $\cl_{j}$ on
%   $\cF_{j-1}$ and $\cF_{j}$ respectively, i.e., $\cl_{\phi[\cF_{j-1}]}
%   = \cl_{j-1}$ and $\cl_{\phi[\cF_j]} = \cl_j$. Thus it can be seen
%   that for all $x \in supp(\cF)$, for all $v \in V(T)$ the following
%   hold true.
  Also, every hypergraph isomorphism $\phi: supp(\cF) \rightarrow
  V(T)$ that induces $\cl$ on $\cF$ also induces $\cl_{j}$ on
  $\cF_{j}$, i.e., $\cl_{\phi[\cF_j]} = \cl_j$. Thus it can be seen
  that for all $x \in supp(\cF)$, for all $v \in V(T)$, if $(x,v) \in
  \phi$ then $v \in \cl_{j}(S)$ for all $S \in \cF_{j}$ such that $x
  \in S$.
% the following
%   hold true.
%   \begin{enumerate}[i. ]
%   \item If $(x,v) \in \phi$ then $v \in \cl_{j-1}(S)$ for all $S \in
%     \cF_{j-1}$ such that $x \in S$.
%   \item If $(x,v) \in \phi$ then $v \in \cl_{j}(S)$ for all $S \in
%     \cF_{j}$ such that $x \in S$
%   \end{enumerate}
  In other words, filter 1 outputs a filtered path labeling that
  ``preserves''
  hypergraph isomorphisms of the original path labeling.\\
  Secondly, line~\ref{ln:exit1} will execute iff the exit condition in
  line~\ref{ln:3waycheck}, i.e. failure of three way intersection
  preservation, becomes true in any iteration of the {\em \bf while}
  loop.  Due to Lemma~\ref{lem:invar1} Invariant IV, the exit
  condition does not occur if the input is a feasible ICPPL.\qed

%   such that $\phi(x) = v$ where $v$ is the leaf considered in the
%   first iterations of while. Clearly, $\phi$ is a renaming of
%   vertices in hypergraph $\cF$ to those in hypergraph $\cF^\cl$. Thus
%   the following facts can be observed in every iteration of the loop.

%   \begin{enumerate}[\hspace{2mm}i. ] \vspace{\topshrink}
%   \item all intersection cardinalities are preserved in this path
%     labeling \vspace{\topshrink}
%   \item element $x$ is exclusive in a hyperedge in $\cF$ since $v$ is
%     exclusive in a hyperedge in $\cF^\cl$.
%   \end{enumerate}

%   Thus the exit condition is never rendered true after $x$ and $v$ are
%   removed from their respective hyperedges. \qed

% \noindent
% This proof uses mathematical induction on the number
%   of iterations $j$, $j \ge 0$, of the loop that executed
%   without exiting. The base case, $j = 0$ is obviously true since the
%   input is an ICPPL and the exit condition cannot hold true due to
%   ICPPL property (iii).  Assume the algorithm executes till the end
%   of $j-1$th iteration without exiting at line
%  ~\ref{ln:3waycheck}. Consider the $j$th iteration. From Lemma
%  ~\ref{lem:feasible} we know that $(\cF_j, \cl_j)$ and $(\cF_{j-1},
%   \cl_{j-1})$ are feasible\remove[AS]{and from the proof in lemma
%     lem:invar1 we know that $(\cF_{j-1}, \cl_{j-1})$ satisfies all the
%     invariants defined in the lemma}.  Thus there exists a bijection
%   $\phi: supp(\cF) \rightarrow V(T)$ such that the induced path
%   % labeling on $\cF_{j-1}$ $\cl_{\phi[\cF_{j-1}]} = \cl_{j-1}$.
%   labeling on $\cF_{j}$, $\cl_{\phi[\cF_{j}]}$ and on $\cF_{j-1}$,
%   $\cl_{\phi[\cF_{j-1}]}$ are equal to $\cl_{j}$ and $\cl_{j-1}$
%   respectively.  We need to prove that for any $R, R', R'' \in
%   \cF_{j}$, $|R \cap R' \cap R''| = |\cl_j(R) \cap \cl_j(R') \cap
%   \cl_j(R'')|$.
%   The following are the possible cases that could arise. From argument
%   above, $|\cl_j(R) \cap \cl_j(R') \cap \cl_j(R'')| =
%   |\cl_{\phi[\cF_{j}]}(R) \cap \cl_{\phi[\cF_{j}]} (R') \cap
%   \cl_{\phi[\cF_{j}]} (R'')|$

%   \begin{enumerate}[a $|$]
%   \item {\em None of the sets are new. $R, R', R'' \in \cF_{j-1}$.}
%     We know $(\cF_{j-1}, \cl_{j-1})$ is feasible. Thus $|R \cap R'
%     \cap R''| = |\cl_{j-1}(R) \cap \cl_{j-1}(R') \cap \cl_{j-1}(R'')|
%     = |\cl_{j}(R) \cap \cl_{j}(R') \cap \cl_{j}(R'')|$.
%   \item {\em Only one, say $R$, is a new set.}  Let $R = S_1 \cap S_2$
%     ($S_1, S_2$ are defined in the proof of lemma
%    ~\ref{lem:invar1}). Now we have $|R \cap R' \cap R''| = |S_1 \cap
%     S_2 \cap R' \cap R''| = |\cl_{j-1}(S_1) \cap \cl_{j-1}(S_2) \cap
%     \cl_{j-1}(R') \cap \cl_{j-1}(R'')| = |\cl_{j}(R) \cap \cl_{j}(R')
%     \cap \cl_{j}(R'')|$. Thus proven. If $R$ is any of the other new
%     sets, the same claim can be verified using lemma
%    ~\ref{lem:setminuscard}.
%     % \item []{\bf Case 3:}
%   \item {\em At least two of $R, R', R''$ are new sets.}  The new sets
%     are disjoint hence this case is vacuously true.
%   \end{enumerate}
%   \qed
%   \tnote[E2]{remove the induction proof. just text saying x and v are
%     exclusive in these sets therefore the intersection cardinalities
%     don't change thus all invariants are still true} 
\end{proof}

\xnoindent As a result of Algorithm~\ref{perms} each leaf $v$ in $T$
is such that there is exactly one set in $\cF$ with $v$ as a vertex in
the path assigned to it.  In Algorithm~\ref{leafasgn} we identify
elements in $supp(\cF)$ whose images are leaves in a hypergraph
isomorphism if one exists.  Let $S \in \cF$ be such that $\cl(S)$ is a
path with leaf and $v \in V(T)$ is the unique leaf incident on it.  We
define a new path labeling $\cl_{new}$ such that $\cl_{new}(\set{x}) =
\set{v}$ where $x$ an arbitrary element from $S \setminus \bigcup_{\hS
  \ne S} \hS$. In other words, $x$ is an element present in no other
set in $\cF$ except $S$. This is intuitive since $v$ is present in no
other path image under $\cl$ other than $\cl(S)$.  The element $x$ and
leaf $v$ are then removed from the set $S$ and path $\cl(S)$
respectively. After doing this for all leaves in $T$, all path images
in the new path labeling $\cl_{new}$ except leaf labels (a path that
has only a leaf is called the {\em leaf label} for the corresponding
single element hyperedge or set) are paths from a new pruned tree $T_0
= T \setminus \{v \mid v \text{ is a leaf in }
T\}$. Algorithm~\ref{leafasgn} is now presented with details.


\begin{algorithm}[h]
  \caption{Leaf labeling from an ICPPL {\tt filter\_2($\cF, \cl, T$)}}
  \label{leafasgn}
  \begin{algorithmic}[\lndisplay]
    \STATE $\cF_0 \assign \cF$, $\cl_0(S) \assign \cl(S)$ for all $S \in \cF_0$
    \COMMENT {Path images are such that no two path images share a
      leaf.}
    \STATE $j \assign 1$\\
    \WHILE {there is a leaf $v$ in $T$ and a unique $S_1 \in
      \cF_{j-1}$ such that $v \in \cl_{j-1}(S_1)$ }\label{uniqueleaf}
    \STATE $\cF_j \assign \cF_{j-1} \setminus \{S_1\}$\\
    \STATE for all $S \in \cF_{j-1}$ such that $S \ne S_1$ set
    $\cl_j(S) \assign
    \cl_{j-1}(S)$\\
    \STATE $X \assign S_1 \setminus \bigcup_{S \in \cF_{j-1}, S \ne S_1}S$\\
    \IF{$X$ is empty} \label{xempty} \STATE {\bf exit} \label{ln:exit2} \ENDIF
    \STATE $x \assign $ arbitrary element from $X$\\
    \STATE $\cF_j \assign \cF_j \cup \{\{x\}, S_1 \setminus \{x\}\} $\\
    \STATE $\cl_j(\{x\}) \assign \{v\}$\\
    \STATE $\cl_j(S_1 \setminus \{x\}) \assign \cl_{j-1}(S_1) \setminus \{v\}$\\
    \STATE $j \assign j+1$\\
    \ENDWHILE
    \STATE $\cF' \assign \cF_j$, $\cl' \assign \cl_j$\\
    \RETURN $(\cF', \cl')$
  \end{algorithmic}
\end{algorithm}

\xnoindent
Suppose the input ICPPL $(\cF, \cl)$ is feasible, yet set $X$ in
Algorithm~\ref{leafasgn} is empty in some iteration of the {\bf while}
loop. This will abort our procedure of finding the hypergraph
isomorphism. The following lemma shows that this will not happen.

\begin{lemma}
  \label{lem:xnotempty}
  If the input ICPPL $(\cF, \cl)$ to Algorithm~\ref{leafasgn} is
  feasible, then for all iterations $j > 0$ of the {\em \bf while}
  loop, the {\em \bf exit} statement in line~\ref{ln:exit2} does not
  execute.
\end{lemma}
\begin{proof}
  Assume $X$ is empty for some iteration $j > 0$. We know that $v$ is
  an element of $\cl_{j-1}(S_1)$. Since it is uniquely present in
  $\cl_{j-1}(S_1)$, it is clear that $v \in \cl_{j-1}(S_1) \setminus
  \bigcup_{(S \in \cF_{j-1}) \wedge (S \ne S_1)}\cl_{j-1}(S)$.  Note
  that for any $x \in S_1$ it is contained in at least two sets due to
  our assumption about cardinality of $X$. Let $S_2 \in \cF_{j-1}$ be
  another set that contains $x$. From the above argument, we know $v
  \notin \cl_{j-1}(S_2)$. Therefore there cannot exist a hypergraph
  isomorphism bijection that maps elements in $S_2$ to those in
  $\cl_{j-1}(S_2)$. This contradicts our assumption that the input is
  feasible. Thus $X$ cannot be empty if input is ICPPL and feasible.
  \qed
\end{proof}

\begin{lemma}
  \label{lem:invar3}
  In Algorithm~\ref{leafasgn}, for all $j > 0$, at the end of the
  $j$th iteration of the {\bf while} loop the four invariants given in
  Lemma~\ref{lem:invar1} hold.
\end{lemma}
\begin{proof}
  By Lemma~\ref{lem:xnotempty} we know that set $X$ will not be empty
  in any iteration of the {\em \bf while} loop if input ICPPL $(\cF,
  \cl)$ is feasible and $\cl_j$ is always computed for all $j >
  0$. Also note that removing a leaf from any path keeps the new path
  connected. Thus invariant I is obviously true. In every iteration $j
  > 0$, we remove exactly one element $x$ from one set $S$ in $\cF$
  and exactly one vertex $v$ which is a leaf from one path
  $\cl_{j-1}(S)$ in $T$. This is because as seen in
  Lemma~\ref{lem:xnotempty}, $x$ is exclusive to $S$ and $v$ is
  exclusive to $\cl_{j-1}(S)$. Due to this fact, it is clear that the
  intersection cardinality equations do not change, i.e., invariants
  II, III, IV remain true. On the other hand, if the input ICPPL is
  not feasible the invariants are vacuously true. \qed
\end{proof}

% \textcolor{cyan}{
% \begin{lemma}
%   \label{lem:notfeasibleexit}
%   \tnote[a]{IS THIS CORRECT?}  If input ICPPL $(\cF, \cl)$ is not
%   feasible, then in one of the recursive calls to Algorithm --3--, the
%   {\em \bf exit} statement in line x in Algorithm --1 or line y in
%   Algorithm --2 will get executed.
% \end{lemma}
% }


\xnoindent We have seen two filtering algorithms above, namely,
Algorithm~\ref{perms} {\tt filter\_1} and Algorithm~\ref{leafasgn}
{\tt filter\_2} which when executed serially respectively result in a
new ICPPL on the same universe $U$ and tree $T$. We also proved that
if the input is indeed feasible, these algorithms do indeed output the
filtered ICPPL. Now we present the algorithmic characterization of a
feasible tree path labeling by way of Algorithm~\ref{al:icppl-find-isomorph}.

\xnoindent Algorithm~\ref{al:icppl-find-isomorph} computes a
hypergraph isomorphism $\phi$ recursively using Algorithm~\ref{perms}
and Algorithm~\ref{leafasgn} and pruning the leaves of the input
tree. In brief, it is done as follows. Algorithm~\ref{leafasgn} gives
us the leaf labels in $\cF_2$, i.e., the elements in $supp(\cF)$ that
map to leaves in $T$, where $(\cF_2, \cl_2)$ is the output of
Algorithm~\ref{leafasgn}. All leaves in $T$ are then pruned away. The
leaf labels are removed from the path labeling $\cl_2$ and the
corresponding elements are removed from the corresponding sets in
$\cF_2$. This tree pruning algorithm is recursively called on the
altered hypergraph $\cF'$, path label $\cl'$ and tree $T'$. The
recursive call returns the bijection $\phi''$ for the rest of the
elements in $supp(\cF)$ which along with the leaf labels $\phi'$ gives
us the hypergraph isomorphism $\phi$.  The following lemma formalizes
the characeterization of feasible path labeling.

\begin{algorithm}[h]
  \caption{{\tt get-hypergraph-isomorphism($\cF, \cl, T$)}}
  \label{al:icppl-find-isomorph}
  \begin{algorithmic}[\lndisplay]

    \IF{$T$ is empty}
    \RETURN $\emptyset$\\
    \ENDIF
    \STATE $L \assign \{v \mid v \text{ is a leaf in }      T\}$\\
    \STATE $(\cF_1, \cl_1) \assign$ {\tt filter\_1($\cF, \cl,
      T$)}\\
    \STATE $(\cF_2, \cl_2) \assign$ {\tt filter\_2($\cF_1,
      \cl_1, T$)}\\

    \STATE $(\cF', \cl') \assign (\cF_2, \cl_2)$\\
    \STATE $\phi' \leftarrow \emptyset$

    \FOR {every $v \in L$}
    \STATE $\phi'(x) \assign v$ where $x \in \cl_2^{-1}(\{v\})$
    \COMMENT {Copy the leaf labels to a one to one function $\phi':
      supp(\cF) \rightarrow L$
      }\\
    \STATE Remove $\{x\}$ and $\{v\}$ from $\cF'$, $\cl'$  appropriately\\
    \ENDFOR

    \STATE $T' \assign T \setminus L$

    \STATE $\phi'' \assign$ {\tt
      get-hypergraph-isomorphism($\cF', \cl', T'$)}
    \STATE $\phi \assign \phi'' \cup \phi'$ \\
    \RETURN $\phi$
  \end{algorithmic}
\end{algorithm}

\begin{lemma}
  \label{lem:hyperiso}  %{lem:perm}
  If $(\cF, \cl)$ is an ICPPL from a tree $T$ and
  Algorithm~\ref{al:icppl-find-isomorph}, {\tt
    get-hypergraph-isomorphism ($\cF, \cl, T$)} returns a non-empty
  function, then there exists a hypergraph isomorphism $\phi :
  supp(\cF) \rightarrow V(T)$ such that the $\phi$-induced tree path
  labeling is equal to $\cl$ or $\cl_\phi = \cl$.
\end{lemma}
\begin{proof}
  It is clear that in the end of every recursive call to
  Algorithm~\ref{al:icppl-find-isomorph}, the function $\phi'$ is one
  to one involving all the leaves in the tree passed as input to that
  recursive call. Moreover, by Lemma~\ref{lem:noexit1} and
  Lemma~\ref{lem:xnotempty} it is consistent with the tree path
  labeling $\cl$ passed. The tree pruning is done by only removing
  leaves in each call to the function and is done till the tree
  becomes empty. Thus the returned function $\phi: supp(\cF)
  \rightarrow V(T)$ is a union of mutually exclusive one to one
  functions exhausting all vertices of the tree. In other words, it is
  a bijection from $supp(\cF)$ to $V(T)$ inducing the given path
  labeling $\cl$ and thus a hypergraph isomorphism. \qed
\end{proof}

\begin{theorem}
  \label{th:charac}
  A path labeling $(\cF, \cl)$ on tree $T$ is feasible iff it is an
  ICPPL and Algorithm~\ref{al:icppl-find-isomorph} with $(\cF, \cl,
  T)$ as input returns a non-empty function.
\end{theorem}
\begin{proof}
  From Lemma~\ref{lem:hyperiso}, we know that if $(\cF, \cl)$ is an
  ICPPL and Algorithm~\ref{al:icppl-find-isomorph} with $(\cF, \cl,
  T)$ as input returns a non-empty function, $(\cF, \cl)$ is feasible.
  Now consider the case where $(\cF, \cl)$ is feasible. i.e. there
  exists a hypergraph isomorphism $\phi$ such that $\cl_\phi =
  \cl$. Lemma~\ref{lem:noexit1} and Lemma~\ref{lem:xnotempty} show us
  that filter 1 and filter 2 do not exit if input is feasible. Thus
  Algorithm~\ref{al:icppl-find-isomorph} returns a non-empty function.\qed
\end{proof}

\noindent
{\bf ICPPL when given tree is a path:}
\label{subsec:icpplicpia}
\xnoindent
Consider a special case of ICPPL with the following properties when the tree $T$
is a path.  Hence, all path labels are can be viewed as intervals assigned to the
sets in $\cF$.  It is shown, in \cite{nsnrs09}, that the filtering algorithms outlined
above need only preserve pairwise intersection cardinalities, and higher level
intersection cardinalities are preserved by the Helly Property of intervals.  Consequently,
the filter algorithms do not need to ever evaluate the additional check to {\em \bf exit}.
%\begin{enumerate}
%\item Given tree $T$ is a path. Hence, all path labels are interval labels.
%\item Only pairwise intersection cardinality
%  preservation is sufficient. i.e. property (iii) in ICPPL is not enforced.
%\item The filter algorithms do not have {\em \bf exit} statements.
%\end{enumerate}
%This is called an Intersection Cardinality Preservation Interval
%Assignment (ICPIA) \cite{nsnrs09}. 
This structure and its algorithm is
used in the next section for finding tree path labeling from a
$k$-subdivided star due to this graph's close relationship with
intervals. 


\section{Case of a Special Tree -- The $k$-subdivided star}
\label{sec:ksubdivstar}
In this section we consider the problem of assigning paths from a
$k$-subdivided star $T$ to a given set system $\cF$ such that each set
$X \in \cF$ is of cardinality at most $k+2$.  Secondly, we present our
results only for the case when overlap graph $\bO(\cF)$ is connected.
%The
%overlap graph is well-known from the work of
%\cite{kklv10,nsnrs09,wlh02}.  We use the notation in
%\cite{kklv10}. Recall from Section~\ref{sec:prelims} that hyperedges
%$S$ and $S'$ are said to overlap, denoted by $S \overlap S'$, if $S$ and $S'$
%have a non-empty intersection but neither of them is contained in the
%other. The overlap graph $\bO(\cF)$ is a graph in which the vertices
%correspond to the sets in $\cF$, and the vertices corresponding to the
%hyperedges $S$ and $S'$ are adjacent if and only if they overlap.  Note
%that the intersection graph of $\cF$, $\bI(\cF)$ is different from
%$\bO(\cF)$ and $\bO(\cF) \subseteq \bI(\cF)$.  
A connected component of $\bO(\cF)$ is called an overlap component of
$\cF$.  An interesting property of the overlap components is that any
two distinct overlap components, say $\cO_1$ and $\cO_2$, are such
that any two sets $S_1 \in \cO_1$ and $S_2 \in \cO_2$ are disjoint,
or, w.l.o.g, all the sets in $\cO_1$ are contained within one set in
$\cO_2$.  This containment relation naturally determines a
decomposition of the overlap components into rooted containment trees.
We consider the case when there is only one rooted containment tree,
and we first present our algorithm when $\bO(\cF)$ is connected.  It
is easy to see that once the path labeling to the overlap component in
the root of the containment tree is achieved, the path labeling to the
other overlap components in the rooted containment tree is essentially
finding a path labeling when the target tree is a path: each target
path is a path that is allocated to sets in the root overlap
component.  Therefore, for the rest of this section, $\bO(\cF)$ is a
connected graph. Recall that we also consider the special case when
all hyperedges are of cardinality at most $k+2$.  By definition, a
$k$-subdivided star has a central vertex which we call the {\em root},
and each root to leaf path is called a {\em ray}.  First, we observe
that by removing the root $r$ from $T$, we get a collection of $p$
vertex disjoint paths of length $k+1$, $p$ being the number of leaves
in $T$.  We denote the rays by $R_1, \ldots, R_p$, and the number of
vertices in $R_i$, $i \in [p]$ is $k+2$.  Let
$\seq{v_{i1},\ldots,v_{i(k+2)}=r}$ denote the sequence of vertices in
$R_i$, where $v_{i1}$ is the leaf. Note that $r$ is a common vertex to
all $R_i$.
  


\subsection{Description of the Algorithm}
In this section the given hypergraph $\cF$, the $k$-subdivided star
and the root of the star are denoted by $\cO$, $T$ and vertex $r$,
respectively.  In particular, note that the vertices of $\cO$
correspond to the sets in $\cF$, and the edges correspond to the
overlap relation.

\noindent
For each hyperedge $X \in \cO$, we will maintain a 2-tuple of non-negative
 numbers $\seq{p_1(X), p_2(X)}$.  The numbers satisfy the property that
 $p_1(X) + p_2(X) \leq |X|$, and at the end of path labeling, for each
 $X$, $p_1(X) + p_2(X) = |X|$.  This signifies the algorithm tracking
 the lengths of subpaths of the path assigned to $X$ from at most two
 rays. We also maintain another parameter called the {\em residue} of
 $X$ denoted by $s(X)=|X| - p_1(X)$. This signifies the residue path
 length that must be assigned to $X$ which must be from another
 ray. For instance, if $X$ is labeled a path from only one ray, then
 $p_1(X) = |X|$, $p_2(X) = 0$ and $s(X) = 0$.

 \xnoindent The algorithm proceeds in iterations, and in the $i$-th
 iteration, $i > 1$, a single hyperedge $X$ that overlaps with a
 hyperedge that has been assigned a path is considered.  At the
 beginning of each iteration hyperedges of $\cO$ are classifed into
 the following disjoint sets.
 \begin{enumerate}
 \item [$\cL_1^i$] {\em Labeled without $r$.} Those that have been
   labeled with a path which does not contain $r$ in one of the
   previous iterations.\\  $\cL_1^i = \set{ X \mid p_1(X) = |X| \text{ and
     } p_2(X) = 0 \text{ and } s(X) = 0, X \in \cO}$
 \item [$\cL_2^i$] {\em Labeled with $r$.} Those that have been labeled
   with two subpaths of $\cl(X)$ containing $r$ from two different rays
   in two previous iterations.\\ $\cL_2^i = \set{X \mid 0 < p_1\left(X\right),
     p_2\left(X\right) < |X|=p_1(X)+p_2(X) \text{ and } s(X) = 0, X \in \cO}$
   \item [$\cT_1^i$] {\em Type 1 / partially labeled.} Those that have
     been labeled with one path containing $r$ from a single ray in one
     of the previous iterations. Here, $p_1(X)$ denotes the length of
     the subpath of $\cl(X)$ that $X$ has been so far labeled
     with. \\
     $\cT_1^i = \set{ X \mid 0 < p_1(X) < |X| \text{ and } p_2(X) = 0
       \text{ and } s(X) = |X|-p_1(X), X \in \cO}$
   \item [$\cT_2^i$] {\em Type 2 / not labeled.} Those that have not been
     labeled with a path in any previous iteration.\\
     $\cT_2^i = \set{ X \mid p_1(X) = p_2(X) = 0 \text{ and } s(X) = |X|,
       X \in \cO}$
 \end{enumerate}
% \vspace{-2mm}
% \begin{align*}
%   \cO &= \cL_1^i \cup \cL_2^i \cup \cT_1^i \cup \cT_2^i, \text{    } \forall i \in [p]\\
%   \cO_i &= \cT_1^i \cup \cT_2^i, \text{    } \forall i \in [p]
% \end{align*}

\noindent
The set $\cO_i$ refers to the set of hyperedges $\cT_1^i \cup \cT_2^i$
in the $i$th iteration.  Note that $\cO_1 = \cO$.  In the $i$th
iteration, hyperedges from $\cO_i$ are assigned paths from $T$ using
the following rules. Also the end of the iteration, $\cL_1^{i+1},
\cL_2^{i+1}, \cT_1^{i+1}, \cT_2^{i+1}$ are set to $\cL_1^{i},
\cL_2^{i}, \cT_1^{i}, \cT_2^{i}$ respectively, along with some
case-specific changes mentioned in the rules below.

\noindent
\begin{enumerate}[I.]
\item {\bf Iteration 1:} Let $S=\{X_1,\ldots,X_s\}$ denote the
  super-marginal hyperedges from $\cO_1$.  If $|S|=s \neq p$, then
  exit reporting failure.  Else, assign to each $X_j \in S$, the path
  from $R_j$ such that the path contains the leaf in $R_j$.  This path
  is refered to as $\cl(X_j)$.  Set $p_1(X_j)=|X|, p_2(X_j)=s(X_j)=0$.
  Hyperedges in $S$ are not added to $\cO_2$ but are added to
  $\cL_1^2$ and all other hyperedges are added to
  $\cO_2$.
\item {\bf Iteration $i$:} Let $X$ be a hyperedge from $\cO_i$ such
  that there exists $Y \in \cL_1^i \cup \cL_2^i$ and $X \overlap
  Y$. Further let $Z \in \cL_1^i \cup L_2^i$ such that $Z \overlap Y$.
  If $X \in \cT_2^i$, and if there are multiple $Y$ candidates then
  any $Y$ is selected.  On the other hand, if $X \in \cT_1^i$, then
  $X$ has a partial path assignment, $\cl'(X)$ from a previous
  iteration, say from ray $R_j$. Then, $Y$ is
  chosen such that $X \cap Y$ has a non-empty intersection with a ray
  different from $R_j$.  The key things that are done in assigning a
  path to $X$ are as follows. The {\em end} of path $\cl(Y)$ where
  $\cl(X)$ would overlap is found, and then based on this the
  existence of a feasible assignment is decided.  It is important to
  note that since $X \overlap Y$, $\cl(X) \overlap \cl(Y)$ in any
  feasible assignment.  Therefore, the notion of the {\em end} at
  which $\cl(X)$ and $\cl(Y)$ overlap is unambiguous, since for any
  path, there are two end points.
  \begin{enumerate}
  \item \label{iendpoint} {\em End point of $\cl(Y)$ where $\cl(X)$ overlaps
      depends on $X \cap Z$:} If $X \cap Z
    \neq \emptyset$, then $\cl(X)$ has an overlap of $|X \cap Y|$ at that
    end of $\cl(Y)$ at which $\cl(Y)$ and $\cl(Z)$ overlap.  If $X
    \cap Z = \emptyset$, then $\cl(X)$ has an overlap of $|X \cap Y|$ at
    that end of $\cl(Y)$ where $\cl(Y)$ and $\cl(Z)$ do not intersect.
  \item {\em Any path of length $s(X)$ at the appropriate end contains
      $r$:} If $X \in \cT_1^i$ then after finding the appropriate end
    as in step~\ref{iendpoint} this the unique path of length $s(X)$
    should end at $r$.  If not, we exit reporting failure.  Else,
    $\cl(X)$ is computed as union of $\cl'(X)$ and this path. If any
    three-way intersection cardinality is violated with this new
    assignment, then exit, reporting failure.  Otherwise, $X$ is added
    to $\cL_2^{i+1}$.  On the other hand, if $X \in \cT_2^i$, then
    after step~\ref{iendpoint}, $\cl(X)$ or $\cl'(X)$ is unique up to
    the root and including it. Clearly, the vertices $\cl(X)$ or
    $\cl'(X)$ contains depends on $|X|$ and $|X \cap Y|$.  If any
    three way intersection cardinality is violated due to this
    assignment, exit, reporting failure.  Otherwise,
    $p_1(X)$ is updated as the length of the assigned path, and $s(X)
    = |X|-p_1(X)$.  If $s(X) > 0$, then $X$ is added to $\cT_1^{i+1}$.
    If $s(X)=0$, then $X$ is added to $\cL_1^{i+1}$.
  \item {\em The unique path of length $s(X)$ overlapping at the
      appropriate end of $Y$ does not contain $r$:} In this case,
    $\cl(X)$ is updated to include this path.  If any three way
    intersection cardinality is violated, exit, reporting failure.
    Otherwise, update $p_1(X)$ and $p_2(X)$ are appropriate, $X$ is
    added to $\cL_1^{i+1}$ or $\cL_2^{i+1}$, as appropriate.
  \end{enumerate}
\end{enumerate}

\noindent {\bf Proof of Correctness and Analysis of Running Time:} It
is clear that the algorithm runs in polynomial time, as at each step,
at most three-way intersection cardinalities need to be checked.
Further, finding super-marginal hyperedges can also be done in
polynomial time, as it involves considering the overlap regions and
checking if the inclusion partial order contains a single minimal
element.  In particular, once the super-marginal edges are identified,
each iteration involes finding the next hyperedge to consider, and
testing for a path to that hyperedge.  To identify the next hyperedge
to consider, we consider the breadth first layering of the hyperedges
with the zeroeth layer consisting of the super-marginal hyperedges.
Since $\cO$ is connected, it follows that all hyperedges of $\cO$ will
be considered by the algorithm.  Once a hyperedge is considered, the
path to be assigned to it can also be computed in constant time.  In
particular, in the algorithm the path to be assigned to $X$ depends on
$\cl(Y), \cl(Z)$, $s(X)$ and the presence or absence of $r$ in the
candidate partial path $\cl'(X)$.  Therefore, once the super-marginal
edges are identified, the running time of the algorithm is linear in
the size of the input.  By the technique used for constructing prime
matrices \cite{wlh02}, the super-marginal edges can be found in linear
time in the input size.  Therefore, the algorithm can be implemented
to run in linear time in the input size.

\noindent
The proof of correctness uses the following main properties:
\begin{enumerate}
\item The $k$-subdivided star has a very symmetric structure.  This
  symmetry is quantified based on the following observation -- either
  there are no feasible path labelings of $\cO$ using paths from $T$,
  or there are exactly $p!$ feasible path labelings.  In other words,
  there is either no feasible assignment, or effectively a unique
  assignment modulo symmetry.
\item The $p$ super-marginal hyperedges, if they exist, will each be
  assigned a path from distinct rays, and each such path contains the
  leaf.
\item For a candidate hyperedge $X$, the partial path assignment
  $\cl'(X)$ is decided by its overlap with $\cl(Y)$ and cardinality of
  intersection with $\cl(Z)$.
\end{enumerate}
These properties are formalized as follows:
\begin{lemma}
  \label{lem:sup-mar}
  If $X \in \cF$ is super-marginal and $\cl$ is a feasible tree path
  labeling to tree $T$, then $\cl(X)$ will contain a leaf in $T$.
\end{lemma}
\begin{proof}
  Suppose $X \in \cF$ is super-marginal and $(\cF, \cl)$ is a feasible
  path labeling from $T$.  Assume $\cl(X)$ does not have a leaf.  Let
  $R_i$ be one of the rays (or the only ray) $\cl(X)$ is part of.
  Since $X$ is in a connected overlap component, there exists $Y_1 \in
  \cF$ and $X \nsubseteq Y_1$ such that $Y_1 \overlap X$ and $Y_1$ has
  at least one vertex closer to the leaf in $R_i$ than any vertex in
  $X$. Similarly with the same argument there exists $Y_2 \in \cF$
  with same subset and overlap relation with $X$ except it has has at
  least one vertex farther away from the leaf in $R_i$ than any vertex
  in $X$. Clearly $Y_1 \cap X$ and $Y_2 \cap X$ cannot be part of same
  inclusion chain which contradicts that assumption $X$ is
  super-marginal. Thus the claim is proved.\qed
\end{proof}
\begin{lemma}
  If $\cO$ does not have any super-marginal edges, then in any
  feasible path labeling $\cl$ of $\cO$ with paths from $T$ is such
  that, for any hyperedge $X$ for which $\cl(X)$ contains a leaf, $|X|
  \geq k+3$.
\end{lemma}
\begin{proof}
  The proof of this lemma is by contradiction.  Let $X$ be a
  hyperedges such that $|X| \leq k+2$ and that $\cl(X)$ has a leaf.
  This implies that the overlap regions with $X$, which are captured by
  the overlap regions with $\cl(X)$, will form a single inclusion
  chain. This shows that $X$ is a marginal hyperedge which
  contradicts the assumption that $\cO$ does not have super-marginal
  hyperedges. \qed
\end{proof}
This lemma is used to prove the next lemma for the case when for all
$X \in \cO$, $|X| \leq k+2$.  The proof is left out as it just uses
the previous lemma and the fact that the hyperedges in $X$ have at
most $k+2$ elements.
\begin{lemma}
  If there is a feasible path labeling for $\cO$ in $T$, then there
  are exactly $p$ super-marginal hyperedges.
\end{lemma}
These lemmas now are used to prove the following theorem.
\begin{theorem}
  Given $\cO$ and a $k$-subdivided star $T$, the above algorithm
  decides correctly if there is a feasible path labeling $\cl$.
\end{theorem}
\begin{proof} {\em Outline.}
%   If the algorithm outputs a path labeling $\cl$, then it is clear
%   that it is an ICPPL. The reason is that the algorithm checks that
%   three-way intersection cardinalities are preserved in each iteration
%   which ensures \icpplpr~\ref{pr:iii}. Moreover, it is clear that
%   $\cl(X)$ for any $X \in \cO$ is computed by maintaining
%   \icpplpr~\ref{pr:i} and \icpplpr~\ref{pr:ii}. For such a labeling
%   $\cl$, the proof that it is feasible is by induction on $k$. What
%   needs to be shown is that Algorithm~\ref{al:icppl-find-isomorph}
%   successfully runs on input $(\cO, \cl)$. In base case $k=0$, $T$ is
%   a star. The claim is clear by observing that after Filter 1 and one
%   iteration of Filter 2, all the leaves have found their pre-images
%   from support $\cO$.  Therefore, in the induction step, it is clear
%   that after Filter 1 and one iteration of Filter 2, the leaves are
%   assigned pre-images.  Removing the leaves from $T$ and the
%   pre-images from support of $\cO$, results in an ICPPL to a
%   $(k-1)$-subdivided star.  Now we apply the induction hypothesis, and
%   we get a isomorphism between the hypergraph $\cO$ and $\cO^\cl$.
  If the algorithm outputs a path labeling $\cl$, then it is clear
  that it is an ICPPL. The reason is that the algorithm checks that
  three-way intersection cardinalities are preserved in each iteration
  which ensures ICPPL \icpplpr~\ref{pr:iii}. Moreover, it is clear that
  $\cl(X)$ for any $X \in \cO$ is computed by maintaining
  ICPPL \icpplpr~\ref{pr:i} and ICPPL \icpplpr~\ref{pr:ii}. For such a labeling
  $\cl$, the proof that it is feasible is by induction on $k$. What
  needs to be shown is that Algorithm~\ref{al:icppl-find-isomorph}
  successfully runs on input $(\cO, \cl)$. In base case $k=0$, $T$ is
  a star. Also every set is at most size 2 ($k+2$) size and thus
  overlaps are at most 1. If two paths share a leaf in {\tt filter\_1}
  one must be of length 2 and the other of length 1. Thus the exit
  condition is not met. Now it is trivial that the exit condition in
  {\tt filter\_2} is also not met. Thus claim proven for base case.
  Now assume the claim to be true when target tree is a
  $(k-1)$-subdivided star. Consider the case of a $k$-subdivided
  star. It is clear that after {\tt filter\_1} and one iteration of
  {\tt filter\_2}, the leaves are assigned pre-images.  Removing the
  leaves from $T$ and the pre-images from support of $\cO$, results in
  an ICPPL to a $(k-1)$-subdivided star.  Now we apply the induction
  hypothesis, and we get a isomorphism between the hypergraph $\cO$
  and $\cO^\cl$.

  \noindent
  In the reverse direction if there is a feasible path labeling $\cl$,
  then we know that $\cl$ is unique up to isomorphism.  Therefore,
  again by induction on $k$ it follows that the algorithm finds $\cl$.
  \qed
\end{proof}


% \noindent
% Consider the overlap graph $\bO(\cF)$ of the given hypergraph
% $\cF$. Let $S_{sm} \in \cF$ be such that it is a super-marginal
% hyperedge.  Algorithm~\ref{al:ktree-label} uses $S_{sm}$ along with
% the overlap graph $\bO(\cF)$ to calculate the feasible tree path
% labeling to the $k$-subdivded tree $T$.
%
% \begin{algorithm}[h]
%   \caption{{\tt compute-ksubtree-path-labeling($X, \cF, T$)}}
%   \label{al:ktree-label}
%   \begin{algorithmic}[\lndisplay]
%     \IF{$X = S_{sm}$}
%     \STATE -- TBD --\\
%     \ELSE
%     \STATE -- TBD --\\
%     \ENDIF
%
%   \end{algorithmic}
% \end{algorithm}


% \subsection{temp section from nsnsr09}

% \begin{algorithm}
%   \caption{Basic step in an algorithm to find an ICPIA for a prime
%     matrix $M'$}
%   \label{ds-algo}
%   ICPIA(Set $S$, Integer $p > 0$)

%   \noindent
%   /* {\tt $S \cap S^i \not = \phi$ for some $i \in \{1,\ldots,p\}$,
%     but $S \not\subseteq S^i$, $S^i \not\subseteq S$.  \\ Assigns to
%     $S$ an interval $I$ such that $\{I^1,\ldots,I^p,I\}$ forms an
%     ICPIA for $\{S^1,\ldots,S^p,S\}$.} */
%   \begin{algorithmic}
%     \STATE Let $|S \cap S^i| = z$. \\
%     \STATE Let $I_l$ be the interval such that $|I_l \cap I^i| = z$, $|I_l| = |S|$ and the $z$ common elements are the smallest elements of $I^i$. \\
%     \STATE Let $I_r$ be the interval such that $|I_r \cap I^i| = z$, $|I_r| = |S|$, and the $z$ common elements are the largest elements of $I^i$. \\
%     \IF {$p == 1$}
%     \STATE Assign $I_l$ to $S$ \\
%     \STATE /* {\tt In this case, $I_r$ could also be assigned to $S$.  This will yield the {\bf other} ICPIA} */\\
%     \ELSE \IF{$|I_l \cap I^q| = |S \cap S^q|$ for each $q \in
%       \{1,\ldots,p\}$}
%     \STATE Assign $I_l$ to $S$ and exit.\\
%     \ENDIF \IF{$|I_r \cap I^q| = |S \cap S^q|$ for each $q \in
%       \{1,\ldots,p\}$}
%     \STATE Assign $I_r$ to $S$ and exit.\\
%     \ENDIF \ENDIF \STATE Report no ICPIA and exit.
%   \end{algorithmic}
% \end{algorithm}


% \begin{theorem}
%   Algorithm~\ref{ds-algo} outputs an ICPIA to a prime matrix $M'$
%   iff there is an ICPIA for $M'$.
% \end{theorem}
% \begin{proof}
%   The only-if part of the theorem is straightforward.  We now show
%   that if there is an ICPIA for $M'$, then Algorithm~\ref{ds-algo}
%   will indeed discover it.  The key fact is that in $M'$ for each
%   set $S$, there is another set $T \in M'$ such that $S \cap T \not
%   = \phi$, and $S$ and $T$ are not contained in each other.  Due to
%   this fact, there are exactly two ICPIAs for $M'$.  The two
%   distinct ICPIAs differ based on the interval assigned to $S_1$,
%   see Algorithm ~\ref{ds-algo}.  If $I_l$ is assigned to $S_1$, then
%   we get one, and the other ICPIA is obtained by assigning $I_r$ to
%   $S_1$.  For each subsequent set, say $S^j$, the interval to be
%   assigned is forced.  It is forced due to the fact that the
%   interval assigned to $S^j$ is based on the interval assigned to
%   $S^i$, where $S^i \cap S^j \not = \phi$, and $S^i \not\subseteq
%   S^j$, and $S^j \not\subseteq S^i$.  Given the fact that the
%   algorithm is an exact implementation of these observations, it
%   follows that Algorithm~\ref{ds-algo} finds an ICPIA if there is
%   one.
% \end{proof}




% \section {Acknowlegements}

% \section {Bibliography}
% \bibliographystyle{plainnat}
\bibliographystyle{alpha} %to have only [i] type of citation
% \bibliographystyle{../lib/llncs2e/splncs03} % current LNCS BibTeX style with aphabetic sorting
\bibliography{../lib/cop-variants}


% \pagebreak
\appendix

% \section{Appendix}
% {\bf }

% \vspace{7mm}
% \noindent
% {\bf:}

% \vspace{7mm}
% \noindent
% {\bf:}
% \begin{proof}[Proof of Lemma~\ref{lem:invar1} (details)]
%   :
%   \begin{enumerate}[\textreferencemark]
%   \item {\em Invariant III}
%   \item [Case 3:] {\em Invariant IV}
%     \begin{enumerate}
%     \item [Case 3.2:] {\em Only R is a new set:}\\
%       If $R = S_1 \cap S_2$, Consider, $|\cl_{j-1}(S_1) \cap
%       \cl_{j-1}(S_2) \cap \cl_{j-1}(R') \cap \cl_{j-1}(R'')|$. We
%       know from lemma~\ref{lem:fourpaths} that the intersection of
%       these four paths is same as the intersection of three distinct
%       paths among the four.  Let us call these four paths $P_1,P_2,
%       P_3,P_4$ and without loss of generality, let it be that
%       $\displaystyle \cap_{i=1}^4 P_i = \cap_{i=1}^3 P_i$. Further
%       $|\cap_{i=1}^4 P_i|=| S_1 \cap S_2 \cap R'|$ by the invariant
%       IV of the induction hypothesis.  Therefore, it follows that
%       $|\cap_{i=1}^4 P_i| \geq |S_1 \cap
%       S_2 \cap R' \cap R''|$.\\
%       \begin{comment}
%         Next, we write $\displaystyle |\cap_{i=1}^4 P_i| = |P_4| +
%         |\displaystyle \cap_{i=1}^3 P_i| - |P_4 \cup \cap_{i=1}^3
%         P_i|$. Clearly $\displaystyle P_4 \cup \cap_{i=1}^3 P_i =
%         P_4$.  By induction hypothesis Invariant I and IV, We can
%         now write $\displaystyle |\cap_{i=1}^4 P_i| = |S_4| +
%         |\cap_{i=1}^3 S_i| - |S_4|$.  Since $\cl_{j-1}$ is an ICPPL
%         in which $S_4$ is mapped to $P_4$, and $P_4$ contains
%         $\displaystyle \cap_{i=1}^3 P_i$, it follows that
%         $|S_4|=|P_4| \geq|\cap_{i=1}^3 P_i| = |S_1 \cap S_2 \cap R'|
%         \geq |S_1 \cap S_2 \cap R' \cap R''|$.  Therefore,
%         $\displaystyle |\cap_{i=1}^4 P_i| \leq |S_1 \cap S_2 \cap R'
%         \cap R''|$, and equality of these two terms follows because
%         we have also proved the inequality in the opposite
%         direction. It now follows that $|\cl_j(R) \cap \cl_j(R')
%         \cap \cl_j(R'')| = |\cl_{j-1}(S_1) \cap \cl_{j-1}(S_2) \cap
%         \cl_{j-1}(R') \cap \cl_{j-1}(R'')| = |\cap_{i=1}^4 P_i| =
%         |S_1 \cap S_2 \cap R' \cap R''| = |R \cap R' \cap
%         R''|$. This completes induction hypothesis in this
%         case. \\
%       \end{comment}
%       If $R = S_1 \setminus S_2$, a similar argument using Lemma
%       ~\ref{lem:setminuscard} and the induction hypothesis completes
%       the proof of this case.\\
%       Thus Invariant IV proven.
%     \end{enumerate}
%   \end{enumerate}
% \end{proof}

% % \vspace{7mm}
% \noindent
% % {\bf:}
% \begin{proof}[Proof of lemma~\ref{lem:invar3} (alternate)]
%   \xnoindent For the rest of the proof we use mathematical induction
%   on the number of iterations $j$. As before, the term ``new sets''
%   will refer to the sets added in $\cF_j$ in the $j$th iteration,
%   i.e. $S_1 \setminus \{x\}$ and $\{x\}$ as defined in line
%   ~\ref{uniqueleaf}.\\
%   For $\cF_0, \cl_0$ all invariants hold because it is output from
%   algorithm~\ref{perms} which is an ICPPL. Hence base case is
%   proved.  Assume the lemma holds for the $j-1$th
%   iteration. Consider $j$th iteration.  \xnoindent
%   \begin{enumerate}[\textreferencemark]
%   \item {\em Invariant} I/II
%     \begin{enumerate}[{I/II}a $|$]
%     \item {\em $R$ is not a new set.} If $R$ is in $\cF_{j-1}$, then
%       by induction hypothesis this case is trivially proven.
%     \item {\em $R$ is a new set.} If $R$ is in $\cF_{j}$ and not in
%       $\cF_{j-1}$, then it must be one of the new sets added in
%       $\cF_j$. Removing a leaf $v$ from path $\cl_{j-1}(S_1)$
%       results in another path. Moreover, $\{v\}$ is trivially a
%       path. Hence regardless of which new set $R$ is, by definition
%       of
%       $\cl_j$, $\cl_{j}(R)$ is a path. Thus invariant I is proven.\\
%       We know $|S_1| = |\cl_{j-1}(S_1)|$, due to induction
%       hypothesis. Therefore $|S_1 \setminus \{x\}| = |\cl_{j-1}(S_1)
%       \setminus \{v\}|$. This is because $x \in S_1$ iff $v \in
%       \cl_{j-1}(S_1)$. If $R = \{x\}$, invariant II is trivially
%       true. Thus invariant II is proven.
%     \end{enumerate}
%   \item {\em Invariant} III
%     \begin{enumerate} [{III}a $|$]
%     \item {\em $R$ and $R'$ are not new sets.} Trivially true by
%       induction hypothesis.
%     \item {\em Only one, say $R$, is a new set.}  By definition,
%       $\cl_{j-1}(S_1)$ is the only path with $v$ and $S_1$ the only
%       set with $x$ in the previous iteration, hence $|R' \cap (S_1
%       \setminus \{x\})| = |R' \cap S_1|$ and $|\cl_{j-1}(R') \cap
%       (\cl_{j-1}(S_1) \setminus \{v\})| = |\cl_{j-1}(R') \cap
%       \cl_{j-1}(S_1)|$ and $|R' \cap \{x\}| = 0$, $|\cl_{j-1}(R')
%       \cap \{v\}| = 0$. Thus case proven.
%     \item {\em $R$ and $R'$ are new sets.} By definition, the new
%       sets and their path images in path label $l_j$ are disjoint so
%       $|R \cap R'| = |l_j(R) \cap l_j(R)| = 0$. Thus case proven.
%     \end{enumerate}
%   \item {\em Invariant} IV
%     \begin{enumerate}[{IV}a $|$]
%     \item {\em $R$, $R'$ and $R''$ are not new sets.} Trivially true
%       by induction hypothesis.
%     \item {\em Only one, say $R$, is a new set.}  By the same
%       argument used to prove invariant III, $|R' \cap R'' \cap (S_1
%       \setminus \{x\})| = |R' \cap R'' \cap S_1|$ and
%       $|\cl_{j-1}(R') \cap \cl_{j-1}(R'') \cap (\cl_{j-1}(S_1)
%       \setminus \{v\})| = |\cl_{j-1}(R') \cap \cl_{j-1}(R'') \cap
%       \cl_{j-1}(S_1)|$. Since $R', R'', S_1$ are all in $\cF_{j-1}$,
%       by induction hypothesis of invariant IV, $|R' \cap R'' \cap
%       S_1| = |\cl_{j-1}(R') \cap \cl_{j-1}(R'') \cap
%       \cl_{j-1}(S_1)|$.  Also, $|R' \cap R'' \cap \{x\}| =
%       |\cl_{j-1}(R') \cap \cl_{j-1}(R'') \cap \{v\}|$ = 0.
%     \item {\em At least two of $R, R', R''$ are new sets.}  If two
%       or more of them are not in $\cF_{j-1}$, then it can be
%       verified that $|R \cap R' \cap R''| = |\cl_j(R) \cap \cl_j(R')
%       \cap \cl_j(R'')|$ since the new sets in $\cF_j$ are disjoint.
% %       \tnote[AS]{The following is not correct. While loop only
% %         handles one leaf at a time. Deleted} \remove[AS]{ or as
% %         follows:
% %         assuming $R, R' \notin \cF_{j-1}$ and new sets are derived
% %         from $S_1, S_2 \in \cF_{j-1}$ with $x_1, x_2$ exclusively
% %         in
% %         $S_1, S_2$, $\cl_j(\{x_1\})=\{v_1\}, \cl_j(\{x_2\}) =
% %         \{v_2\}$
% %         where $ \{x_1\}, \{x_2\} \in \cF_j $ thus $v_1, v_2$ are
% %         exclusively in $\cl_j(\{x_1\})$, $\cl_j(\{x_2\})$
% %         respectively. It follows that $|R \cap R' \cap R''| =$ $
% %         |(S_1
% %         \setminus \{x_1\}) \cap (S_2 \setminus \{x_2\}) \cap R''|
% %         =$ $
% %         |S_1 \cap S_2 \cap R''| = $ $|\cl_{j-1}(S_1) \cap
% %         \cl_{j-1}(S_2) \cap \cl_{j-1}(R'')| = |(\cl_{j-1}(S_1)
% %         \setminus \{v_1\}) \cap (\cl_{j-1}(S_2) \setminus \{v_2\})
% %         \cap \cl_{j-1}(R'')| = |\cl_j(R) \cap \cl_j(R') \cap
% %         \cl_j(R'')|$}
%       Thus invariant IV is also proven.
%     \end{enumerate}
%   \end{enumerate}
%   \qed
% \end{proof}

% \begin{proof} [Proof of theorem~\ref{th:perm} (alternate)]
%   We find $\phi$ part by part by running algorithms~\ref{perms}
%   and~\ref{leafasgn} one after the other in a loop. After each
%   iteration we calculate an exclusive subset of the bijection
%   $\phi$, namely that which involves all the leaves of the tree in
%   that iteration. Then all the leaves are pruned off the tree before
%   the next iteration. The loop terminates when the pruned tree has
%   no nodes at which point the rest of the bijection is obvious, thus
%   completing $\phi$. This is the brief outline of the algorithm. Now
%   we see it in detail below.

%   \xnoindent First, the given ICPPL $(\cF, \cl)$ and tree $T$ are
%   given as input to Algorithm~\ref{perms}. This yields a
%   ``filtered'' ICPPL as the output which is input to
%   Algorithm~\ref{leafasgn}.  Let the output of
%   Algorithm~\ref{leafasgn} be $(\cF',\cl')$. We define a bijection
%   $\phi_1: Y_1 \rightarrow V_1$ where $Y_1 \subseteq supp(\cF)$ and
%   $V_1 = \{v \mid v \text{ is a leaf in } $T$\}$.  It can be
%   observed that the output of Algorithm~\ref{leafasgn} is a set of
%   path assignments to sets and one-to-one assignment of elements of
%   $U$ to each leaf of $T$. These are defined below as $\cl_1$ and
%   $\phi_1$ respectively.

%   \begin{align*}
%     \cl_1(S) = \cl'(S), &\text{ when $\cl'(S)$ has non-leaf vertices} \\
%     \phi_1(x) = v,  &\text{ when }\cl'(S) = \{v\}, v \in V_1,\\
%     &\text{ and } S = \{x\}
%   \end{align*}
%   Consider the tree $T_1 = T[V(T) \setminus V_0]$, i.e. it
%   isomorphic to $T$ with its leaves removed. Let $U_1$ be the
%   universe of the subsystem that is not mapped to a leaf of $T$,
%   i.e. $U_1 = supp(\cF) \setminus Y_1$ .

% %   To be precise, it would be of the form $\cB_0 =
% %   \cA_0 \cup \cL_0$. The leaf assignments are defined in $\cL_0
% %   = \{ (x_i,v_i) \mid x_i \in U, v_i \in T, x_i \ne x_j, v_i \ne
% %   v_j, i
% %   \ne j, i,j \in [k] \}$ where $k$ is the
% %   number of leaves in $T$. The path assignments are defined in
% %   $\cA_0
% %   \subseteq \{(S_i',P_i') \mid S_i' \subseteq U_0, P_i' \text{ is
% %     a path
% %     from } T_0\}$
% %   where $T_0$ is the tree obtained by removing all the
% %   leaves in $T$ and
% %   $U_0 = U \setminus \{ x \mid x \text{ is assigned to
% %     a leaf in }\cL_0 \}$.
%   \xnoindent Let $\cF_1$ be the set system induced by $\cF'$ on
%   universe $U_1$.  Now we have a subproblem of finding the
%   hypergraph isomorphism for $(\cF_1, \cl_1)$ with tree $T_1$.
% %   for the path assignment $\cA_0$ which has paths from tree
% %   $T_0$ and sets from universe $U_0$. Now we repeat the procedure
% %   and
% %   the path assignment $\cA_0$ and tree $T_0$
%   \xnoindent Now we repeat Algorithm~\ref{perms} followed by
%   Algorithm ~\ref{leafasgn} on $(\cF_1, \cl_1)$ with tree $T_1$. As
%   before we define $l_2$ in terms of $l_1$, $\phi_2$ in terms of
%   $V_2 = \{v \mid v \text{ is a leaf in } T_1\}$, prune the tree
%   $T_1$ to get $T_2$ and so on.  Thus in the $i$th iteration, $T_i$
%   is the pruned tree, $\cl_i$ is a feasible path labeling to $\cF_i$
%   if $(\cF_{i-1}, \cl_{i-1})$ is feasible, $\phi_i$ is the leaf
%   labeling of leaves of $T_{i-1}$. Continue this until some $d$th
%   iteration for the smallest value $d$ such that $T_d$ is
%   empty. From the lemma~\ref{lem:invar1} and~\ref{lem:invar3} we
%   know that $(\cF_d, \cl_d)$ is an ICPPL.

%   \xnoindent Now we have exactly one bijection $\phi_j$, $j \in [d]$
%   defined for every element $x \in U$ into some vertex $v \in
%   V(T)$. The bijection for the ICPPL, $\phi: U \rightarrow V(T)$ is
%   constructed by the following definition.  \vspace{\topshrink}
%   \begin{align*}
%     \phi(x) &= \phi_i(x) \\
%     &\text{ where $x$ is in the domain of } \phi_i, i \in [d] %[d+1]
%   \end{align*}
%   It can be verified easily that $\phi$ is the required hypergraph
%   isomorphism. Thus the theorem is proven.  \qed
% \end{proof}



\end{document}





