%%
% Author: N S Narayanaswamy and Anju Srinivasan
%%

\documentclass[11pt,letter]{../lib/llncs} 

\usepackage{fullpage} %% llncs
\usepackage{latexsym}
\usepackage{amssymb}
\usepackage{amsfonts}
\usepackage{amsmath}
\usepackage{comment}
\usepackage{epsfig}
\usepackage{graphicx}
\usepackage{epstopdf}
\usepackage{algorithm}
\usepackage{algorithmic}
\usepackage{enumerate}
\usepackage{textcomp}
\usepackage{pifont}
%\usepackage{natbib}
%%%%%%%%%%%%%%%%%%%%
%   TrackChanges   %
%
% Toggle these declarations for Review or Final version
% 
\usepackage[inline]{../lib/trackchanges}     % uncomment to see review comments
%\usepackage[finalnew]{../lib/trackchanges}   % uncomment to see no review notes
%
%
% finalold
%   Ignore all of the edits. 
%   The document will look as if the edits had not been added.
% finalnew
%   Accept all of the edits. 
%   Notes will not be shown in the final output.
% footnotes
%   Added text will be shown inline. Removed text and notes will be shown as footnotes. 
%   This is the default option.
% margins
%   Added text will be shown inline. Removed text and notes will be
%   shown in the margin. Margin notes will be aligned with the edits when possible.
% inline
%   All changes and notes will be shown inline.
% End TrackChanges %
%%%%%%%%%%%%%%%%%%%%

\DeclareMathAlphabet{\mathpzc}{OT1}{pzc}{m}{it}
\DeclareMathAlphabet{\mathcalligra}{T1}{calligra}{m}{n}

%\bibpunct{[}{]}{;}{a}{,}{,}


%%% string defs
\def\cA{{\cal A}}
\def\cB{{\cal B}}
\def\cC{{\cal C}}
\def\cD{{\cal D}}
\def\cE{{\cal E}}
\def\cF{{\cal F}}
\def\cG{{\cal G}}
\def\cH{{\cal H}}
\def\cI{{\cal I}}
\def\cJ{{\cal J}}
\def\cK{{\cal K}}
\def\cL{{\cal L}}
\def\cM{{\cal M}}
\def\cN{{\cal N}}
\def\cO{{\cal O}}
\def\cP{{\cal P}}
\def\cQ{{\cal Q}}
\def\cR{{\cal R}}
\def\cS{{\cal S}}
\def\cT{{\cal T}}
\def\cU{{\cal U}}
\def\cV{{\cal V}}
\def\cW{{\cal W}}
\def\cX{{\cal X}}
\def\cY{{\cal Y}}
\def\cZ{{\cal Z}}
\def\hA{{\hat A}}
\def\hB{{\hat B}}
\def\hC{{\hat C}}
\def\hD{{\hat D}}
\def\hE{{\hat E}}
\def\hF{{\hat F}}
\def\hG{{\hat G}}
\def\hH{{\hat H}}
\def\hI{{\hat I}}
\def\hJ{{\hat J}}
\def\hK{{\hat K}}
\def\hL{{\hat L}}
\def\hP{{\hat P}}
\def\hQ{{\hat Q}}
\def\hR{{\hat R}}
\def\hS{{\hat S}}
\def\hT{{\hat T}}
\def\hX{{\hat X}}
\def\hY{{\hat Y}}
\def\hZ{{\hat Z}}
\def\eps{\epsilon}
\def\C{{\mathcal C}}
\def\F{{\mathcal F}}
\def\A{{\mathcal A}}
\def\H{{\mathcal H}}
\def\bI{\mathbb I}
\def\bO{\mathbb O}
\def\cl{\mathpzc{l}}
\def\cg{\mathpzc{g}}
\def\overlap{\between}
\def\icppl{\maltese} 
\def\invb{\textreferencemark}
\def\lndisplay{1}


\def\commentboxsize {7cm} %% llncs
\def\xnoindent{\noindent} %% llncs
\def\topshrink{0mm} %% llncs
\def\assign{\leftarrow}
\def\prelimspace{2mm}

%%% new/renew commands
% Format of comments in algorithmic package
\renewcommand{\algorithmiccomment}[1]
{ 
  \vspace {0.5mm}
  \hfill
  {\small
  \begin{tabular}{|r}
   \parbox[right]{\commentboxsize}{ \space \tt{ #1 }}\\  % {\tt /* #1 */}    \hspace{2mm}
  \end{tabular}
  }
}
% \renewcommand{\algorithmiccomment}[1]
% { 
% %  \hfill
% %  \parbox[right]{\commentboxsize} 
% {{\small \tt /* #1 */}}  % {\tt /* #1 */}    \hspace{2mm}
% }


% commands for theorems etc. 
\newtheorem{observation}{Observation}

\newcommand{\Eqr}[1]{Eq.~(\ref{#1})}
\newcommand{\diff}{\ne}
\newcommand{\OO}[1]{O\left( #1\right)}
\newcommand{\OM}[1]{\Omega\left( #1 \right)}
\newcommand{\Prob}[1]{\Pr\left\{ #1 \right\}}
\newcommand{\Set}[1]{\left\{ #1 \right\}}
\newcommand{\Seq}[1]{\left\langle #1 \right\rangle}
\newcommand{\Range}[1]{\left\{1,\ldots, #1 \right\}}
\newcommand{\ceil}[1]{\left\lceil #1 \right\rceil}
\newcommand{\floor}[1]{\left\lfloor #1 \right\rfloor}
\newcommand{\ignore}[1]{}
\newcommand{\eq}{\equiv}
\newcommand{\abs}[1]{\left| #1\right|}
\newcommand{\set}[1]{\left\{ #1\right\}}
\newcommand{\itoj}{{i \rightarrow j}}
\newcommand{\view}{\mbox{$COMM$}}
\newcommand{\pview}{\mbox{$PView$}}
\newcommand{\vx}{\mbox{${\vec x}$}}
\newcommand{\vy}{\mbox{${\vec y}$}}
\newcommand{\vv}{\mbox{${\vec v}$}}
\newcommand{\vw}{\mbox{${\vec w}$}}
\newcommand{\vb}{\mbox{${\vec b}$}}
\newcommand{\basic}{\mbox{\sc Basic}}
\newcommand{\WR}{\mbox{$\lfloor wr \rfloor$}}
\newcommand{\guarantee}{\mbox{\sc BoundedDT}}
\newcommand{\sq}{{\Delta}}
\newcommand{\Smin}{{S_{0}}}
\newcommand{\outt}{{D^{^+}}}
\newcommand{\outtp}{{\overline{D^{^+}}}}
\newcommand{\inn}{{D^{^-}}}
\newcommand{\innp}{{\overline{D^{^-}}}}
\newcommand{\indexx}{{\gamma}}
\newcommand{\D}{{D}}
\newenvironment{denselist}{
  \begin{list}{(\arabic{enumi})}{\usecounter{enumi}
      \setlength{\topsep}{0pt} \setlength{\partopsep}{0pt}
      \setlength{\itemsep}{0pt} }}{\end{list}}
\newenvironment{denseitemize}{
  \begin{list}{$\bullet$}{ \setlength{\topsep}{0pt}
      \setlength{\partopsep}{0pt} \setlength{\itemsep}{0pt}
    }}{\end{list}}
\newenvironment{subdenselist}{
  \begin{list}{(\arabic{enumi}.\arabic{enumii})}{ \usecounter{enumii}
      \setlength{\topsep}{0pt} \setlength{\partopsep}{0pt}
      \setlength{\itemsep}{0pt} }}{\end{list}}

% D O C U M E N T
\begin{document}
\title{Tree Path Labeling of Path Hypergraphs:\\
  A Generalization of
  the Consecutive Ones Property} \author{N.S. Narayanaswamy$^1$ \and
  Anju Srinivasan$^{2,3}$}
\institute{ Indian Institute of Technology
  Madras, Chennai - 600036.\\
  \email{$^1$swamy@cse.iitm.ernet.in, $^2$asz@cse.iitm.ac.in,
    $^3$anjus.math@gmail.com}}


%%%%%%%%%%%%%%%%%%%%% STACS TITLE %%%%%%%%%%%%%%%%%%%%%%%%%%%%%%%%%%%%
% \title{Tree Path Labeling of Path Hypergraphs - A Generalization of
%   Consecutive Ones Property} \titlerunning{Tree Path Labeling of Path
%   Hypergraphs}

% \author{N.S. Narayanaswamy$^1$ and Anju Srinivasan$^{2,3}$}
% \affil{Computer Science and Engineering Department,\\
%   Indian Institute of Technology Madras, Chennai - 600036, India\\
%   \texttt{$^1$swamy@cse.iitm.ernet.in, $^2$asz@cse.iitm.ac.in,
%     $^3$anjuzabil@gmail.com}}

% \authorrunning{N.S. Narayanaswamy and A.
%   Srinivasan} %optional. First: Use abbreviated first/middle
%               %names. Second (only in severe cases): Use first author
%               %plus 'et. al.'

% \Copyright[nc-nd] %choose "nd" or "nc-nd"
% {N. S. Narayanaswamy and Anju Srinivasan}

% \subjclass{XXXXXXXX TBD XXXXXXXX Dummy
%   classification}% mandatory: Please choose ACM 1998 classifications
%                  % from http://www.acm.org/about/class/ccs98-html
%                  % . E.g., cite as "F.1.1 Models of Computation".

% \keywords{XXXXXXXX TBD
%   XXXXXXXX}% mandatory: Please provide 1-5 keywords
%%%%%%%%%%%%%%%%%%%%%%%%%% STACS TITLE END %%%%%%%%%%%%%%%%%%%%%%%%%%%


\maketitle


\begin{abstract}
  In this paper, we explore a natural generalization of results on
  matrices with the Consecutive Ones Property.  We consider the
  following constraint satisfaction problem. Given (i) a set system
  $\F \subseteq$ $(2^{U} \setminus \emptyset)$ of a finite set $U$ of
  cardinality $n$, (ii) a tree $T$ of size $n$ and (iii) a bijection
  called {\em tree path labeling}, $\cl$ mapping the sets in
  $\cF$ to paths in $T$, does there exist at least one bijection
  $\phi:U \rightarrow V(T)$ such that for each $S \in \cF$, $\{\phi(x)
  \mid x \in S\} = \cl(S)$?  A tree path labeling of a set system is
  called {\em feasible} if there exists such a bijection $\phi$.  We
  present an algorithmic characterization of feasible tree path
  labeling. COP is a special instance
  of tree path labeling problem when $T$ is a path.  We also present
  an algorithm to find the tree path labeling of a given set system
  when $T$ is a {\em $k$-subdivided star}.
\end{abstract}

\section{Introduction}
\xnoindent Consecutive ones property (COP) of binary matrices is a
widely studied combinatorial problem. The problem is to rearrange rows
(columns) of a binary matrix in such a way that every column (row) has
its $1$s occur consecutively. If this is possible the matrix is said
to have the COP.  It has several practical applications in diverse
fields including scheduling \cite{hl06}, information retrieval
\cite{k77} and computational biology \cite{abh98}.  Further, it is a
tool in graph theory \cite{mcg04} for interval graph recognition,
characterization of hamiltonian graphs, and in integer linear
programming \cite{ht02,hl06}.  Recognition of COP is polynomial time
solvable by several algorithms. PQ trees \cite{bl76}, variations of PQ
trees \cite{mm09,wlh01,wlh02,mcc04}, ICPIA \cite{nsnrs09} are the
main ones.

\xnoindent
The problem of COP testing can be easily seen as a simple constraint
satisfaction problem involving a system of sets from a universe. Every
column of the binary matrix can be converted into a set of integers
which are the indices of the rows with $1$s in that column. When
observed in this context, if the matrix has the COP, a reordering of
its rows will result in sets that have only consecutive integers. In
other words, the sets after reordering are intervals. Indeed the
problem now becomes finding interval assignments to the given set
system \cite{nsnrs09} with a single permutation of the universe (set
of row indices) which permutes each set to its interval. The result in
\cite{nsnrs09} characterize interval assignments to the sets which can
be obtained from a single permutation of the rows.  They show that for
each set, the cardinality of the interval assigned to it must be same
as the cardinality of the set, and the intersection cardinality of any
two sets must be same as the interesction cardinality of the
corresponding intervals.  While this is naturally a necessary
condition, \cite{nsnrs09} shows this is indeed sufficient.  Such an
interval assignment is called an Intersection Cardinality Preserving
Interval Assignment (ICPIA).  Finally, the idea of decomposing a given
0-1 matrix into prime matrices to check for COP is adopted from
\cite{wlh02} to test if an ICPIA exists for a given set system.\\

\noindent {\bf Our Work.}  A natural generalization of the interval
assignment problem is feasible tree path labeling problem of a set
system. The problem is defined as follows - given a set system $\cF$
from a universe $U$ and a tree $T$, does there exist a bijection from
$U$ to the vertices of $T$ such that each set in the system maps to a
path in $T$.  We refer to this as the {\em tree path labeling problem}
for an input set system, target tree pair - $(\cF,T)$. As a special
case if the tree $T$ is a path, the problem becomes the interval
assignment problem.  We focus on the question of generalizing the
notion of an ICPIA \cite{nsnrs09} to characterize feasible path
assignments.  We show that for a given set system $\cF$, a tree $T$,
and an assignment of paths from $T$ to the sets, there is a feasible
bijection between $U$ and $V(T)$ if and only if all intersection
cardinalities among any three sets (not necessarily distinct) is same
as the intersection cardinality of the paths assigned to them and the
input runs a filtering algorithm (described in this paper) without
prematurely exiting.  This characterization is proved constructively
and it gives a natural data structure that stores all the relevant
feasible bijections between $U$ and $V(T)$.  Further, the filtering
algorithm is also an efficient algorithm to test if a tree path
labeling to the set system is feasible.  This
generalizes the result in \cite{nsnrs09}.\\
\xnoindent It is an interesting fact that for a matrix with the COP,
the intersection graph of the corresponding set system is an interval
graph.  A similar connection to a subclass of chordal graphs and a
superclass of interval graphs exists for the generalization of COP.
In this case, the intersection graph of the corresponding set system
must be a {\em path graph}. Chordal graphs are of great significance,
extensively studied, and have several applications.  One of the well
known and interesting properties of a chordal graphs is its connection
with intersection graphs \cite{mcg04}. For every chordal graph, there
exists a tree and a family of subtrees of this tree such that the
intersection graph of this family is isomorphic to the chordal graph
\cite{plr70,gav78,bp93}.  These trees when in a certain format, are
called clique trees \cite{apy92} of the chordal graph. This is a
compact representation of the chordal graph. There has also been work
done on the generalization of clique trees to clique hypergraphs
\cite{km02}.  If the chordal graph can be represented as the
intersection graph of paths in a tree, then the graph is called path
graph \cite{mcg04}.  Therefore, it is clear that if there is a
bijection from $U$ to $V(T)$ such that for every set, the elements in
it map to vertices of a unique path in $T$, then the intersection
graph of the set system is indeed a path graph.  However, this is only
a necessary condition and can be checked efficiently because path
graph recognition is polynomial time
solvable\cite{gav78,aas93}. Indeed, it is possible to construct a set
system and tree, such that the intersection graph is a path graph, but
there is no bijection between $U$ and $V(T)$ such that the sets map to
paths. Path graph isomorphism is known be isomorphism-complete, see
for example \cite{kklv10}. An interesting area of research would be to
see what this result tells us about the complexity of the tree path
labeling problem (not covered in this paper). In the later part of
this paper, we decompose our search for a bijection between $U$ and
$V(T)$ into subproblems.  Each subproblem is on a set subsystem in
which for each set, there is another set in the set subsystem with
which the intersection is {\em strict}, i.e., there is a non-empty
intersection, but neither is contained in the other.  This is in the
spirit of results in \cite{wlh02,nsnrs09} where to test for the COP in
a given matrix,
the COP problem is solved on an equivalent set of prime matrices.  \\

\noindent
{\bf Roadmap.} The necessary preliminaries are presented in Section
\ref{sec:prelims}. Section \ref{sec:feasible} documents the
characterization of a feasible path labeling and finally,
Section \ref{sec:ksubdivstar} describes a polynomial time
algorithm to find the tree path labeling of a given set system from a
given $k$-subdivided tree.

\section{Preliminaries} \label{sec:prelims} 
In this paper, the set $\F \subseteq (2^{U} \setminus
\emptyset)$ is a {\em set system} of a universe $U$ with $|U| = n$.
The {\em support} of a set system $\F$ denoted by $supp(\cF)$ is the
union of all the sets in $\F$, i.e., $supp(\F) = \bigcup_{S \in
  \F}S$. For the purposes of this paper, a set system is required to
``cover'' the universe, i.e. $ supp(\cF) = U$.
%\vspace{\prelimspace}

\xnoindent
The graph $T$ represents a given tree with $|V(T)| = n$. 
A {\em path system} $\cP$ is a set system of paths from
$T$. Formally, $\cP \subseteq \{P \mid P \subseteq V, \text{ } T[P]
\text{ is a path} \}$.
%\vspace{\prelimspace}

% \xnoindent
% A set system $\cF$ can be alternatively represented by a {\em
%   hypergraph} $\H_\cF$ whose vertex set is $supp(\cF)$ and hyperedges
% are the sets in $\cF$. This is a known representation for interval
% systems in literature \cite{bls99,kklv10}.  We extend this definition here to
% path systems.  
\xnoindent
A set system $\cF$ can be alternatively represented by a {\em
  hypergraph} $\cF_H$ whose vertex set is $supp(\cF)$ and hyperedges
are the sets in $\cF$. This is a known representation for interval
systems in literature \cite{bls99,kklv10}.  We extend this definition here to
path systems. Due to the equivalence of set system and hypergraph, we drop the
subscript $_H$ in the notation and refer to both the structures by $\cF$.
%\vspace{\prelimspace}


\xnoindent The {\em intersection graph} $\bI(\cF)$ of a hypergraph
$\cF$ is a graph such that its vertex set has a bijection to $\cF$ and
there exists an edge between two vertices iff their corresponding
hyperedges have a non-empty intersection \cite{mcg04}.
%\vspace{\prelimspace}


\xnoindent
Two hypergraphs $\cF'$, $\cF''$ are said to be isomorphic ({\em hypergraph
isomorphism})
to each other, denoted by $\cF' \cong \cF''$, iff there exists a
bijection $\phi: supp(\cF') \rightarrow supp(\cF'')$ such that for all
sets $A \subseteq supp(\cF')$, $A$ is a hyperedge in $\cF'$ iff $B$ is a
hyperedge in $\cF''$ where $B = \{\phi(x) \mid x \in A\}$.  
%\vspace{\prelimspace}


\xnoindent If $\cF \cong \cP$ where $\cP$ is a path system, then
$\cF$ is called a {\em path hypergraph} and $\cP$ is called {\em
  path representation} of $\cF$. If isomorphism is $\phi:
supp(\cF) \rightarrow V(T)$, then it is clear that there is
an induced path labeling $l_\phi: \cF \rightarrow \cP$ to the set
system. Note that $supp(\cP) = V(T)$. % In the rest of the document, we may use $\cF$ and/or $\cH_\cF$
% interchangeably to refer the set system and/or its hypergraph.
%\vspace{\prelimspace}


\noindent 
\annote[a]{Let the intersection graphs of two hypergraphs be
  isomorphic, $\bI(\cF) \cong \bI(\cP)$ and $\cP$ be a path
  system. }{REVIEW THIS WHOLE PARA} Then the bijection $\cl: \cF
\rightarrow \cP$ induced by this isomorphism is called a {\em path
  labeling} of the hypergraph $\cF$. To elaborate, let $\cg: V(\cF)
\rightarrow V(\cP)$ be the above mentioned isomorphism where $V(\cF)$
and $V(\cP)$ are the vertex sets that represent the hyperedges for
each hypergraph respectively, $V(\cF) = \{ v_S \mid S \in \cF\}$ and
$V(\cP) = \{ v_P \mid P \in \cP\}$. Then the path labeling $\cl$ is
defined as follows: $\cl(S_1) = P_1$ iff $\cg (v_{S_1}) =
v_{P_1}$. Also, the path system $\cP$ may be alternatively denoted in
terms of $\cF$ and $\cl$ as $\cF^\cl$. In most of the scenarios in
this paper, what is given is the pair $(\cF, \cl)$.

% A graph $G$ is a {\em path graph} if it is isomorphic to the intersection graph
% $\bI(\cP)$ of a path system $\cP$. 
% This
% isomorphism gives a bijection $\cl': V(G) \rightarrow \cP$. Moreover,
% for the purposes of this paper, we require that in a path labeling,
% $supp(\cP) = V(T)$.  If graph $G$ is also isomorphic to $\bI(\cF)$ for some hypergraph
% $\cF$, then clearly there is a bijection $\cl: \cF \rightarrow \cP$
% such that $\cl(S) = \cl'(v_S)$ where $v_S$ is the vertex corresponding
% to set $S$ in $\bI(\cF)$ for any $S \in \cF$. This bijection $\cl$ is
% called the {\em path labeling} of the hypergraph $\cF$ and the path
% system $\cP$ may be alternatively denoted as $\cF^\cl$.
%\vspace{\prelimspace}


\xnoindent
An {\em overlap graph} $\bO(\cF)$ of a hypergraph $\cF$ is a graph
such that its vertex set has a bijection to $\cF$ and there exists an
edge between two of its vertices iff their corresponding hyperedges
overlap. Two hyperedges $S$ and $S'$ are said to {\em overlap},
denoted by $S \overlap S'$, if they have a non-empty intersection and
neither is contained in the other i.e. $S \overlap S' \text{ iff } S
\cap S' \ne \emptyset, S \nsubseteq S', S' \nsubseteq S$. Thus $\bO(\cF)$
is a subgraph of $\bI(\cF)$ and not necessarily connected. Each
connected component of $\bO(\cF)$ is called an {\em overlap
  component}.
% If there are $d$ overlap components
% in $\bO(\cF)$, the set subsystems are denoted by $\cO_1, \cO_2, \ldots
% \cO_d$. Clearly $\cO_i \subseteq \F, i \in [d]$. For any $i, j \in [d]$,
% it can be verified that one of the following is true.
% \begin{enumerate}[a) ]
% \item $supp(\cO_i)$ and $supp(\cO_j)$ are disjoint
% \item $supp(\cO_i)$ is a subset of a set in $\cO_j$
% \item $supp(\cO_j)$ is a subset of a set in $\cO_i$
% \end{enumerate}
%\vspace{\prelimspace}

 
\xnoindent
A path labeling $\cl: \cF \rightarrow \cP$ is defined to be {\em
  feasible} if
$\cF \cong \cP$ and this hypergraph isomorphism $\phi: supp(\cF)
\rightarrow supp(\cP)$ induces a path labeling $\cl_\phi: \cF
\rightarrow \cP$ such that $\cl_\phi = \cl$. 
%\vspace{\prelimspace}


\xnoindent A {\em star} graph is a complete bipartite graph $K_{1,l}$
which is clearly a tree and $p$ is the number of leaves. The vertex
with maximum degree is called the {\em center} of the star and the
edges are called {\em rays} of the star.
%\vspace{\prelimspace}


\xnoindent A {\em $k$-subdivided star} is a star with all its rays
subdivided exactly $k$ times. The path from the center to a leaf is
called a ray of a $k$-subdivided star and they are all of length
$k+2$.


\section{Characterization of Feasible Tree Path  Labeling} 
\label{sec:feasible} 

% Consider a path labeling $\cl: \cF \rightarrow \cP$ for set system $\cF$
% and path system $\cP$ on the given tree $T$. We
% call $\cl$ an {\em Intersection Cardinality Preserving Path Labeling
%   (ICPPL)} if it has the following properties.
Consider a path labeling $(\cF, \cl)$ for hypergraph $\cF$ on the
given tree $T$. We call $\cl$ an {\em Intersection Cardinality
  Preserving Path Labeling (ICPPL)} if it has the following
properties.

\vspace{\topshrink}
{\em
  \begin{enumerate}[i.]
  \item $|S| = |\cl(S)|$\\%\hfill\icppl\\
    for all $S \in \cF$%\hfill\icppl 
\vspace{\topshrink}
  \item $|S_1 \cap S_2| = |\cl(S_1) \cap
    \cl(S_2)|$\\%\hfill\icppl\icppl \\
    for all distinct $S_1, S_2 \in
    \cF$%\hfill\icppl\icppl 
\vspace{\topshrink}
  \item $|S_1 \cap S_2 \cap S_3| = |\cl(S_1)
    \cap \cl(S_2) \cap
    \cl(S_3)|$\\%\hfill\icppl\icppl\icppl\\
    for all distinct $S_1, S_2, S_3 \in
    \cF$%\hfill\icppl\icppl\icppl 
  \end{enumerate}}

\xnoindent The following lemma is useful in characterizing feasible
tree path labelings.  The proof is in the appendix.
\begin{lemma}
  \label{lem:setminuscard}
  If $\cl$ is an ICPPL, and $S_1, S_2, S_3 \in \cF$, then $|S_1 \cap
  (S_2 \setminus S_3)| = |\cl(S_1) \cap (\cl(S_2) \setminus
  \cl(S_3))|$.
\end{lemma}

\xnoindent In the remaining part of this section we describe an
algorithmic characterization for a feasible tree path labeling. We
show that a path labeling is feasible if and only if it is an ICPPL
and it successfully passes the filtering algorithms \ref{perms} and
\ref{leafasgn}. One direction of this claim is clear: that if a path
labeling is feasible, then all intersection cardinalities are
preserved, i.e. the path labeling is an ICPPL. Algorithm \ref{perms}
\annote[a]{has no premature exit condition hence any input will go through
it}{Prove that the filtered sets has ICPPL iff input PL has ICPPL?}. Algorithm \ref{leafasgn} has an exit condition at line
\ref{xempty}. It can be easily verified that $X$ cannot be empty if
$\cl$ is a feasible path labeling. The reason is that a feasible path
labeling has an associated bijection between $supp(\cF)$ and $V(T)$
\remove[a]{i.e. $supp(\cF^{\cl})$} such that the sets map to paths, ``preserving''
the path labeling. The rest of the section is devoted to
constructively proving that it is sufficient for a path labeling to be
an ICPPL and pass the two filtering algorithms.  To describe in brief,
the constructive approaches refine an ICPPL iteratively, such that at
the end of each iteration we have a ``filtered'' path labeling, and
finally we have a path labeling that defines a family of bijections
from $supp(\cF)$ to $V(T)$\remove[a]{ i.e. $supp(\cF^{\cl})$}.

\xnoindent First, we present Algorithm \ref{perms} or Filter 1, and
prove its correctness.  This algorithm refines the path labeling by
considering pairs of paths that share a leaf until no two paths in the
new path labeling share any leaf.


\begin{algorithm}[h]
  \caption{Refine ICPPL {\tt filter\_1($\cF, \cl, T$)}}
  \label{perms}
  \begin{algorithmic}[\lndisplay]
    \STATE $\cF_0 \assign \cF$, $\cl_0(S) \assign \cl(S)$ for all $S \in \cF_0$\\
    \STATE $j = 1$\\
    \WHILE {there is $S_1, S_2 \in \cF_{j-1}$ such that
      $\cl_{j-1}(S_1)$ and $\cl_{j-1}(S_2)$ have a common leaf in
      $T$}\label{shareleaf} \STATE $\cF_j \assign (\cF_{j-1} \setminus
    \{S_1, S_2\})
    \cup \{S_1 \cap S_2, S_1 \setminus S_2, S_2 \setminus S_1 \}$ \label{setbreak} \\
    \COMMENT {Remove $S_1$, $S_2$ and add the ``filtered'' sets}
    \STATE for all $S \in \cF_{j-1}$ such that $S \ne S_1$ and $S \ne
    S_2$, set $\cl_j(S) \assign \cl_{j-1}(S)$\\
    \COMMENT {Do not change path labeling for any set other than
      $S_1$, $S_2$}
    \STATE $\cl_j(S_1 \cap S_2) \assign \cl_{j-1}(S_1) \cap \cl_{j-1}(S_2)$\\
    \STATE $\cl_j(S_1 \setminus S_2) \assign \cl_{j-1}(S_1) \setminus \cl_{j-1}(S_2)$\\
    \STATE $\cl_j(S_2 \setminus S_1) \assign \cl_{j-1}(S_2) \setminus
    \cl_{j-1}(S_1)$

    \IF{$(\cF_j, \cl_j)$ does not satisfy condition (iii) of ICPPL}
    \label{ln:3waycheck}
    \STATE {\bf exit} \label{ln:exit1} \\
    \ENDIF

    \STATE $j \assign j+1$\\
    \ENDWHILE
    \STATE $\cF' \assign \cF_j$, $\cl' \assign \cl_j$\\
    \RETURN $(\cF', \cl')$
  \end{algorithmic}
\end{algorithm}

\begin{lemma} 
 \label{lem:feasible} 
 In Algorithm \ref{perms}, if input $(\cF, \cl)$ is a feasible path
 assignment then at the end of $j$th iteration of the {\bf while}
 loop, $j \ge 0$, $(\cF_j, \cl_j)$ is a feasible path assignment.
\end{lemma}

\begin{lemma}
  \label{lem:invar1} In Algorithm \ref{perms}, at the end of $j$th
  iteration, $j \ge 0$, of the {\bf while} loop of Algorithm
  \ref{perms}, the following invariants are maintained.
 \begin{enumerate}[I {\textreferencemark}] \vspace{\topshrink}
  \item $\cl_j(R)$ is a path in $T$,\hfill 
    for all $R \in \cF_j$\vspace{\topshrink}
  \item $|R| = |\cl_j(R)|$,\hfill 
    for all $R \in \cF_j$\vspace{\topshrink}
  \item $|R \cap R'| = |\cl_j(R) \cap \cl_j(R')|$,\hfill 
    for all $R, R' \in \cF_j$\vspace{\topshrink}
  \item $|R \cap R' \cap R''|=|\cl_j(R) \cap \cl_j(R') \cap
    \cl_j(R'')|$,\hfill for all $R, R', R'' \in \cF_j$
  \end{enumerate}
\end{lemma}

\begin{proof}
%   The detailed proofs of the some of the cases below are in the
%   appendix.  
  Proof is by induction on the number of iterations, $j$. In this
  proof, the term ``new sets'' will refer to the sets added to $\cF_j$
  in $j$th iteration in line \ref{setbreak} of Algorithm \ref{perms},
  $S_1 \cap S_2, S_1 \setminus S_2, S_2 \setminus S_1$ and its
  images in $\cl_j$ where $\cl_{j-1}(S_1)$
  and $\cl_{j-1}(S_2)$ intersect and share a leaf.\\
  \xnoindent In the base case $(\cF_0, \cl_0)$ is an ICPPL, since it
  is the input.  Assume the lemma is true till the $j-1$th
  iteration. Let us consider the possible cases for each of the above invariants for
  the $j$th iteration.

  \xnoindent
 \begin{enumerate}[\textreferencemark]
  \item {\em Invariant} I/II
    \begin{enumerate}[{I/II}a $|$] % \textbullet 
    \item {\em $R$ is not a new set.} It is in $\cF_{j-1}$. Thus
      trivially true by induction hypothesis.
    \item {\em $R$ is a new set.} If $R$ is in $\cF_{j}$ and not in
      $\cF_{j-1}$, then it must be one of the new sets added in
      $\cF_j$. In this case, it is clear that for each new set, the
      image under $\cl_j$ is a path since by definition the chosen
      sets $S_1$, $S_2$ are from $\cF_{j-1}$ and due to the while loop
      condition, $\cl_{j-1}(S_1)$, $\cl_{j-1}(S_2)$ have a
      common leaf. Thus invariant I is proven.\\
      Moreover, due to induction hypothesis of invariant III and the
      definition of $l_j$ in terms of $l_{j-1}$, invariant II is
      indeed true in the $j$th iteration for any of the new sets.  If
      $R = S_1 \cap S_2$, $|R| = |S_1 \cap S_2| = |\cl_{j-1}(S_1) \cap
      \cl_{j-1}(S_2)| = |\cl_j(S_1 \cap S_2)| = |\cl_j(R)|$.
      If $R = S_1 \setminus S_2$, $|R| = |S_1 \setminus S_2| = |S_1| -
      |S_1 \cap S_2| = |\cl_{j-1}(S_1)| - |\cl_{j-1}(S_1) \cap
      \cl_{j-1}(S_2)| = |\cl_{j-1}(S_1) \setminus \cl_{j-1}(S_2)| =
      |\cl_j(S_1 \setminus S_2)|
      = |\cl_j(R)|$. Similarly if $R = S_2 \setminus S_1$.\\
    \end{enumerate}
  \item {\em Invariant} III
    \begin{enumerate}[{III}a $|$]
    \item {\em $R$ and $R'$ are not new sets.} It is in
      $\cF_{j-1}$. Thus trivially true by induction hypothesis.
    \item {\em Only one, say $R$, is a new set.} Due to invariant IV
      induction hypothesis, Lemma \ref{lem:setminuscard} and
      definition of $\cl_j$, it follows that invariant III is true no
      matter which of the new sets $R$ is equal to. If $R = S_1 \cap
      S_2$, $|R \cap R'| = |S_1 \cap S_2 \cap R'| = |\cl_{j-1}(S_1)
      \cap \cl_{j-1}(S_2) \cap \cl_{j-1}(R')| = |\cl_j(S_1 \cap S_2)
      \cap \cl_j(R')| = |\cl_j(R) \cap \cl_j(R')|$.  If $R = S_1
      \setminus S_2$, $|R \cap R'| = |(S_1 \setminus S_2) \cap R'| =
      |(\cl_{j-1}(S_1) \setminus \cl_{j-1}(S_2)) \cap \cl_{j-1}(R')| =
      |\cl_{j}(S_1 \cap S_2) \cap \cl_{j}(R')| = |\cl_{j}(R) \cap
      \cl_{j}(R')|$. Similarly, if $R = S_2 \setminus
      S_1$. Note $R'$ is not a new set.\\

    \item {\em $R$ and $R'$ are new sets.} By definition, the new
      sets and their path images in path label $\cl_j$ are disjoint so
      $|R \cap R'| = |\cl_j(R) \cap \cl_j(R)| = 0$. Thus case proven.
    \end{enumerate}
  \item {\em Invariant} IV
    
    Due to the condition in line \ref{ln:3waycheck}, this invariant is
    ensured at the end of every iteration.
%     \begin{enumerate} [{Case 3.}1:]
%     \item {\em $R$, $R'$ and $R''$ are not new sets.} Trivially
%       true by induction hypothesis.
%     \item {\em Only one, say $R$, is a new set.}
%       If $R = S_1 \cap S_2$,  from lemma \ref{lem:fourpaths} and
%       invariant III hypothesis,  this case is proven. Similarly if $R$
%       is any of the other new  sets, the case is proven by also using
%       lemma  \ref{lem:setminuscard}.
%     \item {\em At least two of $R, R', R''$ are new sets.}
%       The new sets are disjoint hence this case is vacuously true.
%     \end{enumerate}
  \end{enumerate} 
\vspace{-6mm} 
\qed
\end{proof}

\begin{lemma}
  \label{lem:noexit1}
  If the input ICPPL $(\cF, \cl)$ to Algorithm \ref{perms} is
  feasible, then the set of hypergraph isomorphism functions that
  defines $(\cF, \cl)$'s feasibility is the same as the set that
  defines $(\cF_j, \cl_j)$'s feasibility, if any.  Secondly, for any
  iteration $j > 0$ of the {\em \bf while} loop, the {\em \bf exit}
  statement in line \ref{ln:exit1} will not execute.
\end{lemma}
\begin{proof}
  Since $(\cF,\cl)$ is feasible, by Lemma \ref{lem:feasible}
  $(\cF_j,\cl_j)$ for every iteration $j > 0$ is feasible.  % Therefore,
%   every hypergraph isomorphism $\phi: supp(\cF) \rightarrow V(T)$ that
%   induces $\cl$ on $\cF$ also induces $\cl_{j-1}$ and $\cl_{j}$ on
%   $\cF_{j-1}$ and $\cF_{j}$ respectively, i.e., $\cl_{\phi[\cF_{j-1}]}
%   = \cl_{j-1}$ and $\cl_{\phi[\cF_j]} = \cl_j$. Thus it can be seen
%   that for all $x \in supp(\cF)$, for all $v \in V(T)$ the following
%   hold true.
  Also, every hypergraph isomorphism $\phi: supp(\cF) \rightarrow
  V(T)$ that induces $\cl$ on $\cF$ also induces $\cl_{j}$ on
  $\cF_{j}$, i.e., $\cl_{\phi[\cF_j]} = \cl_j$. Thus it can be seen
  that for all $x \in supp(\cF)$, for all $v \in V(T)$, if $(x,v) \in
  \phi$ then $v \in \cl_{j}(S)$ for all $S \in \cF_{j}$ such that $x
  \in S$.
% the following
%   hold true.
%   \begin{enumerate}[i. ]
%   \item If $(x,v) \in \phi$ then $v \in \cl_{j-1}(S)$ for all $S \in
%     \cF_{j-1}$ such that $x \in S$.
%   \item If $(x,v) \in \phi$ then $v \in \cl_{j}(S)$ for all $S \in
%     \cF_{j}$ such that $x \in S$
%   \end{enumerate}
  In other words, filter 1 outputs a filtered path labeling that ``preserves''
  hypergraph isomorphisms of the original path labeling.\\
  Secondly, line \ref{ln:exit1} will execute iff the exit condition in line
  \ref{ln:3waycheck}, i.e. failure of three way intersection
  preservation, becomes true in any iteration of the {\em \bf while}
  loop.  Due to Lemma \ref{lem:invar1} Invariant IV, the exit
  condition does not occur if the input is a feasible ICPPL.\qed

%   such that $\phi(x) = v$ where $v$ is the leaf considered in the
%   first iterations of while. Clearly, $\phi$ is a renaming of
%   vertices in hypergraph $\cF$ to those in hypergraph $\cF^\cl$. Thus
%   the following facts can be observed in every iteration of the loop.

%   \begin{enumerate}[\hspace{2mm}i. ] \vspace{\topshrink}
%   \item all intersection cardinalities are preserved in this path
%     labeling \vspace{\topshrink}
%   \item element $x$ is exclusive in a hyperedge in $\cF$ since $v$ is
%     exclusive in a hyperedge in $\cF^\cl$.
%   \end{enumerate}

%   Thus the exit condition is never rendered true after $x$ and $v$ are
%   removed from their respective hyperedges. \qed

% \noindent
% This proof uses mathematical induction on the number
%   of iterations $j$, $j \ge 0$, of the loop that executed
%   without exiting. The base case, $j = 0$ is obviously true since the
%   input is an ICPPL and the exit condition cannot hold true due to
%   ICPPL condition (iii).  Assume the algorithm executes till the end
%   of $j-1$th iteration without exiting at line
%   \ref{ln:3waycheck}. Consider the $j$th iteration. From lemma
%   \ref{lem:feasible} we know that $(\cF_j, \cl_j)$ and $(\cF_{j-1},
%   \cl_{j-1})$ are feasible\remove[AS]{and from the proof in lemma
%     lem:invar1 we know that $(\cF_{j-1}, \cl_{j-1})$ satisfies all the
%     invariants defined in the lemma}.  Thus there exists a bijection
%   $\phi: supp(\cF) \rightarrow V(T)$ such that the induced path
%   % labeling on $\cF_{j-1}$ $\cl_{\phi[\cF_{j-1}]} = \cl_{j-1}$.
%   labeling on $\cF_{j}$, $\cl_{\phi[\cF_{j}]}$ and on $\cF_{j-1}$,
%   $\cl_{\phi[\cF_{j-1}]}$ are equal to $\cl_{j}$ and $\cl_{j-1}$
%   respectively.  We need to prove that for any $R, R', R'' \in
%   \cF_{j}$, $|R \cap R' \cap R''| = |\cl_j(R) \cap \cl_j(R') \cap
%   \cl_j(R'')|$.
%   The following are the possible cases that could arise. From argument
%   above, $|\cl_j(R) \cap \cl_j(R') \cap \cl_j(R'')| =
%   |\cl_{\phi[\cF_{j}]}(R) \cap \cl_{\phi[\cF_{j}]} (R') \cap
%   \cl_{\phi[\cF_{j}]} (R'')|$

%   \begin{enumerate}[a $|$]
%   \item {\em None of the sets are new. $R, R', R'' \in \cF_{j-1}$.}
%     We know $(\cF_{j-1}, \cl_{j-1})$ is feasible. Thus $|R \cap R'
%     \cap R''| = |\cl_{j-1}(R) \cap \cl_{j-1}(R') \cap \cl_{j-1}(R'')|
%     = |\cl_{j}(R) \cap \cl_{j}(R') \cap \cl_{j}(R'')|$.
%   \item {\em Only one, say $R$, is a new set.}  Let $R = S_1 \cap S_2$
%     ($S_1, S_2$ are defined in the proof of lemma
%     \ref{lem:invar1}). Now we have $|R \cap R' \cap R''| = |S_1 \cap
%     S_2 \cap R' \cap R''| = |\cl_{j-1}(S_1) \cap \cl_{j-1}(S_2) \cap
%     \cl_{j-1}(R') \cap \cl_{j-1}(R'')| = |\cl_{j}(R) \cap \cl_{j}(R')
%     \cap \cl_{j}(R'')|$. Thus proven. If $R$ is any of the other new
%     sets, the same claim can be verified using lemma
%     \ref{lem:setminuscard}.
%     % \item []{\bf Case 3:}
%   \item {\em At least two of $R, R', R''$ are new sets.}  The new sets
%     are disjoint hence this case is vacuously true.
%   \end{enumerate}
%   \qed
%   \tnote[E2]{remove the induction proof. just text saying x and v are
%     exclusive in these sets therefore the intersection cardinalities
%     don't change thus all invariants are still true} 
\end{proof}

\xnoindent As a result of Algorithm \ref{perms} each leaf $v$ in $T$
is such that there is exactly one set in $\cF$ with $v$ as a vertex in
the path assigned to it.  In Algorithm \ref{leafasgn} we identify
elements in $supp(\cF)$ whose images are leaves in a hypergraph
isomorphism if one exists.  Let $S \in \cF$ be such that $\cl(S)$ is a
path with leaf and $v \in V(T)$ is the unique leaf incident on it.  We
define a new path labeling $\cl_{new}$ such that $\cl_{new}(\set{x}) =
\set{v}$ where $x$ an arbitrary element from $S \setminus \bigcup_{\hS
  \ne S} \hS$. In other words, $x$ is an element present in no other
set in $\cF$ except $S$. This is intuitive since $v$ is present in no
other path image under $\cl$ other than $\cl(S)$.  The element $x$ and
leaf $v$ are then removed from the set $S$ and path $\cl(S)$
respectively. After doing this for all leaves in $T$, all path images
in the new path labeling $\cl_{new}$ except leaf labels (a path that
has only a leaf is called the {\em leaf label} for the corresponding
single element hyperedge or set) are paths from a new pruned tree $T_0
= T \setminus \{v \mid v \text{ is a leaf in } T\}$. Algorithm
\ref{leafasgn} is now presented with details.


\begin{algorithm}[h]
  \caption{Leaf labeling from an ICPPL {\tt filter\_2($\cF, \cl, T$)}}
  \label{leafasgn}
  \begin{algorithmic}[\lndisplay]
    \STATE $\cF_0 \assign \cF$, $\cl_0(S) \assign \cl(S)$ for all $S \in \cF_0$
    \COMMENT {Path images are such that no two path images share a
      leaf.}
    \STATE $j \assign 1$\\
    \WHILE {there is a leaf $v$ in $T$ and a unique $S_1 \in
      \cF_{j-1}$ such that $v \in \cl_{j-1}(S_1)$ }\label{uniqueleaf}
    \STATE $\cF_j \assign \cF_{j-1} \setminus \{S_1\}$\\
    \STATE for all $S \in \cF_{j-1}$ such that $S \ne S_1$ set
    $\cl_j(S) \assign
    \cl_{j-1}(S)$\\
    \STATE $X \assign S_1 \setminus \bigcup_{S \in \cF_{j-1}, S \ne S_1}S$\\
    \IF{$X$ is empty} \label{xempty} \STATE {\bf exit} \label{ln:exit2} \ENDIF
    \STATE $x \assign $ arbitrary element from $X$\\
    \STATE $\cF_j \assign \cF_j \cup \{\{x\}, S_1 \setminus \{x\}\} $\\
    \STATE $\cl_j(\{x\}) \assign \{v\}$\\
    \STATE $\cl_j(S_1 \setminus \{x\}) \assign \cl_{j-1}(S_1) \setminus \{v\}$\\
    \STATE $j \assign j+1$\\
    \ENDWHILE
    \STATE $\cF' \assign \cF_j$, $\cl' \assign \cl_j$\\
    \RETURN $(\cF', \cl')$
  \end{algorithmic}
\end{algorithm}

\noindent
Suppose the input ICPPL $(\cF, \cl)$ is feasible, yet set $X$ in
Algorithm \ref{leafasgn} is empty in some iteration of the {\bf while}
loop. This will abort our procedure of finding the hypergraph
isomorphism. The following lemma shows that this will not happen.

\begin{lemma}
  \label{lem:xnotempty}
  If the input ICPPL $(\cF, \cl)$ to Algorithm \ref{leafasgn} is
  feasible, then for all iterations $j > 0$ of the {\em \bf while}
  loop, the {\em \bf exit} statement in line \ref{ln:exit2} does not
  execute.
\end{lemma}
\begin{proof}
  Assume $X$ is empty for some iteration $j > 0$. We know that $v$ is
  an element of $\cl_{j-1}(S_1)$. Since it is uniquely present in
  $\cl_{j-1}(S_1)$, it is clear that $v \in \cl_{j-1}(S_1) \setminus
  \bigcup_{(S \in \cF_{j-1}) \wedge (S \ne S_1)}\cl_{j-1}(S)$.  Note
  that for any $x \in S_1$ it is contained in at least two sets due to
  our assumption about cardinality of $X$. Let $S_2 \in \cF_{j-1}$ be
  another set that contains $x$. From the above argument, we know $v
  \notin \cl_{j-1}(S_2)$. Therefore there cannot exist a hypergraph
  isomorphism bijection that maps elements in $S_2$ to those in
  $\cl_{j-1}(S_2)$. This contradicts our assumption that the input is
  feasible. Thus $X$ cannot be empty if input is ICPPL and feasible.
  \qed
\end{proof}

\begin{lemma}
  \label{lem:invar3}
  In Algorithm \ref{leafasgn}, for all $j > 0$, at the end of the
  $j$th iteration the four invariants given in lemma \ref{lem:invar1}
  are valid.
\end{lemma}
\begin{proof}
  By Lemma \ref{lem:xnotempty} we know that set $X$ will not be empty
  in any iteration of the {\em \bf while} loop if input ICPPL $(\cF,
  \cl)$ is feasible and $\cl_j$ is always computed for all $j >
  0$. Also note that removing a leaf from any path keeps the new path
  connected. Thus invariant I is obviously true. In every iteration $j
  > 0$, we remove exactly one element $x$ from one set $S$ in $\cF$
  and exactly one vertex $v$ which is a leaf from one path
  $\cl_{j-1}(S)$ in $T$. This is because as seen in Lemma
  \ref{lem:xnotempty}, $x$ is exclusive to $S$ and $v$ is exclusive to
  $\cl_{j-1}(S)$. Due to this fact, it is clear that the intersection
  cardinality equations do not change, i.e., invariants II, III, IV
  remain true. On the other hand, if the input ICPPL is not feasible
  the invariants are vacuously true. \qed
\end{proof}

% \textcolor{cyan}{
% \begin{lemma}
%   \label{lem:notfeasibleexit}
%   \tnote[a]{IS THIS CORRECT?}  If input ICPPL $(\cF, \cl)$ is not
%   feasible, then in one of the recursive calls to Algorithm --3--, the
%   {\em \bf exit} statement in line x in Algorithm --1 or line y in
%   Algorithm --2 will get executed.
% \end{lemma}
% }


\xnoindent We have seen two filtering algorithms above, namely,
Algorithm \ref{perms} {\tt filter\_1} and Algorithm \ref{leafasgn}
{\tt filter\_2} which when executed serially repectively result in a
new ICPPL on the same universe $U$ and tree $T$. We also proved that
if the input is indeed feasible, these algorithms do indeed output the
filtered ICPPL. Now we present the algorithmic characterization of a
feasible tree path labeling by way of Algorithm
\ref{al:icppl-find-isomorph}.

\noindent
Algorithm \ref{al:icppl-find-isomorph} computes a hypergraph
isomorphism $\phi$ recursively using Algorithm \ref{perms} and
Algorithm \ref{leafasgn} and pruning the leaves of the given tree. In
brief, it is done as follows. Algorithm \ref{leafasgn} gives us the
leaf labels in $\cF_2$, i.e., the elements in $supp(\cF)$ that map to
leaves in $T$, where $(\cF_2, \cl_2)$ is the output of Algorithm
\ref{leafasgn}. All leaves in $T$ are then pruned away. The leaf
labels are removed from the path labeling $\cl_2$ and the
corresponding elements are removed from the corresponding sets in
$\cF_2$. This tree pruning algorithm is recursively called on the
altered hypergraph $\cF'$, path label $\cl'$ and tree $T'$. The
recursive call returns the bijection $\phi''$ for the rest of the
elements in $supp(\cF)$ which along with the leaf labels $\phi'$ gives
us the hypergraph isomorphism $\phi$.  The following lemma formalizes
the characeterization of feasible path labeling.

\begin{algorithm}[h]
  \caption{{\tt get-hypergraph-isomorphism($\cF, \cl, T$)}}
  \label{al:icppl-find-isomorph}
  \begin{algorithmic}[\lndisplay]

    \IF{$T$ is empty}
    \RETURN $\emptyset$\\
    \ENDIF
    \STATE $L \assign \{v \mid v \text{ is a leaf in }      T\}$\\
    \STATE $(\cF_1, \cl_1) \assign$ {\tt filter\_1($\cF, \cl,
      T$)}\\
    \STATE $(\cF_2, \cl_2) \assign$ {\tt filter\_2($\cF_1,
      \cl_1, T$)}\\

    \STATE $(\cF', \cl') \assign (\cF_2, \cl_2)$\\
    \STATE $\phi' \leftarrow \emptyset$

    \FOR {every $v \in L$}
    \STATE $\phi'(x) \assign v$ where $x \in \cl_2^{-1}(\{v\})$
    \COMMENT {Copy the leaf labels to a one to one function $\phi':
      supp(\cF) \rightarrow L$
      }\\
    \STATE Remove $\{x\}$ and $\{v\}$ from $\cF'$, $\cl'$  appropriately\\
    \ENDFOR

    \STATE $T' \assign T \setminus L$

    \STATE $\phi'' \assign$ {\tt
      get-hypergraph-isomorphism($\cF', \cl', T'$)}
    \STATE $\phi \assign \phi'' \cup \phi'$ \\
    \RETURN $\phi$
  \end{algorithmic}
\end{algorithm}

\begin{lemma}
  \label{lem:hyperiso}  %{lem:perm}
  If $(\cF, \cl)$ is an ICPPL from a tree $T$ and Algorithm
  \ref{al:icppl-find-isomorph}, {\tt get-hypergraph-isomorphism ($\cF, \cl, T$)}
  returns a non-empty function, then there exists a hypergraph
  isomorphism $\phi : supp(\cF) \rightarrow V(T)$ such that the
  $\phi$-induced tree path labeling is equal to $\cl$ or $\cl_\phi =
  \cl$.
\end{lemma}
\begin{proof}
  It is clear that in the end of every recursive call to Algorithm
  \ref{al:icppl-find-isomorph}, the function $\phi'$ is one to one
  involving all the leaves in the tree passed as input to that
  recursive call. Moreover, by Lemma \ref{lem:noexit1} and Lemma
  \ref{lem:xnotempty} it is consistent with the tree path labeling
  $\cl$ passed. The tree pruning is done by only removing leaves in
  each call to the function and is done till the tree becomes
  empty. Thus the returned function $\phi: supp(\cF) \rightarrow V(T)$
  is a union of mutually exclusive one to one functions exhausting all
  vertices of the tree. In other words, it is a bijection from
  $supp(\cF)$ to $V(T)$ inducing the given path labeling $\cl$ and thus
  a hypergraph isomorphism. \qed
\end{proof}

\begin{theorem}
  \label{th:charac}
  A path labeling $(\cF, \cl)$ on tree $T$ is feasible iff it is an
  ICPPL and Algorithm \ref{al:icppl-find-isomorph} with $(\cF, \cl,
  T)$ as input returns a non-empty function.
\end{theorem}
\begin{proof}
  From Lemma \ref{lem:hyperiso}, we know that if $(\cF, \cl)$ is an
  ICPPL and Algorithm \ref{al:icppl-find-isomorph} with $(\cF, \cl,
  T)$ as input returns a non-empty function, $(\cF, \cl)$ is feasible.
  Now consider the case where $(\cF, \cl)$ is feasible. i.e. there
  exists a hypergraph isomorphism $\phi$ such that $\cl_\phi =
  \cl$. Lemma \ref{lem:noexit1} and Lemma \ref{lem:xnotempty} show us
  that filter 1 and filter 2 do not exit if input is feasible. Thus
  Algorithm \ref{al:icppl-find-isomorph} returns a non-empty function.\qed
\end{proof}

\subsection{ICPPL when given tree is a path}
\label{subsec:icpplicpia}
\xnoindent
Consider a special case of ICPPL with the following properties.
\begin{enumerate}
\item Given tree $T$ is a path. Hence, all path labels are interval labels.
\item Only pairwise intersection cardinality
  preservation is sufficient. i.e. condition (iii) in ICPPL is not enforced.
\item The filter algorithms do not have {\em \bf exit} statements.
\end{enumerate}
This is called an Intersection Cardinality Preservation Interval
Assignment (ICPIA) \cite{nsnrs09}. This structure and its algorithm is
used in the next section for finding tree path labeling from a
$k$-subdivided star due to this graph's close relationship with
intervals. 


\section{Testing for feasible path assignments to $k$-subdivided star}
\label{sec:ksubdivstar}
In this section we consider the problem of assigning paths from a $k$-subdivided 
star $T$ to a given set system $\cF$.  We consider $\cF$ for which an associated graph
called the overlap graph, denoted by $\cO(\cF)$ is connected.  The overlap graph is well-know from the work of \cite{kklv10,nsnrs09,wlh02}.   We use the notation in \cite{kklv10}.  Hyperedges A and B overlap, written $A \overlap B$  if $A$ and
$B$ have a nonempty intersection but neither of them is contained in  the other. The
overlap graph $\cO(\cF)$ is a graph in which the vertices correspond to the sets in $\cF$,
and the
vertices corresponding to the hyperedges $A$ and $B$ are adjacent if and only if
they overlap.  Note that the intersection graph of $\cF$ is different from the $\cO(\cF)$.
 A connected component of $\cO(\cF)$ is called an overlap component of $\cF$.  
An interesting property of the overlap components is that any two distinct overlap 
components, say $\cO_1$ and $\cO_2$, are such that any two sets $A_1 \in \cO_1$ and $A_2 \in \cO_2$ are disjoint, or, w.l.o.g, all the sets in $\cO_1$ are contained in one set in $\cO_2$.  
 This containment relation naturally determines a decomposition of the overlap 
components into rooted containment trees.   We consider the case when there is only one rooted containment tree, and we first present our algorithm when $\cO(\cF)$ is
connected.  It is easy to see that once the path assignment for the overlap component in the root is achieved, the path assignment for the other overlap components in the rooted containment tree is
essentially finding a path assignment when the target tree is a path: each target path is a path that is allocated to sets in the root overlap compnent.  Therefore, for the
rest of this section, $\cO(\cF)$ is a connected graph.  

We start by recalling a $k$-subdivided star: Each edge of a star is subdivided $k$ times.
Therefore, a $k$-subdivided star has a central vertex which we call the {\em root}, and each root to leaf path is refered to as a {\em ray}.  First, we observe that by removing the
root from $r$, we get a collection of vertex disjoint paths of equal length.  
We refer to the rays as $R_1, \ldots, R_r$, and number of vertices in $R_i, 1 \leq i \leq r$ is $k+2$.  Let $v_{i1},\ldots,v_{i(k+2)}=r$ denote the set of vertices in $R_i$, and $v_{i1}$ is the leaf. 
  

\xnoindent We will denote the given set system, $k$-subdivided star and the root of the
star by $\cO(\cF)$, $T$ and vertex $r$, respectively.  
% We generalize the interval
% assignment algorithm for an overlap component from a prime matrix in
% \cite{nsnrs09} (algorithm 4 in their paper) to find tree path labeling
% for overlap component $\cO$.
 The algorithm is as follows:runs in two phases. For each hyperedge $X$, the algorithm maintains a 2-tuple of non-negative numbers $(p_1(X),p_2(X))$.  The numbers satisfy the property
that $p_1(X)+p_2(X) \leq |X|$, and at the end of the algorithm for each $X$, $p_1(X) + p_2(X) = |X|$.  The meaning is that the algorithm tracks the path length assigned assigned to
$X$ from at most two rays.  If $X$ gets a path in one ray, then $p_1(X) = |X|$, and $p_2(X)=0$.  In the first phase, it iteratively considers each ray.  At the beginning of each iteration  hyperedges of $\cO$ are classifed
into disjoint sets: 
\begin{enumerate}
\item Those that have been assigned a path which does not contain $r$ in one of the previous iterations.  
\item Those that have been assigned two paths containing $r$ from two different rays in two previous iterations.  
\item Those that have been assigned one path containing $r$ from a ray in one of the previous iterations.  We refer to these as Type-1 hyperedges. For such hyperedges $X$, $p_1(X)$
denote the length of a sub-path of a ray that has been assigned to $X$, $p_2(X) = 0$, and
a parameter $s(X)=|X|-p_1(X)$ refered to as the {\em residue} of $X$ that must be assigned a path.  Also, it is clear that such a path must start at $r$ and must be in a ray different from the one from a path has already been assigned to $X$.  
\item Those that have not been assigned a path in any previous iteration.  We refer to these as Type-2 hyperedges.
\end{enumerate}
In the $i$-th iteration we refer to the set of hyperedges in the last two classes in the above enumeration as the set $\cO_i$.  Note that $\cO_1 = \cO$.  In the $i$-th iteration, hyperedges from $\cO_i$ are assigned paths from $R_i$ using the following rules:
\begin{enumerate}
\item {\bf Step 1:}  
\begin{enumerate}
\item {\bf There are no type-2 edge:}  Select a maximal collection of type-1 hyperedges which form an inclusion chain, and for each such $X$ assign the path in $R_i$ of length $s(X)$starting at $v_{i,k+2}=r$.  $X$ is then removed from $\cO_i$, and $p_2(X)=s(X)$, and $s(X)=0$.If any intersection cardinality is violated because of this assignment,  EXIT reporting the non-existence of a feasible assignment. 
\item {\bf There are no marginal type-2 edges:}  If all type-2 edges of length at most $k+1$, EXIT by saying there is no ICPPL.  Otherwise, Find an $X$ of type-2 such that $|X| \geq k+2$, assign it the unique path starting at $v_{i(k+1)}=r$ of length $k+2$, and set $p_1(X)=k+2, s(X)=|X|-(k+2)$, mark this as a type-1 hyperedge, and add it to $\cO_{i+1}$ after removing it from $\cO_i$.  Next, iteratively, consider each type-1 edge $X \in \cO_i$, and if it intersects a previously assigned $Y$ such that $\cl(Y) \subseteq R_i$, assign a path of length $s(X)$ containing $v_{i(k+2)}=r$, set $p_2(X)=s(X), s(X)=0$, and remove $X$ from $\cO_i$. 

\item {\bf There are marginal type-2 edges:}
Let $X \in \cO_i$ be a type-2 hyperedge such that for all $Y
  \overlap X$, the overlap sets $X \cap Y$ form a single inclusion
  chain \cite{kklv10}. $X$ is refered to as a {\em marginal hyperedge}.  If $|X| \leq k+2$, then $X$ is assigned
the path of length $|X|$ starting at $v_{i1}$, and $p_1(X) = |X|, p_2(X)=0$. $X$ is removed from $O_i$ and not added to $O_{i+1}$.  If $|X| > k+2$, then $p_1(X)=k+2$, and $s(X) = |X| - (k+2)$.   In this case, $X$ is classified as a Type-1 edge, removed from $O_i$, and added to $O_{i+1}$.   
\end{enumerate}
\item {\bf Step 2:} 
Iteratively, a hyperedge $X$ is selected from $\cO_i$ that has a overlap with one of the hyperedges $Y$ such that $\cl(Y) \in R_i$, and a unique path is assigned to $X$.   The path, say $U(X)$, that is assigned can be decided unambiguously since the $X \overlap Y$, and all intersection cardinalities can be preserved only at one of the ends of $\cl(Y)$.  {\bf Can we put a picture for this}.  Let $\cl(X)$ denote the unique path assigned to $X$.  If $X$ is a type-2 hyperedge: an if the unique path of length $|X|$ does not contain $r$, then $p_1(X) = |X|, p_2(X)=0$, and $X$ is
removed from $\cO_i$ and added to $\cO_{i+1}$.  In the case when $\cl(X)$ has to contain $r$, then $p_1(X)$ is the length of the path, $p_2(X)=0$, and $s(X) = |X|-p_1(X)$. Further $X$ is 
classified as a type-1 hyperedge, added to $\cO_{i+1}$ and removed from $\cO_i$.   In the case when $X$ is a type-1 hyperedge, then we check if $U(X)$, which is of length $s(X)$ contains $r$. If it does, then we assign $\cl(X) \leftarrow \cl(X) \cup U(X)$, remove $X$ from $\cO_i$ and do not add it  to $\cO_{i+1}$.  If not, then we {\bf Exit} reporting that an assignment cannot be found.  The iteration ends when no hyperedge in $\cO_i$  has an overlap with a hyperedge assigned to $R_i$.
\end{enumerate}
In the following lemmas we identify a set of necessary conditions for $\cF$ to have an 
ICPPL in the $k$-subdivided star $T$.  If during the execution of the above algorithm, one
of these necessary conditions is not satisfied, the algorithm exits reporting the non-existence of an ICPPL.
\begin{lemma}
Let all hyperedges in $\cO_i$ be type-1 edges.  Then there is a maximal subset $\cT_i \subseteq \cO_i$ with the following properties:
\begin{enumerate}
\item $\cT_i$ form an inclusion chain.
\item For all $X \in \cT_i$, $s(X) \leq k+2$, and There is a an $X \in \cT_i$ such that $s(X)=k+2$.  
\end{enumerate}
\end{lemma}
\begin{lemma}
  If there are no marginal type-2 edges in $\cO_i$, then there exists at least $r-i$ type-2 hyperedges $X \in \cO_i$ such that $|X| \geq k+2$.
%is a exist type-2 hyperedges of length at least $k+2$, and there must be at least $r-i$ such type-2 hyperedge, basically at least one per ray.  Each hyperedge contained in such an hyperedges of length at least $k+2$, is transitively related under the overlap relation to one that gets assigned a path that contains $r$. Basically, the proof will argue that otherwise, there will be a marginal type-2 hyperedge, or that it is infeasible.  Also, it is here we assume that we assume that the size of the two universes is the same. 
\end{lemma}
\begin{lemma}
At the end of Step-1 in the $i$-th iteration, if one hyperedge $X$ of type-2 is such that $\cl(X) \subseteq R_i$, then all other hyperedges in $\cO_i$ are connected to $X$ in the overlap component.
\end{lemma}
\begin{lemma}
At the end of Step-2, If control has exit at any time, there is no ICPPL. If control has not exit, then $R_i$ is saturated.  No hyperedge of $\cO_{i+1}$ will get a path from $R_i$ in 
the future iterations.  No type-2 hyperedge of $\cO_{i+1}$ will get a path from $R_i$.
Basically $R_i$ is done.
\end{lemma}
\begin{lemma}
Finally, we need to prove that the assignment is an ICPPL. Secondly, if there is a permutation then maps sets to paths, then it is indeed an ICPPL, and our algorithm will basically 
find it.  It is a unique assignment upto permutation of the leaves.
\end{lemma}





\section {Acknowlegements} 
We thank the anonymous referees of the WG 2011 committee and our
colleagues who helped us much with the readability of this document.

%\section {Bibliography}
%\bibliographystyle{plainnat} 
\bibliographystyle{alpha} %to have only [i] type of citation 
%\bibliographystyle{agsm}  % another formatting of natbib %% CATS %%
\bibliography{../lib/cop-variants}


\pagebreak
\appendix

\section{Appendix}
%{\bf }
\begin{proof}[Proof of Lemma \ref{lem:setminuscard}]
  Let $P_i = \cl(S_i)$, for all $1 \le i \le  3$.
  $|S_1 \cap (S_2 \setminus S_3)| = |(S_1 \cap S_2) \setminus S_3| =
  |S_1 \cap S_2| - |S_1 \cap S_2 \cap S_3|$. Due to conditions (ii)
  and (iii) of ICPPL, $|S_1 \cap S_2| - |S_1 \cap S_2 \cap S_3| = |P_1
  \cap P_2| - |P_1 \cap P_2 \cap P_3| = |(P_1 \cap P_2) \setminus P_3|
  = |P_1 \cap (P_2 \setminus P_3)|$. Thus lemma is proven. \qed
\end{proof}


%\vspace{7mm}
\noindent
%{\bf:}
\begin{proof}[Proof of Lemma \ref{lem:feasible}]
  We will prove this by mathematical induction on the number of
  iterations. The base case $(\cF_0, \cl_0)$ is feasible since it is
  the input itself which is given to be feasible. Assume the lemma is
  true till $j-1$th iteration. i.e. every hypergraph isomorphism
  $\phi: supp\left(\cF_{j-1}\right) \rightarrow V\left(T \right)$ that
  defines $(\cF, \cl)$'s feasibility, is such
  that the induced path labeling on $\cF_{j-1}$,
  $\cl_{\phi[{\cF_{j-1}}]}$ is equal to $\cl_{j-1}$. We will prove
  that $\phi$ is also the bijection that makes $(\cF_j, \cl_j)$
  feasible. Note that $supp(\cF_{j-1}) = supp(\cF_{j})$ since the new
  sets in $\cF_j$ are created from basic set operations to the sets in
  $\cF_{j-1}$. For the same reason and $\phi$ being a bijection, it is
  clear that when applying the $\phi$ induced path labeling on
  $\cF_j$, $ \cl_{\phi[{\cF_{j}}]}(S_1 \setminus S_2) =
  \cl_{\phi[{\cF_{j-1}}]}(S_1) \setminus
  \cl_{\phi[{\cF_{j-1}}]}(S_2)$. Now observe that $ \cl_j(S_1
  \setminus S_2) = \cl_{j-1}(S_1) \setminus \cl_{j-1}(S_2) =
  \cl_{\phi[{\cF_{j-1}}]}(S_1) \setminus
  \cl_{\phi[{\cF_{j-1}}]}(S_2)$. Thus the induced path labeling
  $\cl_{\phi[{\cF_{j}}]} = \cl_{j}$. Therefore lemma is proven.  \qed
\end{proof}

%\vspace{7mm}
%\noindent
%{\bf:}
% \begin{proof}[Proof of Lemma \ref{lem:invar1} (details)]
%   :
%   \begin{enumerate}[\textreferencemark]
%   \item {\em Invariant III}
%   \item [Case 3:] {\em Invariant IV}
%     \begin{enumerate}
%     \item [Case 3.2:] {\em Only R is a new set:}\\
%       If $R = S_1 \cap S_2$, Consider, $|\cl_{j-1}(S_1) \cap
%       \cl_{j-1}(S_2) \cap \cl_{j-1}(R') \cap \cl_{j-1}(R'')|$. We know
%       from lemma \ref{lem:fourpaths} that the intersection of these
%       four paths is same as the intersection of three distinct paths
%       among the four.  Let us call these four paths $P_1,P_2, P_3,P_4$
%       and without loss of generality, let it be that $\displaystyle
%       \cap_{i=1}^4 P_i = \cap_{i=1}^3 P_i$. Further $|\cap_{i=1}^4
%       P_i|=| S_1 \cap S_2 \cap R'|$ by the invariant IV of the
%       induction hypothesis.
%       Therefore, it follows that $|\cap_{i=1}^4 P_i| \geq |S_1 \cap
%       S_2 \cap R' \cap R''|$.\\ 
%       \begin{comment}
%         Next, we write $\displaystyle |\cap_{i=1}^4 P_i| = |P_4| +
%         |\displaystyle \cap_{i=1}^3 P_i| - |P_4 \cup \cap_{i=1}^3
%         P_i|$. Clearly $\displaystyle P_4 \cup \cap_{i=1}^3 P_i =
%         P_4$.  By induction hypothesis Invariant I and IV, We can now
%         write $\displaystyle |\cap_{i=1}^4 P_i| = |S_4| +
%         |\cap_{i=1}^3 S_i| - |S_4|$.  Since $\cl_{j-1}$ is an ICPPL in
%         which $S_4$ is mapped to $P_4$, and $P_4$ contains
%         $\displaystyle \cap_{i=1}^3 P_i$, it follows that $|S_4|=|P_4|
%         \geq|\cap_{i=1}^3 P_i| = |S_1 \cap S_2 \cap R'| \geq |S_1 \cap
%         S_2 \cap R' \cap R''|$.  Therefore, $\displaystyle
%         |\cap_{i=1}^4 P_i| \leq |S_1 \cap S_2 \cap R' \cap R''|$, and
%         equality of these two terms follows because we have also
%         proved the inequality in the
%         opposite direction. It now follows that $|\cl_j(R) \cap
%         \cl_j(R') \cap \cl_j(R'')| = |\cl_{j-1}(S_1) \cap
%         \cl_{j-1}(S_2) \cap \cl_{j-1}(R') \cap \cl_{j-1}(R'')| =
%         |\cap_{i=1}^4 P_i| = |S_1 \cap S_2 \cap R' \cap R''| = |R \cap
%         R' \cap R''|$. This completes induction hypothesis in this
%         case. \\ 
%       \end{comment}
%       If $R = S_1 \setminus S_2$, a similar argument using Lemma
%       \ref{lem:setminuscard} and the induction hypothesis completes
%       the proof of this case.\\ 
%       Thus Invariant IV proven.
%    \end{enumerate}
%   \end{enumerate}
% \end{proof}

% %\vspace{7mm}
% \noindent
% %{\bf:}
% \begin{proof}[Proof of lemma \ref{lem:invar3} (alternate)]
%   \xnoindent For the rest of the proof we use mathematical induction
%   on the number of iterations $j$. As before, the term ``new sets''
%   will refer to the sets added in $\cF_j$ in the $j$th iteration,
%   i.e. $S_1  \setminus \{x\}$ and $\{x\}$ as defined in line
%   \ref{uniqueleaf}.\\ 
%   For $\cF_0, \cl_0$ all invariants hold because it is output from
%   algorithm \ref{perms} which is an ICPPL. Hence base case is proved.
%   Assume the lemma holds for the $j-1$th iteration. Consider $j$th
%   iteration.  \xnoindent
%   \begin{enumerate}[\textreferencemark]
%   \item {\em Invariant} I/II
%     \begin{enumerate}[{I/II}a $|$]
%     \item {\em $R$ is not a new set.} If $R$ is in $\cF_{j-1}$, then
%       by induction hypothesis this case is trivially proven.
%     \item {\em $R$ is a new set.} If $R$ is in $\cF_{j}$ and not in
%       $\cF_{j-1}$, then it must be one of the new sets added in
%       $\cF_j$. Removing a leaf $v$ from path $\cl_{j-1}(S_1)$ results
%       in another path. Moreover, $\{v\}$ is trivially a path. Hence
%       regardless of which new set $R$ is, by definition of
%       $\cl_j$, $\cl_{j}(R)$ is a path. Thus invariant I is proven.\\
%       We know $|S_1| = |\cl_{j-1}(S_1)|$, due to induction
%       hypothesis. Therefore $|S_1 \setminus \{x\}| = |\cl_{j-1}(S_1)
%       \setminus \{v\}|$. This is because $x \in S_1$ iff $v \in
%       \cl_{j-1}(S_1)$. If $R = \{x\}$, invariant II is trivially
%       true. Thus invariant II is proven.
%     \end{enumerate}
%   \item {\em Invariant} III
%     \begin{enumerate} [{III}a $|$]
%     \item {\em $R$ and $R'$ are not new sets.} Trivially true by
%       induction hypothesis.
%     \item {\em Only one, say $R$, is a new set.}  By definition,
%       $\cl_{j-1}(S_1)$ is the only path with $v$ and $S_1$ the only
%       set with $x$ in the previous iteration, hence $|R' \cap (S_1
%       \setminus \{x\})| = |R' \cap S_1|$ and $|\cl_{j-1}(R') \cap
%       (\cl_{j-1}(S_1) \setminus \{v\})| = |\cl_{j-1}(R') \cap
%       \cl_{j-1}(S_1)|$ and $|R' \cap \{x\}| = 0$, $|\cl_{j-1}(R') \cap
%       \{v\}| = 0$. Thus case proven.
%     \item {\em $R$ and $R'$ are new sets.} By definition, the new sets
%       and their path images in path label $l_j$ are disjoint so $|R
%       \cap R'| = |l_j(R) \cap l_j(R)| = 0$. Thus case proven.
%     \end{enumerate}
%   \item {\em Invariant} IV
%     \begin{enumerate}[{IV}a $|$]
%     \item {\em $R$, $R'$ and $R''$ are not new sets.} Trivially true
%       by induction hypothesis.
%     \item {\em Only one, say $R$, is a new set.}  By the same argument
%       used to prove invariant III, $|R' \cap R'' \cap (S_1 \setminus
%       \{x\})| = |R' \cap R'' \cap S_1|$ and $|\cl_{j-1}(R') \cap
%       \cl_{j-1}(R'') \cap (\cl_{j-1}(S_1) \setminus \{v\})| =
%       |\cl_{j-1}(R') \cap \cl_{j-1}(R'') \cap \cl_{j-1}(S_1)|$. Since
%       $R', R'', S_1$ are all in $\cF_{j-1}$, by induction hypothesis
%       of invariant IV, $|R' \cap R'' \cap S_1| = |\cl_{j-1}(R') \cap
%       \cl_{j-1}(R'') \cap \cl_{j-1}(S_1)|$.  Also, $|R' \cap R'' \cap
%       \{x\}| = |\cl_{j-1}(R') \cap \cl_{j-1}(R'') \cap \{v\}|$ = 0.
%     \item {\em At least two of $R, R', R''$ are new sets.}  If two or
%       more of them are not in $\cF_{j-1}$, then it can be verified
%       that $|R \cap R' \cap R''| = |\cl_j(R) \cap \cl_j(R') \cap
%       \cl_j(R'')|$ since the new sets in $\cF_j$ are disjoint.
% %       \tnote[AS]{The following is not correct. While loop only
% %         handles one leaf at a time. Deleted} \remove[AS]{ or as follows:
% %         assuming $R, R' \notin \cF_{j-1}$ and new sets are derived
% %         from $S_1, S_2 \in \cF_{j-1}$ with $x_1, x_2$ exclusively in
% %         $S_1, S_2$, $\cl_j(\{x_1\})=\{v_1\}, \cl_j(\{x_2\}) = \{v_2\}$
% %         where $ \{x_1\}, \{x_2\} \in \cF_j $ thus $v_1, v_2$ are
% %         exclusively in $\cl_j(\{x_1\})$, $\cl_j(\{x_2\})$
% %         respectively. It follows that $|R \cap R' \cap R''| =$ $ |(S_1
% %         \setminus \{x_1\}) \cap (S_2 \setminus \{x_2\}) \cap R''| =$ $
% %         |S_1 \cap S_2 \cap R''| = $ $|\cl_{j-1}(S_1) \cap
% %         \cl_{j-1}(S_2) \cap \cl_{j-1}(R'')| = |(\cl_{j-1}(S_1)
% %         \setminus \{v_1\}) \cap (\cl_{j-1}(S_2) \setminus \{v_2\})
% %         \cap \cl_{j-1}(R'')| = |\cl_j(R) \cap \cl_j(R') \cap
% %         \cl_j(R'')|$}
%        Thus invariant IV is also proven. 
%     \end{enumerate}  
%   \end{enumerate} 
% \qed
% \end{proof}

% \begin{proof} [Proof of theorem \ref{th:perm} (alternate)]  
%   We find $\phi$ part by part by running
%   algorithms \ref{perms} and \ref{leafasgn} one after the other in a
%   loop. After each iteration we calculate an exclusive subset of the
%   bijection $\phi$, namely that which involves all the leaves of the
%   tree in that iteration. Then all the leaves are pruned off the tree
%   before the next iteration. The loop terminates when the pruned tree
%   has no nodes at which point the rest of the bijection is obvious,
%   thus completing $\phi$. This is the brief outline of the
%   algorithm. Now we see it in detail below.

%  \xnoindent First, the given ICPPL $(\cF, \cl)$ and tree $T$ are
%   given as input to Algorithm \ref{perms}. This yields a ``filtered''
%   ICPPL as the output which is input to Algorithm \ref{leafasgn}.  Let
%   the output of Algorithm \ref{leafasgn} be $(\cF',\cl')$. We define a
%   bijection $\phi_1: Y_1 \rightarrow V_1$ where $Y_1 \subseteq
%   supp(\cF)$ and $V_1 = \{v \mid v \text{ is a leaf in } $T$\}$.  It
%   can be observed that the output of Algorithm \ref{leafasgn} is a set
%   of path assignments to sets and one-to-one assignment of elements of
%   $U$ to each leaf of $T$. These are defined below as $\cl_1$ and
%   $\phi_1$ respectively.

%   \begin{align*}
%     \cl_1(S) = \cl'(S), &\text{ when $\cl'(S)$ has non-leaf vertices} \\
%     \phi_1(x) = v,  &\text{ when }\cl'(S) = \{v\}, v \in V_1,\\
%     &\text{ and } S = \{x\}
%   \end{align*}
%   Consider the tree $T_1 = T[V(T) \setminus V_0]$, i.e. it isomorphic
%   to $T$ with its leaves removed. Let $U_1$ be the universe of the
%   subsystem that is not mapped to a leaf of $T$, i.e. $U_1 = supp(\cF)
%   \setminus Y_1$ .

% % To be precise, it would be of the form $\cB_0 =
% % \cA_0 \cup \cL_0$. The leaf assignments are defined in $\cL_0
% % = \{ (x_i,v_i) \mid x_i \in U, v_i \in T, x_i \ne x_j, v_i \ne v_j, i
% % \ne j, i,j \in [k] \}$ where $k$ is the 
% % number of leaves in $T$. The path assignments are defined in $\cA_0
% % \subseteq \{(S_i',P_i') \mid S_i' \subseteq U_0, P_i' \text{ is a path
% %   from } T_0\}$ 
% % where $T_0$ is the tree obtained by removing all the
% % leaves in $T$ and
% % $U_0 = U \setminus \{ x \mid x \text{ is assigned to
% %   a leaf in }\cL_0 \}$. 
% \xnoindent
% Let $\cF_1$ be the set system induced by $\cF'$ on universe $U_1$.
% Now we have a subproblem of finding the
% hypergraph isomorphism for $(\cF_1, \cl_1)$ with tree $T_1$.
% % for the path assignment $\cA_0$ which has paths from tree
% % $T_0$ and sets from universe $U_0$. Now we repeat the procedure and
% % the path assignment $\cA_0$ and tree $T_0$
% \xnoindent Now we repeat Algorithm \ref{perms} followed by Algorithm
% \ref{leafasgn} on $(\cF_1, \cl_1)$ with tree $T_1$. As before we
% define $l_2$ in terms of $l_1$, $\phi_2$ in terms of $V_2 = \{v \mid v
% \text{ is a leaf in } T_1\}$, prune the tree $T_1$ to get $T_2$ and so
% on.  Thus in the $i$th iteration, $T_i$ is the pruned tree, $\cl_i$ is
% a feasible path labeling to $\cF_i$ if $(\cF_{i-1}, \cl_{i-1})$ is
% feasible, $\phi_i$ is the leaf labeling of leaves of
% $T_{i-1}$. Continue this until some $d$th iteration for the smallest
% value $d$ such that $T_d$ is empty. From the lemma \ref{lem:invar1} and \ref{lem:invar3} we know
% that $(\cF_d, \cl_d)$ is an ICPPL. 

% \xnoindent Now we have exactly one bijection $\phi_j$,
% $j \in [d]$ defined for every element $x
% \in U$ into some vertex $v \in V(T)$. The bijection for the ICPPL,
% $\phi: U \rightarrow V(T)$ is constructed by the following definition.
% \vspace{\topshrink}
% \begin{align*}
%   \phi(x) &= \phi_i(x) \\
%   &\text{ where $x$ is in the domain of } \phi_i, i \in [d] %[d+1]
% \end{align*}
% It can be verified easily that $\phi$ is the required hypergraph
% isomorphism. Thus the theorem is proven.  \qed
% \end{proof}



\end{document}





