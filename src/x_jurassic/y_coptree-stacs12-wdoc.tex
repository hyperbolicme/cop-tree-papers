%%
% Author: N S Narayanaswamy and Anju Srinivasan
%%


\documentclass[a4paper,UKenglish,numberwithinsect]{lipics}
  %for A4 paper format use option "a4paper", for US-letter use option "letterpaper"
  %for british hyphenation rules use option "UKenglish", for american hyphenation rules use option "USenglish"
 % for section-numbered lemmas etc., use "numberwithinsect"
 
\usepackage{microtype} %if unwanted, comment out or use option "draft"

%\graphicspath{{./graphics/}}%helpful if your graphic files are in another directory

\bibliographystyle{plain}% the recommended bibstyle




\usepackage{latexsym}
\usepackage{amssymb}
\usepackage{amsfonts}
\usepackage{amsmath}
\usepackage{comment}
\usepackage{epsfig}
\usepackage{graphicx}
\usepackage{epstopdf}
\usepackage{algorithm}
\usepackage{algorithmic}
% \usepackage{natbib}

%%%            %%%
%  TrackChanges  %
%%%            %%%
\usepackage[finalnew]{trackchanges}
% finalold
%   Ignore all of the edits. 
%   The document will look as if the edits had not been added.
% finalnew
%   Accept all of the edits. 
%   Notes will not be shown in the final output.
% footnotes
%   Added text will be shown inline. Removed text and notes will be shown as footnotes. 
%   This is the default option.
% margins
%   Added text will be shown inline. Removed text and notes will be
%   shown in the margin. Margin notes will be aligned with the edits when possible.
% inline
%   All changes and notes will be shown inline.
\addeditor{AS} 
\addeditor{NSN} 


\DeclareMathAlphabet{\mathpzc}{OT1}{pzc}{m}{it}
\DeclareMathAlphabet{\mathcalligra}{T1}{calligra}{m}{n}

% for Natbib
%\bibpunct{(}{)}{;}{a}{,}{,}

\def\Remark{\noindent{\bf Remark:~}}
\long\def\denspar #1\densend {#1}
\def\DEF{\stackrel{\rm def}{=}}

% \def\mathbi#1{\textbf{\em #1}}  % bold italic font in math mode
% use \mathfrak, \mathsf, \mathbb,\mathpzc, \mathcalligra for differnt math fonts. some
%  have small letters too.

%%% string defs
\def\cA{{\cal A}}
\def\cB{{\cal B}}
\def\cC{{\cal C}}
\def\cD{{\cal D}}
\def\cE{{\cal E}}
\def\cF{{\cal F}}
\def\cG{{\cal G}}
\def\cH{{\cal H}}
\def\cI{{\cal I}}
\def\cJ{{\cal J}}
\def\cK{{\cal K}}
\def\cL{{\cal L}}
\def\cM{{\cal M}}
\def\cN{{\cal N}}
\def\cO{{\cal O}}
\def\cP{{\cal P}}
\def\cQ{{\cal Q}}
\def\cR{{\cal R}}
\def\cS{{\cal S}}
\def\cT{{\cal T}}
\def\cU{{\cal U}}
\def\cV{{\cal V}}
\def\cW{{\cal W}}
\def\cX{{\cal X}}
\def\cY{{\cal Y}}
\def\cZ{{\cal Z}}
\def\hA{{\hat A}}
\def\hB{{\hat B}}
\def\hC{{\hat C}}
\def\hD{{\hat D}}
\def\hE{{\hat E}}
\def\hF{{\hat F}}
\def\hG{{\hat G}}
\def\hH{{\hat H}}
\def\hI{{\hat I}}
\def\hJ{{\hat J}}
\def\hK{{\hat K}}
\def\hL{{\hat L}}
\def\hP{{\hat P}}
\def\hQ{{\hat Q}}
\def\hR{{\hat R}}
\def\hS{{\hat S}}
\def\hT{{\hat T}}
\def\hX{{\hat X}}
\def\hY{{\hat Y}}
\def\hZ{{\hat Z}}
\def\eps{\epsilon}
\def\C{{\mathcal C}}
\def\F{{\mathcal F}}
\def\A{{\mathcal A}}
\def\H{{\mathcal H}}
\def\bI{\mathbb I}
\def\bO{\mathbb O}
\def\cl{\mathpzc{l}}
\def\overlap{\between}

\def\hd{\hat{\delta}}
\def\Lr{\Leftrightarrow}
\def\If{{\bf if }}
\def\Then{{\bf then }}
\def\Else{{\bf else }}
\def\Do{{\bf do }}
\def\While{{\bf while }}
\def\Continue{{\bf continue }}
\def\Repeat{{\bf repeat }}
\def\Until{{\bf until }}
\def\eqdef{\stackrel {\triangle}{=}}
\def\squarebox#1{\hbox to #1{\hfill\vbox to #1{\vfill}}}
\def\cmt{5cm}


%%% new/renew commands
% \renewcommand{\algorithmiccomment}[1]
% { 
%   \vspace {1mm}
%   \hfill
%   {\small
%   \begin{tabular}{|r}
%    \parbox[right]{7cm}{ \space \tt{ #1 }}\\  % {\tt /* #1 */}    \hspace{2mm}
%   \end{tabular}
%   }
% }
\renewcommand{\algorithmiccomment}[1]
{ 
  \vspace {1mm}
%  \hfill
  {\small
  { \tt{/* #1 */}}\\  % {\tt /* #1 */}    \hspace{2mm}
 }
}


% commands for theorems etc.
% \newtheorem{theorem}{Theorem}[section]
% \newtheorem{lemma}[theorem]{Lemma}
% \newtheorem{proposition}[theorem]{Proposition}
% \newtheorem{corollary}[theorem]{Corollary}
% \newenvironment{proof}[1][Proof]{\begin{trivlist}
% \item[\hskip \labelsep {\bfseries #1}]}{\end{trivlist}}
% \newtheorem{definition}{Definition}[section]
\newtheorem{observation}{Observation}
% \newcommand{\qed}{\hfill \mbox{\raggedright \rule{.07in}{.1in}}}

 % comment in algorithmic
\newcommand{\Eqr}[1]{Eq.~(\ref{#1})}
\newcommand{\diff}{\ne}
\newcommand{\OO}[1]{O\left( #1\right)}
\newcommand{\OM}[1]{\Omega\left( #1 \right)}
\newcommand{\Prob}[1]{\Pr\left\{ #1 \right\}}
\newcommand{\Set}[1]{\left\{ #1 \right\}}
\newcommand{\Seq}[1]{\left\langle #1 \right\rangle}
\newcommand{\Range}[1]{\left\{1,\ldots, #1 \right\}}
\newcommand{\ceil}[1]{\left\lceil #1 \right\rceil}
\newcommand{\floor}[1]{\left\lfloor #1 \right\rfloor}
\newcommand{\ignore}[1]{}
\newcommand{\eq}{\equiv}
\newcommand{\abs}[1]{\left| #1\right|}
\newcommand{\set}[1]{\left\{ #1\right\}}
\newcommand{\itoj}{{i \rightarrow j}}
\newcommand{\view}{\mbox{$COMM$}}
\newcommand{\pview}{\mbox{$PView$}}
\newcommand{\vx}{\mbox{${\vec x}$}}
\newcommand{\vy}{\mbox{${\vec y}$}}
\newcommand{\vv}{\mbox{${\vec v}$}}
\newcommand{\vw}{\mbox{${\vec w}$}}
\newcommand{\vb}{\mbox{${\vec b}$}}
\newcommand{\basic}{\mbox{\sc Basic}}
\newcommand{\WR}{\mbox{$\lfloor wr \rfloor$}}
\newcommand{\guarantee}{\mbox{\sc BoundedDT}}
\newcommand{\sq}{{\Delta}}
\newcommand{\Smin}{{S_{0}}}
\newcommand{\outt}{{D^{^+}}}
\newcommand{\outtp}{{\overline{D^{^+}}}}
\newcommand{\inn}{{D^{^-}}}
\newcommand{\innp}{{\overline{D^{^-}}}}
\newcommand{\indexx}{{\gamma}}
\newcommand{\D}{{D}}
\newenvironment{denselist}{
  \begin{list}{(\arabic{enumi})}{\usecounter{enumi}
      \setlength{\topsep}{0pt} \setlength{\partopsep}{0pt}
      \setlength{\itemsep}{0pt} }}{\end{list}}
\newenvironment{denseitemize}{
  \begin{list}{$\bullet$}{ \setlength{\topsep}{0pt}
      \setlength{\partopsep}{0pt} \setlength{\itemsep}{0pt}
    }}{\end{list}}
\newenvironment{subdenselist}{
  \begin{list}{(\arabic{enumi}.\arabic{enumii})}{ \usecounter{enumii}
      \setlength{\topsep}{0pt} \setlength{\partopsep}{0pt}
      \setlength{\itemsep}{0pt} }}{\end{list}}

% Review comment. in a parbox that wraps before right margin. vertical
% lines on the left for easy identification.
%%\newcommand{\rcomment}[1]{}
\newcommand{\rcomment}[1]
{ \begin{tabular}{ |||||r| } 
    \hline
    \parbox{\cmt}{\tiny{#1}}\\
    \hline
  \end{tabular}
}

%%\newcommand{\rfootnote}[1] {} %{\footnote{{#1}}}
\newcommand{\rfootnote}[1] {\footnote{{#1}}}

% D O C U M E N T
\begin{document}

\title{Tree Path Labeling of Path Hypergraphs - A Generalization of
  Consecutive Ones Property}
\titlerunning{Tree Path Labeling of Path Hypergraphs}

\author[1]{N.S. Narayanaswamy and Anju Srinivasan}
%\author[2]{}
\affil[1]{Computer Science and Engineering Department,\\
  Indian Institute of Technology Madras, Chennai - 600036, India\\
  \texttt{swamy@cse.iitm.ernet.in, asz@cse.iitm.ac.in}}
% \affil[2]{Computer Science and Engineering Department,\\
%   Indian Institute of Technology Madras, Chennai - 600036, India\\
%   \texttt{asz@cse.iitm.ac.in}}
\authorrunning{N.S. Narayanaswamy and A. Srinivasan} %optional. First: Use abbreviated first/middle names. Second (only in severe cases): Use first author plus 'et. al.'

\Copyright[nc-nd]%choose "nd" or "nc-nd"
          {N. S. Narayanaswamy and Anju Srinivasan}

\subjclass{XXXXXXXX TBD XXXXXXXX Dummy classification}% mandatory: Please choose ACM 1998 classifications from http://www.acm.org/about/class/ccs98-html . E.g., cite as "F.1.1 Models of Computation". 
\keywords{XXXXXXXX TBD XXXXXXXX}% mandatory: Please provide 1-5 keywords
%%%%%%%%%%%%%%%%%%%%%%%%%%%%%%%%%%%%%%%%%%%%%%%%%%%%%%%%%



\maketitle


\begin{abstract}
  We consider the following constraint satisfaction problem. Given (i)
  a set system $\F \subseteq$ $(2^{U} \setminus \emptyset)$ of a finite
  set $U$ of cardinality $n$, (ii) a tree $T$ of size $n$ and (iii) a
  bijection $\cl$, defined as {\em tree path labeling}, mapping the
  sets in $\cF$ to paths in $T$, does there exist at least one
  bijection $\phi:U \rightarrow V(T)$ such that for each $S \in \cF$,
  $\{\phi(x) \mid x \in S\} = \cl(S)$?  A tree path labeling of a set
  system is called {\em feasible} if there exists such a bijection
  $\phi$.  In this paper, we characterize feasible tree path labeling
  of a given set system to a tree.  This result is a natural
  generalization of results on matrices with the Consecutive Ones
  Property. COP is a special instance of tree path labeling problem
  when $T$ is a path.  We also present an algorithm to find the tree
  path labeling of a given set system when $T$ is a {\em
    $k$-subdivided star} as well as a characterization of set systems
  which have a feasible tree path labeling.
\end{abstract}

\section{Introduction}
\noindent
Consecutive ones property (COP) of binary matrices is a widely studied
combinatorial problem. The problem is to rearrange rows (columns) of a
binary matrix in such a way that every column (row) has its $1$s occur
consecutively. If this is possible the matrix is said to have the COP.  It
has several practical applications in diverse fields including
scheduling \cite{hl06}, information retrieval \cite{k77} and
computational biology \cite{abh98}.  Further, it is a tool in graph
theory \cite{mcg04} for interval graph recognition, characterization of
hamiltonian graphs, and in integer linear programming \cite{ht02, hl06}.
Recognition of COP is polynomial time solvable by several
algorithms. PQ trees \cite{bl76}, variations of PQ
trees \cite{mm09, wlh01, wlh02, mcc04}, ICPIA \cite{nsnrs09} are the main
ones.

\noindent
The problem of COP testing can be easily seen as a simple constraint
satisfaction problem involving a system of sets from a universe. Every
column of the binary matrix can be converted into a set of integers
which are the indices of the rows with $1$s in that column. When
observed in this context, if the matrix has the COP, a reordering of its
rows will result in sets that have only consecutive integers. In other
words, the sets after reordering are intervals. Indeed the problem now
becomes finding 
interval assignments to the given set system \cite{nsnrs09} with a
single permutation of the universe (set of row indices) which permutes each
set to its interval. The result in \cite{nsnrs09} characterize
interval assignments to the sets which can be obtained from a single
permutation of the rows.  They show that for each set, the cardinality
of the interval assigned to it must be same as the cardinality of the set,
and the intersection cardinality of any two sets must be same as the
interesction cardinality of the corresponding intervals.  While this
is naturally a necessary condition, \cite{nsnrs09} shows this is indeed
sufficient.  Such an interval assignment is called an Intersection
Cardinality Preserving Interval Assignment (ICPIA).  Finally, the idea of
decomposing a given 0-1 matrix into prime matrices to check for COP is
adopted from 
\cite{wlh02} to test if an ICPIA exists for a given set system.\\
{\bf Our Work.}
A natural generalization of the interval assignment problem is
feasible tree path labeling problem of a set system. The problem is
defined as follows - given a set system 
$\cF$ from a universe $U$ and a tree $T$, does there exist a bijection
from the $U$ to the vertices of $T$ such that each set in the system
maps to a path in $T$.  We refer to this as the tree path labeling
problem for an input $(\cF,T)$ pair. As a special case if $T$ is a path
the problem becomes the interval assignment problem.  We focus on the
question of generalizing the notion of an ICPIA \cite{nsnrs09} to
characterize feasible path assignments.  We show that for a given set
system $\cF$, a tree $T$, and an assignment of paths from $T$ to the sets,
there is a bijection between $U$ and $V(T)$ if and only if all
intersection cardinalities among any three sets (not necessarily distinct)
is same as the intersection cardinality of the paths assigned to them
and the input runs a filtering algorithm (described in
this paper) without prematurely exiting.
This characterization is proved constructively and it gives a natural
data structure that stores all the relevant bijections between $U$ and
$V(T)$.  
Further, the filtering algorithm is also an efficient algorithm to test if a
tree path labeling to the set system is feasible.  This 
generalizes the result in \cite{nsnrs09}.\\
\noindent
It is an interesting fact that for a matrix with the COP, the
intersection graph of the corresponding set system is an interval
graph.  A similar connection to a subclass of chordal graphs and a
superclass of interval graphs exists for the generalization of COP.
In this case, the intersection graph of the corresponding set system
must be a {\em path graph}. Chordal graphs are of great significance,
extensively studied, and have several applications.  One of the well
known and interesting properties of a chordal graphs is its connection
with intersection graphs \cite{mcg04}. For every chordal graph, there
exists a tree and a family of subtrees of this tree such that the
intersection graph of this family is isomorphic to the chordal graph
\cite{plr70,gav78,bp93}.  These trees when in a certain format, are
called clique trees \cite{apy92} of the chordal graph. This is a
compact representation of the chordal graph. There has also been work
done on the generalization of clique trees to clique hypergraphs
\cite{km02}.  If the chordal graph can be represented as the
intersection graph of paths in a tree, then the graph is called path
graph \cite{mcg04}.  Therefore, it is clear that if there is a
bijection from $U$ to $V(T)$ such that for every set, the elements in
it map to vertices of a unique path in $T$, then the intersection
graph of the set system is indeed a path graph.  However, this is only
a necessary condition and can be checked efficiently because path
graph recognition is polynomial time
solvable\cite{gav78,aas93}. Indeed, it is possible to construct a set
system and tree, such that the intersection graph is a path graph, but
there is no bijection between $U$ and $V(T)$ such that the sets map to
paths. Path graph isomorphism is known be isomorphism-complete, see
for example \cite{kklv10}. An interesting area of research would be to see
what this result tells us about the complexity of the tree path
labeling problem (not covered in this paper). In the later part of
this paper, we decompose our search for a bijection between $U$ and
$V(T)$ into subproblems.  Each subproblem is on a set subsystem in which
for each set, there is another set in the set subsystem with which the
intersection is {\em strict} - i.e., there is a non-empty intersection, but
neither is contained in the other.  This is in the spirit of results
in \cite{wlh02,nsnrs09} where to test for the COP in a given matrix,
the COP problem is solved on an equivalent set of prime matrices.  \\
{\bf Roadmap.} In Section \ref{sec:prelims} we present the necessary
preliminaries, in Section \ref{sec:feasible} we present our
characterization of feasible tree path assignments, and in Section
\ref{sec:decompos} we present the characterizing subproblems for
finding a bijection between $U$ and $V(T)$ such that sets map to tree
paths. In Section \ref{sec:ksubdivstar} we present a polynomial time
algorithm to find the tree path labeling of a given set system from a
given $k$-subdivided tree.
% \noindent
% It has been long known that interval graph recognition is in
% logspace \cite{rei84}. Recently interval graph isomorphism was also
% shown to be logspace decidable using a logspace canonization algorithm
% by \cite{kklv10}. 
% This result is built on top of logspace results of
% undirected graph connectivity \cite{rei08}, logspace tractability
% using a certain logical formalism called FP+C and modular
% decomposition of interval graphs \cite{lau10} etc.
% Interval graphs are closely connected to binary matrices with COP. The
% maximal clique vertex incidence matrix (matrix with rows representing
% maximal cliques and columns representing vertices of a graph) has COP
% on columns iff the graph is an interval graph \cite{fg65}. This follows
% from the interval graph characterization by \cite{gh64}. Due to this
% close relation it is natural to see if consecutive ones property can be
% tested in logspace. \\
% \noindent
% We also explore some extensions of the interval assignment problem in
% \cite{nsnrs09}, namely caterpillar path assignment problem.

% We present a logspace algorithm here that uses the
% ICPIA characterization of binary matrices with COP (set system
% associated with such a matrix) \cite{nsnrs09}.


% \section{Preliminaries}
% A {\em hypergraph} $\cH=(V,E)$  has vertex set $V=\{x_1,x_2, \dots x_n\}$
% and edge set $E \subseteq V$.

% \noindent
% The collection $\F = \{S_i \mid S_i \subseteq U, S_i \ne \emptyset, i \in
% I\}$ is a {\em set system} of a universe $U$.

% \noindent
% Consider a {\em binary matrix} $M$ of order $n \times m$.  The set
% sysetm corresponding to the binary matrix is $\cF_M = \{S_i \mid S_i
% \subseteq U, i \in [m]\}$ where $U = \{x_i \mid i \in [n]\}$ such that
% $x_j \in S_i$ iff $M_{ij} = 1$.

\section{Preliminaries} \label{sec:prelims} 
\noindent
In this paper, the set $\F \subseteq (2^{U} \setminus \emptyset)$ is a
{\em set system} of a universe $U$ with $|U| = n$. 
The {\em support} of a set system $\F$ denoted by  $supp(\cF)$ is the
union of all the sets in $\F$, i.e., $supp(\F) =
\bigcup_{S \in \F}S$.
For the purposes of this paper, a set system is required to ``cover'' the universe,
i.e. $ supp(\cF) = U$. In brief, the {\em intersection graph} $\bI(\cF)$ of a
set system $\cF$ is a graph such that its vertex set has a bijection
to $\cF$ and there exists an edge between two vertices iff their
corresponding sets have a non-empty
intersection \cite{mcg04}. \\
\noindent
The graph $T$ represents a given tree with $|V(T)| = n$. 
A {\em path system} $\cP$ is a set system of paths from
$T$ i.e., $\cP = \{P \mid P \subseteq V, \text{ } T[P]
\text{ is a path.} \}$
\noindent
A {\em star} graph is a complete bipartite graph
$K_{1,l}$ which is clearly a tree and $l$ is the number of leaves. The vertex with maximum degree is called the {\em center} of
the star and the edges are called {\em rays} of the star.
A {\em $k$-subdivided star} is a star with all its rays subdivided exactly
$k$ times. The path from the center to a leaf is called a ray of a
$k$-subdivided star and they are all of length $k+2$.\\
\noindent
A graph $G$ that is isomorphic to the intersection graph $\bI(\cP)$ of a
path system $\cP$ of $T$, is a {\em path graph}. This
isomorphism gives a bijection $\cl': V(G) \rightarrow \cP$ and is
called a path labeling of $G$. Moreover, for the purposes of this paper, we
require that in a path labeling, $supp(\cP) = V(T)$. 
% This path system $\cP$ is called a {\em path
% representation} of $G$ and may also be denoted by $G^\cl$. 
If $G = \bI(\cF)$ where $\cF$ is any set system, then clearly there is
a bijection $\cl: \cF \rightarrow \cP$ such that $\cl(S) = \cl'(v_S)$
where $v_S$ is the vertex corresponding to set $S$ in $\bI(\cF)$ for
any $S \in \cF$. 
This bijection $\cl$ is called the {\em path labeling} of set system
$\cF$ and the path system $\cP$ may be alternatively denoted as
$\cF^\cl$. 
 
\noindent
A set system $\cF$ can be alternatively represented by a {\em
  hypergraph} $\H_\cF$ whose vertex set is $supp(\cF)$ and hyperedges
are the sets in $\cF$. This is a known representation for interval
systems in literature \cite{bls99}.  We extend this definition here to
path systems.  Two hypergraphs $\cH$, $\cK$ are said to be isomorphic
to each other, denoted by $\cH \cong \cK$, iff there exists a
bijection $\phi: supp(\cH) \rightarrow supp(\cK)$ such that for all
sets $H \subseteq supp(\cH)$, $H$ is a hyperedge in $\cH$ iff $K$ is a
hyperedge in $\cK$ where $K = \{\phi(x) \mid x \in H\}$.  If $\H_\cF
\cong \H_\cP$ where $\cP$ is a path system, then $\H_\cF$ is
called a {\em path hypergraph}
%  (of course, 
% $\H_\cP$ is trivially a path hypergraph)
and $\cP$ is called {\em path representation} of $\H_\cF$. If
isomorphism is $\phi: supp(\H_\cF) \rightarrow supp(\H_\cP)$, then it
is clear that there is an induced path labeling $l_\phi: \cF
\rightarrow \cP$ to the
set system. In the rest of the document, we may use $\cF$ and
$\cH_\cF$ interchangeably to refer the set system and/or its hypergraph.

\noindent
An {\em overlap graph} $\bO(\F)$ of a set system $\cF$ is a graph such
that its vertex set has a bijection to $\cF$ and there exists an edge
between two vertices iff their corresponding sets overlap. Two sets
$A$ and $B$ are said to overlap, denoted by $A \overlap B$, if they
have a non-empty intersection and neither is contained in the other
i.e. $A \overlap B \text{ iff } A \cap B \ne \emptyset, A \nsubseteq B, B
\nsubseteq A$. Thus $\bO(\cF)$ is a subgraph of $\bI(\F)$ and not
necessarily connected. Each connected component of $\bO(\cF)$ is
called an {\em overlap component}. If there are $d$ overlap components
in $\bO(\cF)$, the set subsystems are denoted by $\cO_1, \cO_2, \ldots
\cO_d$. Clearly $\cO_i \subseteq \F, i \in [d]$. For any $i, j \in [d]$,
it can be verified that one of the following is true.
\begin{enumerate}
\item[i.] $supp(\cO_i)$ and $supp(\cO_j)$ are disjoint
\item[ii.] $supp(\cO_i)$ is a subset of a set in $\cO_j$
\item[iii.] $supp(\cO_j)$ is a subset of a set in $\cO_i$
\end{enumerate}
 
\noindent
A path labeling $\cl: \cF \rightarrow \cP$ of setsystem $\cF$ is defined to be {\em
  feasible} if their hypergraphs are isomorphic to each other,
$\cH_\cF \cong \cH_\cP$ and if this isomorphism $\phi: supp(\cF)
\rightarrow supp(\cP)$ induces a path labeling $\cl_\phi: \cF
\rightarrow \cP$ such that $\cl_\phi = \cl$. 

\noindent
For any partial order $(X, \preccurlyeq)$,  the notation $mub(X)$
represents an element $X_m \in X$ which is called
a {\em maximal upper bound} on $X$.  The element $X_m$ is a maximal upper bound of
$X$ if $\nexists X_q \in X$ such that $X_m \preccurlyeq X_q$. 

% \noindent
% When refering to a tree as $T$ it could be a reference to the tree
% itself, or the vertices of the tree. This will be clear from the
% context.\\

\noindent
An {\em in-tree} is a directed rooted tree in which all edges are
directed toward to the root.

\section{Characterization of Feasible Tree Path  Labeling} 
\label{sec:feasible} 

Consider a path labeling $\cl: \cF \rightarrow \cP$ for set system $\cF$
and path system $\cP$ on the given tree $T$. We
call $\cl$ an {\em Intersection Cardinality Preserving Path Labeling
  (ICPPL)} if it has the following properties.
\begin{enumerate}
\item [i.]  $|S| = |\cl(S)|$ for all $S \in \cF$
\item [ii.] $|S_1 \cap S_2| = |\cl(S_1) \cap \cl(S_2)|$ for all
  distinct $S_1, S_2 \in \cF$
\item [iii.] $|S_1 \cap S_2 \cap S_3| = |\cl(S_1) \cap \cl(S_2) \cap
  \cl(S_3)|$ for all distinct  $S_1, S_2, S_3 \in \cF$
\end{enumerate}
The following two lemma are useful in characterizing feasible tree path labelings.  Their
proofs are in the appendix.
\begin{lemma}
  \label{lem:setminuscard}
  If $\cl$ is an ICPPL, and $S_1, S_2, S_3 \in \cF$, then $|S_1 \cap
  (S_2 \setminus S_3)| = |\cl(S_1) \cap (\cl(S_2) \setminus
  \cl(S_3))|$.
\end{lemma}
\begin{lemma}\label{lem:fourpaths} 
  Consider four paths in a tree $P_1, P_2, P_3, P_4$ such that they
  have non-empty pairwise intersections and paths $P_1, P_2$ share a
  leaf. Then there exist distinct integers $i, j, k \in \{1,2,3,4\}$ such
  that, $P_1 \cap P_2 \cap P_3 \cap P_4 = P_i \cap P_j \cap P_k$.
\end{lemma}

\noindent
In the remaining part of this section we describe an algorithmic
characterization for a feasible tree path labeling. We show that a
path labeling is feasible if and only if it is an ICPPL and it
successfully passes the filtering algorithms \ref{perms} and
\ref{leafasgn}.  One direction of this claim is clear: that if a path
labeling is feasible, then all intersection cardinalities are
preserved, i.e. the path labeling is an ICPPL. Algorithm \ref{perms}
has \annote[AS]{no premature exit condition}{it does. verify.} hence any input will go through
it. Algorithm \ref{leafasgn} has an exit condition at line
\ref{xempty}. \annote[AS]{It can be easily verified that $X$ cannot be empty if
$\cl$ is a feasible path labeling.}{Maybe this needs to be expanded in
the appendix}  The reason is
that a feasible path labeling has an associated bijection between
$supp(\cF)$ and $V(T)$ i.e. $supp(\cF^{\cl})$ such that the sets map to paths, ``preserving''
the path labeling. The rest of the section is devoted to
constructively proving that it is sufficient for a path labeling to be
an ICPPL and pass the two filtering algorithms.  To describe in brief, the
constructive approaches refine an ICPPL iteratively, such that at the
end of each iteration we have a ``filtered'' path labeling, and
finally we have a path labeling that defines a family of
bijections from $supp(\cF)$ to $V(T)$ i.e. $supp(\cF^{\cl})$.  \\
\noindent
First we present Algorithm \ref{perms} or Filter 1, and prove its correctness.
This algorithm refines the path labeling by considering pairs of paths
that share a leaf.


\begin{algorithm}[h]
\caption{FILTER 1: Refine ICPPL ($\cF$, $\cl$)}
\label{perms}
\begin{algorithmic}
\STATE Let $\cF_0 = \cF$\\
\STATE Let $\cl_0(S) = \cl(S)$ for all $S \in \cF_0$\\
\STATE $j = 1$\\
\label{shareleaf} \WHILE {there is $S_1, S_2 \in \cF_{j-1}$ such that $\cl_{j-1}(S_1)$ and
  $\cl_{j-1}(S_2)$ have a common leaf in $T$}
\label{setbreak} \STATE $\cF_j = (\cF_{j-1} \setminus \{S_1, S_2\})
\cup \{S_1 \cap S_2, S_1 \setminus S_2, S_2 \setminus S_1 \}$
\COMMENT {Remove $S_1$, $S_2$ and add the ``filtered'' sets}
\STATE for all $S \in \cF_{j-1}$ such that $S \ne S_1$ and $S \ne
S_2$, set $\cl_j(S) = \cl_{j-1}(S)$
\COMMENT {Do not change path labeling for any set other than $S_1$, $S_2$}
\STATE $\cl_j(S_1 \cap S_2) = \cl_{j-1}(S_1) \cap \cl_{j-1}(S_2)$
\STATE $\cl_j(S_1 \setminus S_2) = \cl_{j-1}(S_1) \setminus \cl_{j-1}(S_2)$
\STATE $\cl_j(S_2 \setminus S_1) = \cl_{j-1}(S_2) \setminus \cl_{j-1}(S_1)$

\label{ln:3waycheck} \IF{$(\cF_j, \cl_j)$ does not satisfy condition (iii) of ICPPL}
  \STATE {\bf exit}
\ENDIF

\STATE $j = j+1$\\
\ENDWHILE
\STATE $\cF' = \cF_j$, $\cl' = \cl_j$\\
\STATE {\bf return} $\cF', \cl'$\\
\end{algorithmic}
\end{algorithm}


\begin{lemma} 
  \label{lem:feasible} In Algorithm \ref{perms}, if input $(\cF, \cl)$ is a
  feasible path assignment then at the end of $j$th iteration of the
  {\bf while} loop, $j \ge 0$, 
  $(\cF_j, \cl_j)$ is a feasible path assignment.
\end{lemma}
\begin{proof}
  We will prove this by mathematical induction on the number of
  iterations. The base case $(\cF_0, \cl_0)$ is feasible since it is
  the input itself. Assume the lemma is true till $j-1$th
  iteration. i.e. there is a bijection $\phi: supp(\cF_{j-1})
  \rightarrow V(T)$ such that the induced path labeling on $\cF_{j-1}$,
   $\cl_{\phi[{\cF_{j-1}}]}$ is equal to $\cl_{j-1}$. We will prove that $\phi$ is
  also the bijection that makes $(\cF_j, \cl_j)$ feasible. Note that
  $supp(\cF_{j-1}) = supp(\cF_{j})$ since the new sets in $\cF_j$ are
  created from basic set operations to the sets in $\cF_{j-1}$. For
  the same reason and
  $\phi$ being a bijection, it is clear that $ \cl_{\phi[{\cF_{j}}]}(S_1
  \setminus S_2) = \cl_{\phi[{\cF_{j-1}}]}(S_1) \setminus
  \cl_{\phi[{\cF_{j-1}}]}(S_2)$. Now observe that $ \cl_j(S_1
  \setminus S_2) = \cl_{j-1}(S_1) \setminus \cl_{j-1}(S_2) =
  \cl_{\phi[{\cF_{j-1}}]}(S_1) \setminus
  \cl_{\phi[{\cF_{j-1}}]}(S_2)$. Thus the induced path labeling $\cl_{\phi[{\cF_{j}}]} =
  \cl_{j}$. Therefore lemma is proven. 
\qed
\end{proof}

\begin{lemma} 
  \label{lem:invar1} In Algorithm \ref{perms}, at the end of $j$th
  iteration, $j \ge 0$, of the {\bf while} loop of Algorithm
  \ref{perms}, the following invariants are maintained.
\begin{itemize}
\item []\underline {Invariant I}: $\cl_j(R)$ is a path in $T$ for each $R \in \cF_j$
\item []\underline {Invariant II}: $|R| = |\cl_j(R)|$ for each $R \in \cF_j$
\item []\underline {Invariant III}: For any two $R, R' \in \cF_j$,
  $|R \cap R'| = |\cl_j(R) \cap \cl_j(R')|$
\item []\underline {Invariant IV}: For any three, $R, R', R'' \in \cF_j$,
  $|R \cap R' \cap R''|=|\cl_j(R) \cap \cl_j(R') \cap \cl_j(R'')|$
\end{itemize}
\end{lemma}
\begin{proof}
The detailed proofs of the some of the cases below are in the appendix.
  Proof is by induction on the number of iterations, $j$. In the rest
  of the proof, the term ``new sets'' will refer to the sets added
  to $\cF_j$ in $j$th iteration in line \ref{setbreak} of Algorithm
  \ref{perms}, $\{S_1 \cap S_2, S_1 \setminus S_2, S_2 \setminus S_1
  \}$ and its images in $\cl_j$ where $\cl_{j-1}(S_1)$ and
  $\cl_{j-1}(S_2)$
  intersect and share a leaf.\\
  \noindent
  The base case, $\cl_0$ is an ICPPL on $\cF_0$, since it is the
  input.  Assume the lemma is true till the $j-1$ iteration. Let us consider
the possible cases for each invariant for the  $j$th  iteration.

  \noindent
  \begin{enumerate} % dummy list to get the indendation right.
  \item []
  \begin{enumerate}
  \item [Case 1:] {\em Invariant I and II} 
    \begin{enumerate}
    \item [Case 1.1:] {\em $R$ is not a new set.} If $R$ is in
      $\cF_{j-1}$, then by induction hypothesis this case is trivially
      proven.
    \item [Case 1.2:] {\em $R$ is a new set.} If $R$ is in $\cF_{j}$
      and not in $\cF_{j-1}$, then it must be one of the new sets
      added in $\cF_j$. In this case, it is clear that for each new
      set, the image under $\cl_j$ is a path since by definition the
      chosen sets $S_1$, $S_2$ are from $\cF_{j-1}$ and due to the
      while loop condition, $\cl_{j-1}(S_1)$, $\cl_{j-1}(S_2)$ have a
      common leaf. Thus invariant I is proven.\\
      Moreover, due to induction hypothesis of invariant III ($j-1$th
      iteration) and the definition of $l_j$ in terms of $l_{j-1}$,
      invariant II is indeed true in the $j$th iteration for any of
      the new sets.
   \end{enumerate}
  \item [Case 2:] {\em Invariant III}
    \begin{enumerate}
    \item [Case 2.1:] {\em $R$ and $R'$ are not new sets.} Trivially
      true by induction hypothesis.
    \item [Case 2.2:] {\em Only one, say $R$, is a new set.} Due to
  invariant IV induction hypothesis, lemma \ref{lem:setminuscard} and
      definition of $l_j$, it follows that invariant III is true no matter which of
      the new sets $R$ is equal to. It is important to note that $R'$ is not a new set here.
    \item [Case 2.3:] {\em $R$ and $R'$ are new sets.} By definition,
      the new sets and their path images in path label $l_j$ are
      disjoint so $|R \cap R'| = |l_j(R) \cap l_j(R)| = 0$. Thus case
      proven.
    \end{enumerate}
  \item [Case 3:] {\em Invariant IV}
    Due to the condition in line \ref{ln:3waycheck}, invariant IV is
    ensured at the end of every iteration.
%     \begin{enumerate}
%     \item [Case 3.1:] {\em $R$, $R'$ and $R''$ are not new sets.} Trivially
%       true by induction hypothesis.
%     \item [Case 3.2:] {\em Only one, say $R$, is a new set.}
%       If $R = S_1 \cap S_2$,  from lemma \ref{lem:fourpaths} and
%       invariant III hypothesis,  this case is proven. Similarly if $R$
%       is any of the other new  sets, the case is proven by also using
%       lemma  \ref{lem:setminuscard}.
%     \item [Case 3.3:] {\em At least two of $R, R', R''$ are new sets.}
%       The new sets are disjoint hence this case is vacuously true.
%     \end{enumerate}
  \end{enumerate} 
\end{enumerate} 
\qed
\end{proof}

\begin{lemma}
  \label{lem:noexit1} In Algorithm \ref{perms}, consider an ICPPL
  input $(\cF, \cl)$ which is also a feasible path labeling. Then in
  the execution of the algorithm its exit condition in line
  \ref{ln:3waycheck}, i.e. failure of three way intersection preservation,
% in $(\cF_j, \cl_j)$) - will not be true in the $j$th iteration where
%   $j \ge 0$ and 
  will not be true in any iteration of the {\em \bf while} loop and the
  algorithm executes without a premature exit.
\end{lemma}
\begin{proof} 
This proof uses mathematical induction on the number of
iterations $j$, $j \ge 0$, of the {\bf while} loop that executed
without exiting. The base case, $j = 0$ is obviously true since the
input is an ICPPL and the exit condition cannot hold true due to ICPPL
condition (iii).  Assume the algorithm executes till the end of
$j-1$th iteration without exiting at line \ref{ln:3waycheck}. Consider
the $j$th iteration. From lemma \ref{lem:feasible} we know that
$(\cF_j, \cl_j)$ and $(\cF_{j-1}, \cl_{j-1})$ are feasible\remove[AS]{and from the proof in lemma lem:invar1 we know that
  $(\cF_{j-1}, \cl_{j-1})$ satisfies all the invariants defined in the
  lemma}.  Thus
there exists a bijection $\phi: supp(\cF) \rightarrow V(T)$ such that
the induced path
%labeling on $\cF_{j-1}$ $\cl_{\phi[\cF_{j-1}]} = \cl_{j-1}$.
labeling on $\cF_{j}$, $\cl_{\phi[\cF_{j}]}$ and on $\cF_{j-1}$,
$\cl_{\phi[\cF_{j-1}]}$ are equal to $\cl_{j}$ and $\cl_{j-1}$ respectively.
We need to prove that for any $R, R', R'' \in \cF_{j}$, $|R \cap R'
\cap R''| = |\cl_j(R) \cap \cl_j(R') \cap \cl_j(R'')|$. 

\noindent
The following are the possible cases that could arise. From argument
above, $|\cl_j(R) \cap \cl_j(R') \cap \cl_j(R'')| =
|\cl_{\phi[\cF_{j}]}(R) \cap \cl_{\phi[\cF_{j}]} (R') \cap
\cl_{\phi[\cF_{j}]} (R'')|$
  \begin{enumerate} % dummy list to get the indendation right.
  \item []
    \begin{enumerate}
    \item [Case 1:] {\em None of the sets are new. $R, R', R'' \in
        \cF_{j-1}$.}  We know $(\cF_{j-1}, \cl_{j-1})$ is
      feasible. Thus $|R \cap R' \cap R''| = |\cl_{j-1}(R) \cap
      \cl_{j-1}(R') \cap \cl_{j-1}(R'')| = |\cl_{j}(R) \cap
      \cl_{j}(R') \cap \cl_{j}(R'')|$.
    \item [Case 2:]{\em Only one, say $R$, is a new set.}  Let $R =
      S_1 \cap S_2$ ($S_1, S_2$ are defined in the proof of lemma
      \ref{lem:invar1}). Now we have $|R \cap R' \cap R''| = |S_1 \cap
      S_2 \cap R' \cap R''| = |\cl_{j-1}(S_1) \cap \cl_{j-1}(S_2) \cap
      \cl_{j-1}(R') \cap \cl_{j-1}(R'')| = |\cl_{j}(R) \cap
      \cl_{j}(R') \cap \cl_{j}(R'')|$. Thus proven. If $R$ is any of
      the other new sets, the same claim can be verified using lemma
      \ref{lem:setminuscard}.
    %\item []{\bf Case 3:} 
    \item [Case 3:] {\em At least two of $R, R', R''$ are new sets.}
      The new sets are disjoint hence this case is vacuously true.
    \end{enumerate}
  \end{enumerate}
\qed
\end{proof}

\noindent
As a result of Algorithm \ref{perms} each leaf $v$ in $T$ is such that
there is exactly one set in $\cF$ such that $v$ is a node in the path
assigned to it.  In Algorithm \ref{leafasgn} we identify elements in
$supp(\cF)$ whose images are leaves in a feasible path labeling if one
exists.  Let vertex $v \in T$ be the unique leaf incident on a path
image $P$ in $\cl$.  We define a new path labeling $\cl_{new}$ such
that $\cl_{new}(\{x\}) = \{v\}$ where $x$ an arbitrary element from
$\cl^{-1}(P) \setminus \bigcup_{\hP \ne P} \cl^{-1}(\hP)$. In other
words, $x$ is an element present in no other set in $\cF$ except
$\cl^{-1}(P)$. This is intuitive since $v$ is present in no other path
image other than $P$.  The element $x$ and leaf $v$ are then removed
from the set $\cl^{-1}(P)$ and path $P$ respectively. The tree is
pruned off $v$ and the refined set system will have $\cl^{-1}(P)
\setminus \{x\}$ instead of $\cl^{-1}(P)$. After doing this for all
leaves in $T$, all path images in the new path labeling $\cl_{new}$
except single leaf labels (the pruned out vertex is called the {\em
  leaf label} for the corresponding set item) are paths from the
pruned tree $T_0 = T \setminus \{v \mid v \text{ is a leaf in }
T\}$. Algorithm \ref{leafasgn} is now presented with details.


\begin{algorithm}[h]
\caption{Leaf labeling from an ICPPL $(\cF,\cl)$}
\label{leafasgn}
\begin{algorithmic}
\STATE Let $\cF_0 = \cF$\\
\STATE Let $\cl_0(S) = \cl(S)$ for all $S \in \cF_0$. Note: Path images are such that no
two path images share a leaf.\\
\STATE $j = 1$\\
\label{uniqueleaf}\WHILE {there is a leaf $v$ in $T$ and a unique $S_1 \in \cF_{j-1}$ such that
  $v \in \cl_{j-1}(S_1)$ }
\STATE $\cF_j = \cF_{j-1} \setminus \{S_1\}$\\
\STATE for all $S \in \cF_{j-1}$ such that $S \ne S_1$ set $\cl_j(S) =
\cl_{j-1}(S)$\\
\STATE $X = S_1 \setminus \bigcup_{S \in \cF_{j-1}, S \ne S_1}S$\\
\label{xempty}\IF{$X$ is empty} 
\STATE {\bf exit}
\ENDIF
\STATE Let $x = $ arbitrary element from $X$\\
\STATE $\cF_j = \cF_j \cup \{\{x\}, S_1 \setminus \{x\}\} $\\
\STATE $\cl_j(\{x\}) = \{v\}$\\
\STATE $\cl_j(S_1 \setminus \{x\}) = \cl_{j-1}(S_1) \setminus \{v\}$\\
\STATE $j = j+1$\\
\ENDWHILE
\STATE $\cF' = \cF_j$\\
\STATE $\cl' = \cl_j$\\
\STATE {\bf return} $\cF', \cl'$\\
\end{algorithmic}
\end{algorithm}

\begin{lemma}
\label{lem:invar3}
In Algorithm \ref{leafasgn}, for all $j \geq 0$, at the end of the
$j$th iteration the four invariants given in lemma \ref{lem:invar1}
are valid.  
\end{lemma}
\begin{proof}
  Suppose the input ICPPL is also feasible but $X$ is empty. This will
  prematurely exit the algorithm and thus prevent us from finding the
  permutation.  We will now show that this cannot happen. We know that
  $v$ is an element of $\cl_{j-1}(S_1)$. Since it is uniquely present
  in $\cl_{j-1}(S_1)$, it is clear that $v \in \cl_{j-1}(S_1)
  \setminus \bigcup_{S \in \cF_{j-1}, S \ne S_1}\cl_{j-1}(S)$.  Note
  that for any $x \in S_1$ it is contained in at least two sets due to
  our assumption about cardinality of $X$. Let $S_2 \in \cF_{j-1}$ be
  another set that contains $x$. From the above argument, we know $v
  \notin \cl_{j-1}(S_2)$. Therefore there cannot exist a permutation
  that maps elements of $S_2$ to $\cl_{j-1}(S_2)$. This contradicts
  our assumption that the input is feasible. Thus $X$ cannot be empty
  if input is ICPPL and feasible.
  For the rest of the proof we use mathematical induction on the
  number of iterations $j$. As before, the term ``new sets'' will
  refer to the sets added in $\cF_j$ in the $j$th iteration, i.e. $S_1
  \setminus \{x\}$ and $\{x\}$ as defined in line \ref{uniqueleaf}.\\
  For $\cF_0, \cl_0$ all invariants hold because it is output from
  algorithm \ref{perms} which is an ICPPL. Hence base case is proved.
  Assume the lemma holds for the $j-1$th iteration. Consider $j$th
  iteration. 
  \noindent
  \begin{enumerate} % dummy list to get the indendation right.
  \item []
  \begin{enumerate}
  \item [Case 1:] {\em Invariant I and II} 
    \begin{enumerate}
    \item [Case 1.1:] {\em $R$ is not a new set.} If $R$ is in
      $\cF_{j-1}$, then by induction hypothesis this case is trivially
      proven.
    \item [Case 1.2:] {\em $R$ is a new set.} If $R$ is in $\cF_{j}$
      and not in $\cF_{j-1}$, then it must be one of the new sets
      added in $\cF_j$. Removing a leaf $v$ from path $\cl_{j-1}(S_1)$
      results in another path. Moreover, $\{v\}$ is trivially a
      path. Hence regardless of which new set $R$ is, by definition of
      $\cl_j$, $\cl_{j}(R)$ is a path. Thus invariant I is proven.\\
      We know $|S_1| = |\cl_{j-1}(S_1)|$, due to induction
      hypothesis. Therefore $|S_1 \setminus \{x\}| = |\cl_{j-1}(S_1)
      \setminus \{v\}|$. This is because $x \in S_1$ iff $v \in
      \cl_{j-1}(S_1)$. If $R = \{x\}$, invariant II is trivially
      true. Thus invariant II is proven.
  \end{enumerate}
  \item [Case 2:] {\em Invariant III}
    \begin{enumerate}
    \item [Case 2.1:] {\em $R$ and $R'$ are not new sets.} Trivially
      true by induction hypothesis.
    \item [Case 2.2:] {\em Only one, say $R$, is a new set.}  By
      definition, $\cl_{j-1}(S_1)$ is the only path with $v$ and $S_1$
      the only set with $x$ in the previous iteration, hence $|R' \cap
      (S_1 \setminus \{x\})| = |R' \cap S_1|$ and $|\cl_{j-1}(R') \cap
      (\cl_{j-1}(S_1) \setminus \{v\})| = |\cl_{j-1}(R') \cap
      \cl_{j-1}(S_1)|$ and $|R' \cap \{x\}| = 0$, $|\cl_{j-1}(R') \cap
      \{v\}| = 0$. Thus case proven.
    \item [Case 2.3:] {\em $R$ and $R'$ are new sets.} By definition,
      the new sets and their path images in path label $l_j$ are
      disjoint so $|R \cap R'| = |l_j(R) \cap l_j(R)| = 0$. Thus case
      proven.
    \end{enumerate}
  \item [Case 3:] {\em Invariant IV}
    \begin{enumerate}
    \item [Case 3.1:] {\em $R$, $R'$ and $R''$ are not new sets.} Trivially
      true by induction hypothesis.
    \item [Case 3.2:] {\em Only one, say $R$, is a new set.}  By the
      same argument used to prove invariant III, $|R' \cap R'' \cap
      (S_1 \setminus \{x\})| = |R' \cap R'' \cap S_1|$ and
      $|\cl_{j-1}(R') \cap \cl_{j-1}(R'') \cap (\cl_{j-1}(S_1)
      \setminus \{v\})| = |\cl_{j-1}(R') \cap \cl_{j-1}(R'') \cap
      \cl_{j-1}(S_1)|$. Since $R', R'', S_1$ are all in $\cF_{j-1}$,
      by induction hypothesis of invariant IV, $|R' \cap R'' \cap S_1|
      = |\cl_{j-1}(R') \cap \cl_{j-1}(R'') \cap \cl_{j-1}(S_1)|$.
      Also, $|R' \cap R'' \cap \{x\}| = |\cl_{j-1}(R') \cap
      \cl_{j-1}(R'') \cap \{v\}|$ = 0.
    \item [Case 3.3:] {\em At least two of $R, R', R''$ are new sets.}
      If two or more of them are not in $\cF_{j-1}$, then it can be
      verified that $|R \cap R' \cap R''| = |\cl_j(R) \cap \cl_j(R')
      \cap \cl_j(R'')|$ since the new sets in $\cF_j$ are
      disjoint \annote[AS]{}{The following is not
        correct. While loop only handles one leaf at a
        time.} \remove[AS]{ or as follows: assuming $R, R' \notin
        \cF_{j-1}$ and new sets are derived from $S_1, S_2 \in
        \cF_{j-1}$ with $x_1, x_2$ exclusively in $S_1, S_2$,
        $\cl_j(\{x_1\})=\{v_1\}, \cl_j(\{x_2\}) = \{v_2\}$ where $
        \{x_1\}, \{x_2\} \in \cF_j $ thus $v_1, v_2$ are exclusively
        in $\cl_j(\{x_1\})$, $\cl_j(\{x_2\})$ respectively. It follows
        that $|R \cap R' \cap R''| =$ $ |(S_1 \setminus \{x_1\}) \cap
        (S_2 \setminus \{x_2\}) \cap R''| =$ $ |S_1 \cap S_2 \cap R''|
        = $ $|\cl_{j-1}(S_1) \cap \cl_{j-1}(S_2) \cap \cl_{j-1}(R'')|
        = |(\cl_{j-1}(S_1) \setminus \{v_1\}) \cap (\cl_{j-1}(S_2)
        \setminus \{v_2\}) \cap \cl_{j-1}(R'')| = |\cl_j(R) \cap
        \cl_j(R') \cap \cl_j(R'')|$}. Thus invariant IV is also
      proven.
    \end{enumerate}
  \end{enumerate} 
  \end{enumerate} 
\qed
\end{proof}


\noindent
We have seen two filtering algorithms above - algorithms \ref{perms}
and \ref{leafasgn}. We also proved that if the input is indeed
feasible, these algorithms do not exit prematurely and succesfully
filters the input keeping the ICPPL conditions intact.
Using these algorithms we now prove the following theorem.

\begin{theorem}
\label{th:perm}
  If $\cF$ has an ICPPL $\cl$ to a tree $T$, then there exists a hypergraph
  isomorphism $\phi : supp(\cF) \rightarrow supp(\cF^\cl)$ such that
  the $\phi$-induced tree path labeling is equal to $\cl$, $\cl_\phi = \cl$.
\end{theorem}
\begin{proof} 
This is a contructive proof. We find $\phi$ part by part by
running algorithms \ref{perms} and \ref{leafasgn} one after the other
in a loop. After each iteration we calculate an exclusive subset of
the bijection $\phi$, namely that which involves all the leaves of the
tree in that iteration. Then all the leaves are pruned off the tree
before the next iteration. The loop terminates when the pruned tree
becomes a single path after which ICPIA algorithm is used to find the
final subset (interval assignment) that exhausts $\phi$. This is
the brief outline of the algorithm and now we describe it in detail
below.
%The algorithm is presented as follows.
% \begin{algorithm}[h]
% \caption{find the tree path bijection of ICPPL ($\cF$, $\cl$) on given
%   tree $T$}
% \label{alg:treepathbij}
% \begin{algorithmic}
% \STATE $(\cF_0, \cl_0) = (\cF, \cl)$
% \WHILE {$T_i$ is not a path}
%   \STATE $(\cF_i', \cl_i') = $ Refine ICPPL $(\cF_i, \cl_i)$ by
%   calling algorithm \ref{perms}
%   \STATE $(\cF_i'', \cl_i'') = $ Get leaf assignment $(\cF_i', \cl_i')$
%   by calling algorithm \ref{leafassign}
%   \STATE 
  
% \ENDWHILE

% \end{algorithmic}
% \end{algorithm}
\noindent
First, the given ICPPL $(\cF, \cl)$ and tree $T$ are given as input to
Algorithm \ref{perms}. This yields a ``filtered'' ICPPL as the output
which is input to Algorithm \ref{leafasgn}.  Let the output of
Algorithm \ref{leafasgn} be $(\cF',\cl')$. We define a bijection
$\phi_1: Y_1 \rightarrow L_1$ where $Y_1 \subseteq supp(\cF)$ and $V_1
= \{v \mid v \text{ is a leaf in } $T$\}$.  It can be observed that
the output of Algorithm \ref{leafasgn} is a set of path assignments to
sets and one-to-one assignment of elements of $U$ to each leaf of
$T$. These are defined below as $\cl_1$ and $\phi_1$ respectively.
\vspace{-3mm}
\begin{align*}
  \cl_1(S) &= \cl'(S) &\text{ when $\cl'(S)$ has non leaf vertices} \\
  \phi_1(x) &= v  &\text{ when $\cl'(S)$ has only a leaf, and } v \in \cl'(S), x \in S
\end{align*}
Consider the tree $T_1$ which is isomorphic to $T[V(T) \setminus
V_0]$, i.e. it is $T$ with all its leaves removed. Let $U_1$ be the
universe of the subsystem that is not mapped to a leaf, i.e. $U_1 =
supp(\cF) \setminus \{ x \mid x = \cl_1^{-1}(v), v \in V_1\}$ .

% To be precise, it would be of the form $\cB_0 =
% \cA_0 \cup \cL_0$. The leaf assignments are defined in $\cL_0
% = \{ (x_i,v_i) \mid x_i \in U, v_i \in T, x_i \ne x_j, v_i \ne v_j, i
% \ne j, i,j \in [k] \}$ where $k$ is the 
% number of leaves in $T$. The path assignments are defined in $\cA_0
% \subseteq \{(S_i',P_i') \mid S_i' \subseteq U_0, P_i' \text{ is a path
%   from } T_0\}$ 
% where $T_0$ is the tree obtained by removing all the
% leaves in $T$ and
% $U_0 = U \setminus \{ x \mid x \text{ is assigned to
%   a leaf in }\cL_0 \}$. 
\noindent
Let $\cF_1$ be the set system induced by $\cF'$ on universe $U_1$.
Clearly, now we have a subproblem of finding the
hypergraph isomorphism for $(\cF_1, \cl_1)$ with tree $T_1$.
% for the path assignment $\cA_0$ which has paths from tree
% $T_0$ and sets from universe $U_0$. Now we repeat the procedure and
% the path assignment $\cA_0$ and tree $T_0$
\noindent
Now we repeat Algorithm \ref{perms} and Algorithm \ref{leafasgn} on
$(\cF_1, \cl_1)$ with tree $T_1$. As before we define $l_2$ in terms
of $l_1$, $\phi_2$ in terms of leaves of $T_1$, prune the tree $T_1$
to get $T_2$ and so on.
Thus in the $i$th iteration, $T_i$ is the pruned tree, $\cl_i$ is a
feasible path labeling to $\cF_i$ if $(\cF_{i-1}, \cl_{i-1})$ is
feasible, $\phi_i$ is the leaf labeling of leaves of
$T_{i-1}$. Continue this until some $d$th iteration for the smallest
value $d$ such that $T_d$ is
a path. From
the lemma \ref{lem:invar1} and \ref{lem:invar3} we know that $(\cF_d, \cl_d)$  is an
ICPPL. We also know that the special case of ICPPL when the tree is a
path is the interval assignment (ICPIA) problem. 
We now
run the ICPIA algorithms \cite{nsnrs09} on $(\cF_d, \cl_d)$.

\noindent
It is true that $T_d$ is not  precisely an interval in the sense of consecutive integers
because they could be arbitrarily named nodes a tree. However, it is easy to see that
the nodes of $T_{d}$ can be ordered from left to right and ranked to get
intervals $I_i$ for every path $S_i \in \cF_d$ as follows. $I_i = \{[l,r]
\mid l = \text{ the lowest rank of the nodes in }\cl_d(S_i) \text{, } r = l+|\cl_d(S_i)|-1
\}$, where $S_i \in \cF_d$. We define an interval assignment $\cA = \{ (S_i, I_i) \mid S_i
\in \cF_d\}$ which is an ICPIA and also in the format ICPIA algorithm
requires. The ICPIA algorithms give us $\cA$ and the bijection $\phi_{d+1} : U_d \rightarrow T_d$. 
The bijection $\phi: U \rightarrow V(T)$ defined as follows is the bijection for the
ICPPL.
\begin{align*}
  \phi(x) &= \phi_i(x) \text{ where $x$ is in the domain of $\phi_i$},
  i \in [d+1]
\end{align*}
It can be verified that $\phi$ is a bijection on $supp(\cF)$ into
$V(T)$ which is the path hypergraph isomorphism between $\cF$ and
$\cF^\cl$ such that $\cl_\phi = \cl$. Thus the theorem is proven.
\qed
\end{proof}

% is given as input to Algorithm \ref{perms}. The output of this
% algorithm is given to Algorithm \ref{leafasgn} to get a new
% union of path and leaf assignments $\cB_1 =
% \cA_1 \cup \cL_1$ defined similar to $\cB_0, \cL_0, \cA_0$. In
% general, the two algorithms are run on
% path assignment $\cA_{i-1}$ with paths from tree $T_{i-1}$ to get a new
% subproblem with path assignment $\cA_i$ and tree $T_{i}$. $T_i$ is
% the subtree of $T_{i-1}$ obtained by removing all its leaves. More importantly, it gives leaf
% assignments $\cL_{i}$ to the leaves in tree $T_{i-1}$. This is
% continued until we get a subproblem with path assignment $\cA_{d-1}$ and
% tree $T_{d-1}$ for some $d \le n$ which is just a
% path. From the last lemma we know that $\cA_{d-1}$ is an
% ICPPA. Another observation is that an ICPPA with all its tree paths
% being intervals (subpaths from a path) is nothing but an ICPIA \cite{nsnrs09}.
% Let $\cA_{d-1}$ be equal to $\{(S_i'',P_i'') \mid S_i'' \subseteq U_{d-1}, P_i'' \text{ is a path
%   from } T_{d-1} \}$. 
% It is true that the paths $P_i''$s
% may not be precisely an interval in the sense of consecutive integers
% because they are some nodes from a tree. However, it is easy to see that
% the nodes of $T_{d-1}$ can be ordered from left to right and ranked to get
% intervals $I_i$ for every path $P_i''$ as follows. $I_i = \{[l,r]
% \mid l = \text{ the lowest rank of the nodes in }P_i'', r = l+|P_i''|-1
% \}$. Let asssignment $\cA_d$ be with the renamed paths. $\cA_d = \{ (S_i'', I_i) \mid (S_i'', P_i'') \in \cA_{d-1}
% \}$. What has been effectively done is renaming the nodes in $T_{d-1}$
% to get a tree $T_d$.
% The ICPIA $\cA_d$ is now in the format that the ICPIA algorithm
% requires which gives us the permutation $\sigma' : U_{d-1} \rightarrow T_{d-1}$

% \noindent
% $\sigma'$ along with all the leaf assignments $\cL_i$
% gives us the permutation for the original path assignment $\cA$.
% More precisely, the permutation for tree path assignment $\cA$ is defined as
% follows. $\sigma: U \rightarrow T$ such that the following
% is maintained.
% \begin{align*}
%  \sigma(x) &= \sigma'(x),   \text{ if } x \in U_{d-1} \\
%            &= \cL_i(x),     \text{ where $x$ is assigned to a leaf in a
%              subproblem $\cA_{i-1}, T_{i-1}$}
% \end{align*}

% \noindent
% To summarize, run algorithm \ref{perms} and
% \ref{leafasgn} on $T$. After the leaves have been assigned to specific
% elements from $U$, remove all leaves from $T$ to get new tree
% $T_0$. The leaf assignments are in $\cL_0$. Since only leaves were removed $T_0$ is indeed a tree. Repeat
% the algorithms on $T_0$ to get leaf assignments $\cL_{1}$. Remove the
% leaves in $T_0$ to get $T_1$ and so on until the pruned tree $T_d$
% is a single path. Now run ICPIA algorithm on $T_d$ to get
% permutation $\sigma'$. The relation $\cL_0 \cup \cL_1 \cup .. \cup
% \cL_{d} \cup \sigma'$ gives the bijection required in the original problem.\qed
%\end{proof}


\section{Finding tree path labeling from $k$-subdivided stars}
\label{sec:ksubdivstar}
As we saw earlier, the algorithm \ref{Al:icppa-main} for the problem
of tree path labeling to a path system in a general tree is not
polynomial time. Algorithm \ref{Al:icppa-main} line
\ref{l:icppasubtree} leaves an unsolved problem in the main ICPPL
algorithm where ICPPL needs to be found out for the mub of each
partition $X_i$ i.e, $X_{i0}$ on subtree $T_i$. Essentially this is
the problem of finding a path labeling to an overlap component of $\cH_\cF$ from a
subtree of $T$.
When the subtrees are restricted to a smaller class, namely 
$k$-subdivided stars, we have an algorithm which has better time complexity.

\noindent
Following the notation in the previous section, the subtree assigned to the partition
$X_i$ is $T_i$. We saw that it is sufficient to find the ICPPL for $X_{i0}$
from $T_i$ to find the ICPPL for the set subsystems corresponding to
the whole partition $X_i$. Hence in this section, we are interested in
the mub $X_{i0}$ of partition $X_i$.  Let the set subsystem
corresponding to $X_{i0}$ be $\cO_{i0}$. For ease of notation and due to our focus here being only on the overlap subsystem
of the mub and the assigned subtree, we will drop the subscripts, and
call $\cO$ and $T$ as the set system and tree (rather than set
subsystem and subtree) respectively.

\noindent
Note that here we assume the partitioning of the tree $T$
into subtrees $\{T_i \mid T_i \text{ assigned to } X_i,$ $T_i \subseteq T,$  $i \in [t]\}$ has been
done. The problem of partitioning $T$ is a problem that needs to be
addressed separately and is not covered in this paper at the moment.


\noindent
We generalize the interval assignment algorithm for an overlap
component from a prime matrix in \cite{nsnrs09} (algorithm 4 in their paper) to find
tree path labeling for overlap component $\cO$. The tree $T$ is a
$k$-subdivided star. The vertex $r$ is the center of the star.

\noindent
The outline of the algorithm is as follows. Notice that the path
between a leaf and the center vertex has the property that none of the
vertices except the center has degree greater than 2. Thus each ray excluding the center
can be considered as independent intervals. 
 So we begin by labeling of hyperedges to paths that have vertices
 from a single ray only and the center vertex. Clearly this can be
 done using ICPIA alone. This is done for each ray one
after another till a condition for a blocking hyperedge is reached
for each ray which is described below. This part of the algorithm is called the
{\em initialization of path labeling}. 

\noindent
When considering labeling from any
particular ray, we will reach a point in the algorithm were we cannot
proceed further with ICPIA alone because the overlap properties of the
hyperedge will require a path that will cross the center of the star
to another ray and ICPIA cannot tell us which ray that would be. Such
a hyperedge is called {\em blocking hyperedge} of that ray. At this
point we make the following observation about the classification of
the hyperedges in $\cO$.
\begin{itemize}
\item[i] {\em Type 0/ labeled hyperedges}: The hyperedges that have been labeled.
\item[ii] {\em Type 1/ unlabeled non-overlapping hyperedges}: The hyperedges that are either contained or
  disjoint from type 0 hyperedges.
\item[iii] {\em Type 2/ unlabeled overlaping hyperedges}: The hyperedges that overlap with at least one
  labeled hyperedge, say $H$, but cannot be labeled to a path in the
  same ray as $\cl(H)$ alone. It requires verices from another ray
  also in its labeling. A {\em blocking hyperedge} is one of this kind
  which is encountered in each iteration of the initialization of rays algorithm.
\end{itemize}
\noindent
Since $\cO$ is an overlap component, the type 1 hyperedges overlap
with some type 2 hyperedge and can be handled after type 2
hyperedges. Note that in the algorithm outlined above, we find a single
blocking hyperedge and it is a type 2 hyperedge, per ray. Consider a
ray $R_i = \{v \mid v \in V(T), v \text{ is in $i$th ray or is the center}\}$ and its
corresponding blocking hyperedge $B_i$. Now we try
to make a partial path labeling such that for every $i \in [l]$. We partition the blocking
hyperedge into two subsets $B_i = B'_i \cup B''_i$ such that $B'_i,
B''_i$ map to paths $P'_i, P''_i$ respectively which are defined as
follows. 
\begin{table}[h]
  \centering
  \begin{tabular}[h]{ll}
% \begin{align*}
%     P'_i &\subseteq R_i \text{ such that } r \in P'_i \\
%     &\text{ and } |P'_i| = k+2- |supp(\{P \mid P \text{ is a path from $R_i$
%       assigned to type 0 hyperedges} \})|\\
%     P''_i  &\in \{P_{i,j} \mid j \in [l], j \ne i\} \\
%     &\text{where } P_{i,j} = \{v_{j,p} \mid v_{j,0} \text{ is
%       adjacent to $r$ on $R_j$, }\\
%     \text{ for all } 0 < p |B_i \setminus P'_i|-1, v_{j,p} \text{ is
%       adjacent to } v_{j,p-1}\} \\  
%     P'_i \cup P''_i & \text{ is a path in $T$} 
% \end{align*}
    $P'_i$ &$\subseteq R_i \text{ such that } $$r \in P'_i $\\
    &and $|P'_i| = k + 2 - |supp(\{ P \mid P \text{ is a path from } R_i
      \text{ assigned to type 0 hyperedges} \})|$\\
    $P''_i$ &$\in \{P_{i,j} \mid j \in [l], j \ne i\}$ \\
    &$\text{where } P_{i,j} = \{v_{j,p} \mid v_{j,0} \text{ is
      adjacent to $r$ on $R_j$, }$\\
    &for all $ 0 < p \le |B_i \setminus P'_i|-1, v_{j,p} \text{ is
      adjacent to } v_{j,p-1}\}$ \\  
    $P'_i \cup P''_i$ & $\text{ is a path in $T$}$ 
  \end{tabular}
\end{table}

\noindent
The path $P'_i$ is obvious and the following procedure is used to find
$P''_i$. It is clear that a hyperedge cannot be blocking more than two rays,
since a path cannot have vertices from more than two rays. 

\begin{observation}
\label{obs:sameblock}
If the blocking hyperedge $B_a$ of ray $R_a$ is also the blocking
  hyperedge for another ray $R_b$ (i.e. $B_a = B_b$), then clearly
  $P''_a = P_{a,b}$ (and $P''_b = P_{b,a}$). 
\end{observation}

\begin{observation}
\label{obs:diffblock}
If $B_a$ does not block
  any other rays of the star other than $R_a$, then we find that it
  must intersect with exactly one other blocking hyperedge, say
  $B_b$. Once we find the second 
ray,  then clearly $P''_a = P_{a,b}$.
\end{observation}
\noindent
Note that $P''_b \ne P_{b,a}$ in
observation \ref{obs:diffblock} else it would have been covered in
observation \ref{obs:sameblock}.
Now we continue to find new blocking hyperedges on all rays until the
path labeling is complete. 

\noindent
The algorithm is formally described as follows. Algorithm
\ref{al:icppl-kleaves_symm_starlike__4} is the main algorithm which
uses algorithms \ref{al:icppl-initialize_rays_symm_starlike},
\ref{al:icppl-saturate_rays_symm_starlike} and
\ref{al:icppl-partial_labeling_symm_starlike} as subroutines. The
function $dist(u,v)$ returns the number of vertices between the
vertices $u$ and $v$ on the path that connects them (including $u$ and $v$).


\begin{algorithm}[h]
\caption{Algorithm (main subroutine) to find an ICPPL $\cl$ for an overlap
  component $\cO$ from $k$-subdivided star graph $T$:
  $overlap\_ICPPL\_l\_leaves\_symstarlike3(\cO, T$)} 
\label{al:icppl-kleaves_symm_starlike__4}
%{\tiny
\begin{algorithmic}[1]

\STATE $\cL$  
\COMMENT {$\cL \subseteq \cO$ is a global variable for the set subsystem that has a
  path labeling so far. It is the domain of the feasible path labeling $\cl$ at
  any point in the algorithm.}

\STATE $\cl$ 
\COMMENT {$\cl: \cL \rightarrow \cP$, is a global
  variable representing a feasible path
  labeling of $\cL$ to some path system $\cP$ of $T$. It is
  the partial feasible path labeling of $\cO$ at any point in the algorithm.}

%\STATE $i \leftarrow initialize\_rays(\cO, T)$ 
\STATE $initialize\_rays(\cO, T)$ 
\COMMENT {Call algorithm
  \ref{al:icppl-initialize_rays_symm_starlike} for initialization of rays.
  This is when a hyperedge is assigned to a path with the ray's
  leaf.} % $i$ rays are initialized.}

\WHILE {$\cL \ne \cO$}
  \STATE $saturate\_rays\_and\_find\_blocking\_hyperedges(\cO, T)$
  \COMMENT {Saturate all rays of $T$ by using algorithm 
  \ref{al:icppl-saturate_rays_symm_starlike}. This subroutine also
  finds the blocking hyperedge $\cB_i$ of each ray $i$. A blocking
  hyperedge is one that needs to be labeled to a path that has
  vertices from exactly two rays.}

  \STATE $partial\_path\_labeling\_of\_blocking\_hyperedges(\cO, T)$
  \COMMENT {Find path labeling of blocking hyperedges by using
    algorithm \ref{al:icppl-partial_labeling_symm_starlike}. This subroutine
  finds the part of the blocked hyperedge's path label that comes from the second ray.}

\ENDWHILE % \cL \ne \cO

\end{algorithmic} %}
\end{algorithm}


\begin{algorithm}[h]
\caption{$initialize\_rays(\cO, T)$}
\label{al:icppl-initialize_rays_symm_starlike}
%{\tiny
\begin{algorithmic}[1]

\STATE Let $\{ v_i
\mid i \in [l], l \text{ is number of leaves of $T$} \}$ 
\COMMENT {Also note $k+2$ is the length of the path from the center to
  any leaf since $T$ is $k$-subdivided star.}

\STATE $\cK  \leftarrow \{ H \mid H \in \cO$, $N(H)$ in $\cO$ is a
clique $\}$ \COMMENT {Local variable to hold the marginal
  hyperedges. A marginal hyperedge is one that has exactly one
  inclusion chain of interections with every set it overlaps with,
  i.e., its neighbours in the overlap graph form a clique.}

% \STATE $\cK  \leftarrow \{ H \mid H \in \cO$ s.t. neighbours of $H$ in
% the overlap graph form a clique$\}$ 
\FOR {every inclusion chain $C \subseteq \cK$ }
  \STATE Remove from $\cK$ all sets in $C$ except the set
  $H_{C-icpia-max}$ which is the set closest to the maximal inclusion set $H_{C-max}$
  such that $|H_{C-icpia-max}| \le k+2$. 
\ENDFOR % every inclusion chain $C \subseteq \cK$ 

\IF {$|\cK| > l$}
  \STATE Exit. %%%%%%%%%%******** NEEDS PROOF
  \COMMENT {No labeling possible since $\cO$ is an overlap component and
  $T$ does not have enough rays.}
\ENDIF

%\IF {$|\cK| < l$}
\label{line:Li_is_empty}  
\STATE \COMMENT { $H_{C-icpia-max}$ does not exist for at least one ray. Labeling
  could still be possible because $H_{C-max}$ could be a viable blocking
  hyperedge itself. Hence proceed.}
%\ENDIF

\STATE $i \leftarrow 0$
\FOR {every hyperedge $H \in \cK$}
  \STATE $i \leftarrow i+1$
  \STATE $\cl(H) \leftarrow P_i$ where $P_i$ is the path in $T$
  containing leaf $v_i$ such that $|P_i| = |H|$.
  \STATE $\cL \leftarrow \cL \cup \{H\}$
\ENDFOR % every $H \in \cK$
\STATE Return the number of initialized rays, $i$.
\end{algorithmic} %}
\end{algorithm}



\begin{algorithm}[h]
\caption{$saturate\_rays\_and\_find\_blocking\_hyperedge(\cO, T)$}
\label{al:icppl-saturate_rays_symm_starlike}
%{\tiny
\begin{algorithmic}[1]
 
\STATE Variable $\cB_i$ shall store the blocking hyperedge for $i$th ray. Init
variables: for every $i \in [l]$, $\cB_i \leftarrow \emptyset$
\STATE Let $\cL_i \subseteq \cL$ containing hyperedges labeled to $i$th ray i.e. $\cL_i = \{H \mid
\cl(H) \subseteq R_i\}$

\FOR {every $i \in [l]$} 
\STATE \COMMENT {for each ray}
  \IF {$L_i = \emptyset$} 
  \STATE \COMMENT {Due to the condition \ref{line:Li_is_empty} in algorithm
  \ref{al:icppl-initialize_rays_symm_starlike}}
    \STATE $\cK  \leftarrow \{ H \mid H \in \cO \setminus \cL$ s.t. neighbours of
    $H$ in the overlap graph of $\cO$ form a clique$\}$ 
    \STATE Pick an inclusion chain $C \subseteq \cK$ and let
    $H_{C-max}$ be the maximal inclusion hyperedge in $C$.
    \STATE $\cB_i \leftarrow H_{C-max}$ \COMMENT {Since $H_{C-max} \in
    \cL$, and due to earlier subroutines, $|H_{C-max}| > k+2 $}
  \ENDIF % $L_i = \emptyset$

  \WHILE {$\cB_i = \emptyset$ and there exists $H \in \cO \setminus \cL$, such that $H$ overlaps with some
    hyperedge $H' \in \cL_i$} 
    \STATE $d \leftarrow |H \setminus H'|$ 
    \STATE Let $u$ be the end vertex of the path $\cl(H')$ that is
    closer to the center $r$, than its other end vertex 

    \IF {$d \le dist(u, r)+1$} 
      \STATE Use ICPIA to assign path $P \subseteq R_i$ to $H$ 
      \STATE $\cl(H) \leftarrow P$ \COMMENT {Update variables}
      \STATE $\cL \leftarrow \cL \cup \{H\}$, $\cL_i \leftarrow \cL_i \cup \{H\}$
   \ELSE
      \STATE $\cB_i \leftarrow H$
      \STATE Continue \COMMENT{Found the blocking hyperedge for this ray; move on to
      next ray}
    \ENDIF    
  \ENDWHILE % $\cB_i = \emptyset$
\ENDFOR

\end{algorithmic} %}
\end{algorithm}


\begin{algorithm}[h]
\caption{$partial\_path\_labeling\_of\_blocking\_hyperedges(\cO, T)$}
\label{al:icppl-partial_labeling_symm_starlike}
%{\tiny
\begin{algorithmic}[1]

\STATE \COMMENT {Process equal blocking hyperedges. At this point for
  all $i \in [l]$, $\cB_i \ne \emptyset$.} 
\FOR {every $i \in [l], \cB_i \ne \emptyset$}
  \FOR {every $j \in [l]$}
    \IF {$\cB_i = \cB_j$}
      \STATE \COMMENT {Blocking hyperedges of $i$th and $j$th rays are same}
      \STATE Let $H \leftarrow \cB_i$ \COMMENT { or $\cB_j$ }
      \STATE Find path $P$ on the path $R_i \cup R_j$ to assign to
      $H$ using ICPIA
      \STATE $\cl(H) \leftarrow P$
      \STATE $\cL \leftarrow \cL \cup \{H\}$
      \STATE $\cB_i \leftarrow \emptyset$, $\cB_j \leftarrow \emptyset$ \COMMENT
      {Reset blocking hyperedges for $i$th and $j$th rays}
    \ENDIF
  \ENDFOR % j > i
\ENDFOR % i

\STATE \COMMENT {Process intersecting blocking hyperedges}
\FOR {every $i \in [l], \cB_i \ne \emptyset$}
  \FOR {every $j \in [l]$}
    \IF {$\cB_i \cap (supp(\cL_j \cup
    \cB_j)) \ne \emptyset$}
      \STATE \COMMENT {Blocking hyperedge of $i$th ray intersects with hyperedge
    associated with $j$th ray}
      \STATE Find interval $P_i$ for $\cB_i$, on the path $R_i \cup R_j$ that
      satisfies ICPIA. 
      \STATE $\cl(\cB_i) \leftarrow P_i$
      \STATE $\cB_i \leftarrow \emptyset$
      \STATE $\cL \leftarrow \cL \cup \{\cB_i\}$
    \ENDIF
  \ENDFOR % j
\ENDFOR % i

\end{algorithmic} %}
\end{algorithm}



\section{Finding an assignment of tree paths to a set
  system} \label{sec:decompos} 


observation: T[V-c] is a collection of independent paths. c is the center.


In the previous section we have shown that
the problem of finding a Tree Path Labeling to an input $(\cF,T)$
is equivalent to finding an ICPPL to $\cF$ in tree $T$.  In this
section we characterize those set systems that have an ICPPL in a
given tree.  As a consequence of this characterization we identify two
sub-problems that must be solved to obtain an ICPPL.  We do not solve
these subproblems but use them as blackboxes to describe the rest of
the algorithm. In the next section, we solve one of these subproblems
for a smaller class of trees, $k$-subdivided stars.

\noindent
A set system can be concisely represented by a binary matrix where the
row indices denote the universe of the set system and the column
indices denote each of the sets. Let the binary matrix be $M$ with
order $n \times m$, the set system be $\cF = \{S_i \mid i \in [m]\}$,
universe of set system $U = \{x_i \mid i \in [n]\}$. We say $M$ represents
$\cF$, if $(i,j)$th element of $M$, $M_{ij} =
1$ iff $x_i \in S_j$. If $\cF$ has a feasible tree path labeling $\cl:
\cF \rightarrow \cP$, where $\cP$ is a set of paths from a given tree
$T$
% with vertex set $V(T) = \{ v_i \mid i \in [n] \}$
%$\cA = \{(S_i,P_i) \mid i \in [m]\}$, 
then we say its 
corresponding matrix $M$ has an ICPPL. 
Conversely, we say that a matrix
$M$ has an ICPPL if there exists an ICPPL $\cl$ as defined
above.

\noindent
We now consider the overlap graph of
$\cF$. The usage of overlap graph to
decompose the problem of consecutive ones testing was first introduced
by \cite{fg65}. They showed that a binary matrix or its corresponding
set system has the COP iff each connected component of the overlap
graph (the sets corresponding to this component or its corresponding
submatrix) has the COP. The same approach is also described in
\cite{wlh02,nsnrs09}. We use this idea to decompose $M$ and construct
a partial order on the components similarly. The resulting structural
observations are used to come up with the required algorithm for tree
path assignment. 

\noindent
A prime sub-matrix of $M$ is defined as the matrix formed by a set of
columns of $M$ which correspond to a connected component of the graph
$\bO(\cF)$.  Let us denote the prime sub-matrices by $M_1,\ldots,M_p$ each
corresponding to one of the $p$ components of $\bO(\cF)$. Clearly, two
distinct matrices have a distinct set of columns.  Let $col(M_i)$ be
the set of columns in the sub-matrix $M_i$.  The support of a prime
sub-matrix $M_i$ is defined as $supp(M_i) = \bigcup_{j
  \in col(M_i)}S_j$.  Note that for each $i$, $supp(M_i) \subseteq U$.
For a set of prime sub-matrices $X$ we define
$supp(X) = \bigcup_{M \in X} supp(M)$. 

\noindent
Consider the relation $\preccurlyeq$ on the prime sub-matrices $M_1,
\ldots, M_p$ defined as follows:
\begin{equation} 
\nonumber \{(M_i, M_j) \mid \text{ a set } S \in
  M_i \text{ is contained in a set } S' \in M_j\} \cup \{(M_i,M_i) \mid i \in[p]\} 
\end{equation}

\noindent
This relation is the same as that defined in \cite{nsnrs09}. The prime
submatrices and the above relation can be defined for any set
system. We will use this structure of prime submatrices to present our
results on an ICPPL for a set system $\cF$. Recall the following
lemmas and theorem that $\preccurlyeq$ is a partial order, from
\cite{nsnrs09}.

\begin{lemma} \label{lem:containment}
Let $(M_i,M_j) \in \preccurlyeq$.  Then there is a set $S' \in M_j$ such that for each $S \in M_i$, $S \subseteq S'$. 
\end{lemma}
\vspace{-3mm}
\begin{lemma}
For each pair of prime sub-matrices, either $(M_i,M_j) \not\in \preccurlyeq$ or $(M_j,M_i) \not\in \preccurlyeq$.
%If $(M_i,M_j) \in \preccurlyeq$ and $(M_j,M_i) \in \preccurlyeq$, then $i = j$ and $|M_i| = 1$.
\end{lemma}
\vspace{-3mm}
\begin{lemma}
If $(M_i,M_j) \in \preccurlyeq $ and $(M_j,M_k) \in \preccurlyeq$, then $(M_i,M_k) \in \preccurlyeq$.
\end{lemma}
\vspace{-3mm}
\begin{lemma} \label{lem:twoparents}
If $(M_i,M_j) \in \preccurlyeq$ and $(M_i,M_k) \in \preccurlyeq$, then
either $(M_j,M_k) \in \preccurlyeq$ or $(M_k,M_j) \in \preccurlyeq$. 
\end{lemma}
\vspace{-3mm}
\begin{theorem} \label{thm:partitionold}
  $\preccurlyeq$ is a partial order on the set of prime sub-matrices
  of $M$.  Further, it uniquely partitions the prime sub-matrices of
  $M$ such that on each set in the partition $\preccurlyeq$ induces a
  total order.
\end{theorem}
\noindent
For the purposes of this paper, we refine the total order mentioned in
Theorem \ref{thm:partitionold}. We do this by identifying an in-tree
rooted at each maximal upper bound under $\preccurlyeq$.  Each of
these in-trees will be on disjoint vertex sets, which in this case
would be disjoint sets of prime-submatrices.  The in-trees are
specified by selecting the appropriate edges from the Hasse diagram
associated with $\preccurlyeq$.  Let $\cI$ be the following set:
\vspace{-3mm}
\begin{align*}
  \cI = \{ (M_i,M_j) \space \in \space \preccurlyeq \mid \nexists M_k \text{ s.t. } M_i \preccurlyeq M_k, M_k \preccurlyeq M_j
  \} \cup \{ (M_i,M_i), i \in [p] \}
\end{align*}

\begin{theorem} \label{thm:partition} Consider the directed graph $X$
  whose vertices correspond to the prime sub-matrices, and the edges
  are given by $\cI$.  Then, $X$ is a vertex disjoint collection of
  in-trees and the root of each in-tree is a maximal upper bound in
  $\preccurlyeq$.
\end{theorem}
\begin{proof}
To observe that $X$ is a collection of in-trees, we observe that for         
vertices corresponding to maximal upper bounds, no out-going edge is
present in $X$.  Secondly, for each other element, exactly one
out-going edge is chosen (due to lemma \ref{lem:twoparents} and the
condition in set $\cI$ definition), and for the 
minimal lower bound, there is no in-coming edge.  Consequently, $X$ is
acyclic, and since each vertex has at most one edge leaving it, it
follows that $X$ is a collection of in-trees, and for each in-tree,
the root is a maximal upper bound in $\preccurlyeq$.  Hence the
theorem. 
\qed
\end{proof}

\noindent
Let the partition of $X$ given by Theorem \ref{thm:partition} be
$\{X_1,\ldots,X_r\}$.  Further, each in-tree itself  
can be layered based on the distance from the root.  The root is
considered to be at level zero. For $j \geq 0$, Let $X_{i,j}$ denote
the set of prime matrices in level $j$ of in-tree $X_i$.   

\begin{lemma}
\label{lem:subicppa}
Let $M$ be a matrix and let $X$ be the directed graph whose vertices
are in correspondence with the prime submatrices of $M$.  Further let
$\{X_1,\ldots,X_r\}$ be the partition of $X$ into in-trees as defined
above. 
Then, matrix $M$ has an ICPPL in tree $T$ iff $T$ can be partitioned
into vertex disjoint subtrees 
 $\{T_1, T_2, \dots T_r\}$ such that, for each $1 \leq i \leq r$, the
 set of prime sub-matrices corresponding to vertices in $X_i$ has an
 ICPPL in $T_i$. 
\end{lemma}
\begin{proof}
Let us consider the reverse direction first.  Let us assume that $T$
can be partitioned into $T_1, \ldots, T_r$ such 
that for each $1 \leq i \leq r $, the set of prime sub-matrices
corresponding to vertices in $X_i$ has an ICPPL in $T_i$.  It is clear
from the properties of the partial order $\preccurlyeq$ that these
ICPPLs naturally yield an ICPPL  of $M$ in $T$.  The main property
used in this inference is that for each $1 \leq i \neq j \leq r$,
$supp(X_i) \cap supp(X_j) = \emptyset$.   

\noindent
To prove the forward direction, we show that if $M$ has an ICPPL, say
$\cA$, in $T$, then there exists 
  a partition of $T$ into vertex disjoint subtree $T_1, \ldots, T_r$
  such that for each $1 \leq i \leq r$, the set of prime sub-matrices
  corresponding to vertices in $X_i$ has an ICPPL in $T_i$.  For each
  $1 \leq i \leq r$, we define based on $\cA$ a subtree  
$T_i$ corresponding to $X_i$.  We then argue that the trees thus
defined are vertex disjoint, and complete the proof. 
 Consider $X_i$ and consider the prime sub-matrix in $X_{i,0}$.
 Consider the paths assigned under $\cA$ to the sets in the prime
 sub-matrix in $X_{i,0}$.  Since the component in $G_f$ corresponding
 to this matrix is a connected component, it follows that union of
 paths assigned to this prime-submatrix is a subtree of $T$.  We call
 this sub-tree $T_i$.  All other prime-submatrices in $X_i$ are
 assigned paths in $T_i$ since $\cA$ is an ICPPL, and the support of
 other prime sub-matrices in $X_i$ are contained in the support of the
 matrix in $X_{i,0}$.  Secondly, for each $1 \leq i \neq j \leq r$,
 $supp(X_i) \cap supp(X_j) = \emptyset$, and since $\cA$ is an ICPPL, it
 follows that $T_i$ and $T_j$ are vertex disjoint.  Finally, since
 $|U| = |V(T)|$, it follows that $T_1, \ldots, T_r$ is a partition of
 $T$ into vertex disjoint sub-trees such that for each $1 \leq i \leq
 r$, the set of matrices corresponding to nodes in $X_i$ has an ICPPL
 in $T_i$.  Hence the lemma. 
\qed
\end{proof}

\noindent
 The essence of the following lemma is that an ICPPL only needs to be
 assigned to the prime sub-matrix corresponding to the root of each
 in-tree, and all the other prime sub-matrices only need to have an
 Intersection Cardinality Preserving {\em Interval} Assignments (ICPIA).
 Recall, an ICPIA is an assignment of intervals to sets such that the
 cardinality of an assigned interval is same as the cardinality of the
 interval, and the cardinality of intersection of any two sets is same
 as the cardinality of the intersection of the corresponding
 intervals.  It is shown in \cite{nsnrs09} that the existence of an
 ICPIA is a necessary and sufficient condition for a matrix to have
 the COP. 

\noindent 
We present the pseudo-code to test if $M$ has an ICPPL in $T$. 
\begin{lemma} \label{lem:rooticppa}
Let $M$ be a matrix and let $X$ be the directed graph whose vertices
are in correspondence with the prime submatrices of $M$.  Further let
$\{X_1,\ldots,X_r\}$ be the partition of $X$ into in-trees as defined
earlier in this section. 
Let $T$ be the given tree and let $\{T_1, \ldots, T_r\}$ be a given
partition of $T$ into vertex disjoint sub-trees. 
Then, for each $1 \leq i \leq r$, the set of matrices corresponding to
vertices of $X_i$ has an ICPPL in $T_i$ if and only if the matrix in
$X_{i,0}$ has an ICPPL in $T_i$ and all other matrices in $X_i$ have
an {\bf {\em ICPIA}} on their path in $T_i$. 
\end{lemma}
\begin{proof}
The proof is based on the following fact - $\preccurlyeq$ is a partial
order and $X$ is a digraph which is the disjoint 
union of in-trees.  Each edge in the in-tree is a containment
relationship among the supports of the corresonding
sub-matrices. Therefore, any ICPPL to a prime sub-matrix that is not
the root is contained in a path assigned to the sets in the parent
matrix.  Consequently, any ICPPL to the prime sub-matrix that is not
at the root is an ICPIA, and any ICPIA can be used to construct an
ICPPL to the matrices corresponding to nodes in $X_i$ provided the
matrix in the root has an ICPPL in $T_i$.   Hence the lemma. \qed
\end{proof}
Lemma \ref{lem:subicppa} and Lemma \ref{lem:rooticppa} point out two
algorithmic challenges in finding an ICPPL for a given set system
$\cF$ in a tree $T$.  Given $\cF$, finding $X$ and its partition
$\{X_1,\ldots,X_r\}$ into in-trees can be done in polynomial time.  On
the other hand, as per lemma \ref{lem:subicppa} we need to parition
$T$ into vertex disjoint sub-trees $\{T_1, \ldots, T_r\}$ such that
for each $i$, the set of matrices corresponding  to nodes in $X_i$
have an ICPPL in $T_i$.  This seems to be a challenging step, and it
must be remarked that this step is easy when $T$ itself is a path, as
each individual $T_i$ would be sub-paths.  The second algorithmic
challenge is identified by lemma \ref{lem:rooticppa} which is to
assign an ICPPL from a given tree to the matrix associated with the
root node of $X_i$. 
  
\begin{algorithm}[h]
\caption{Algorithm to find an ICPPL for a matrix $M$ on tree $T$: $main\_ICPPL(M, T$)}
\label{Al:icppa-main}
\begin{algorithmic}
\STATE Identify the prime sub-matrices. This is done by constructing
the strict overlap graph and identifying the connected components. Each
connected component yields a prime sub-matrix.   \\ 
\STATE Construct the partial order $\preccurlyeq$ on the set of prime
sub-matrices.  \\ 
\STATE Construct the partition $X_1,\ldots,X_r$ of the  prime
sub-matrices induced by $\preccurlyeq$ \\
\STATE For each $1 \leq i \leq r$, Check if all matrices except those
in $X_{i,0}$ has an ICPIA.  If a matrix does not have ICPIA exit with
a negative answer.  To check for the existence of ICPIA, use the
result in \cite{nsnrs09}. 
\label{l:icppasubtree} \STATE  Find a partition of $T_1, \ldots, T_r$ such that matrices in
$X_{i,0}$ has an ICPPL in $T_i$.  If not such 
partition exists, exit with negative answer.  
\end{algorithmic}
\end{algorithm}




\section {Acknowlegements} 
We thank the anonymous referees of the WG 2011 committee and our
colleagues who helped us with the readability of this document.


%\bibliographystyle{plainnat}
%\bibliographystyle{alpha} %to have only [i] type of citation
%\bibliographystyle{agsm}  % another formatting of natbib
%\bibliographystyle{plain} %to have only [i] type of citation

\bibliography{../lib/cop-variants}


\pagebreak
\appendix

\section{Detailed proofs}
{\bf Proof of Lemma \ref{lem:setminuscard}}\\
\begin{proof}
  Let $P_i = \cl(S_i)$, for all $1 \le i \le  3$.
  $|S_1 \cap (S_2 \setminus S_3)| = |(S_1 \cap S_2) \setminus S_3| =
  |S_1 \cap S_2| - |S_1 \cap S_2 \cap S_3|$. Due to conditions (ii)
  and (iii) of ICPPL, $|S_1 \cap S_2| - |S_1 \cap S_2 \cap S_3| = |P_1
  \cap P_2| - |P_1 \cap P_2 \cap P_3| = |(P_1 \cap P_2) \setminus P_3|
  = |P_1 \cap (P_2 \setminus P_3)|$. Thus lemma is proven. \qed
\end{proof}

\noindent
{\bf Proof of Lemma \ref{lem:fourpaths}}\\
\begin{proof}
  {\em Case 1:} Consider the path $P = P_3 \cap P_4$ (intersection of
  two paths is a path).
  % Since $P_1, P_2$ share a leaf, the following are paths $P_1
  % \setminus P_2$, $P_2 \setminus P_1$, $P_1 \cap P_2$ and they are
  % mutually disjoint.
  Suppose in this case, $P$ does not intersect with $P_1 \setminus
  P_2$, i.e. $P \cap (P_1 \setminus P_2) = \emptyset$. Then $P \cap P_1 \cap
  P_2 = P \cap P_2$. Similarly, if $P \cap (P_2 \setminus P_1) = \emptyset$,
  $P \cap P_1 \cap P_2 = P \cap P_1$. Thus it is clear that if the
  intersection of any two paths does not intersect with any of the set
  differences of
  the remaining two paths, the claim in the lemma is true.\\
  {\em Case 2:} The other possibilty is the compliment of the previous
  case which is as follows. So let us assume that the intersection of
  any two paths intersects with both the set differences of the other
  two. First let us consider $P \cap (P_1 \setminus P_2) \ne \emptyset$ and
  $P \cap (P_2 \setminus P_1) \ne \emptyset$, where $P = P_3 \cap P_4$. Since
  $P_1$ and $P_2$ share a leaf, there is exactly one vertex at which
  they branch off from the path $P_1 \cap P_2$ into two paths $P_1
  \setminus P_2$ and $P_2 \setminus P_1$. Let this vertex be $v$. It
  is clear that if path $P_3 \cap P_4$, must intersect with paths $P_1
  \setminus P_2$ and $P_2 \setminus P_1$, it must contain $v$ since
  these are paths from a tree. Moreover, $P_3 \cap P_4$ intersects
  with $P_1 \cap P_2$ at exactly $v$ and only at $v$ which means that
  $P_1 \cap P_2$ does not intersect with $P_3 \setminus P_4$ or $P_4
  \setminus P_3$ which contradicts the assumption of this case. Thus
  this  case cannot occur and case 1 is the only possible scenario. \\
  Thus lemma is proven. \qed
\end{proof}

\noindent
{\bf Proof of Lemma \ref{lem:invar1}}\\
\begin{proof}
  \noindent
  \begin{enumerate}
  \item [Case 1:] {\em Invariant I and II} 
    \begin{enumerate}
    \item [Case 1.2:] {\em $R$ is a new set:}\\ 
      If $R = S_1 \cap S_2$, $|R| = |S_1 \cap S_2| = |\cl_{j-1}(S_1) \cap
      \cl_{j-1}(S_2)|\footnote{Inv III hypothesis} = |\cl_j(S_1 \cap
      S_2)|\footnote{$\cl_j$ definition} = |\cl_j(R)|$\\
      If $R = S_1 \setminus S_2$, 
         $|R| = |S_1 \setminus S_2| 
              = |S_1| - |S_1 \cap S_2| 
              = |\cl_{j-1}(S_1)| - |\cl_{j-1}(S_1) \cap \cl_{j-1}(S_2)|\footnote{Inv II and III hypothesis} 
              = |\cl_{j-1}(S_1) \setminus \cl_{j-1}(S_2)| 
              = |\cl_j(S_1 \setminus S_2)|\footnote{$\cl_j$ definition}
              = |\cl_j(R)|$. \\
      Thus Invariant II proven.
   \end{enumerate}
  \item [Case 2:] {\em Invariant III}
    \begin{enumerate}
    \item [Case 2.2:] {\em Only $R$ is a new set:}\\
      If $R = S_1 \cap S_2$, $|R \cap R'| = |S_1 \cap S_2 \cap R'| = |\cl_{j-1}(S_1) \cap
      \cl_{j-1}(S_2) \cap \cl_{j-1}(R')|\footnote{Inv IV hypothesis} = |\cl_j(S_1 \cap
      S_2) \cap \cl_j(R')|\footnote{$\cl_j$ definition. Note that $R'$ is not a
        new set} = |\cl_j(R) \cap \cl_j(R')|$\\
      If $R = S_1 \setminus S_2$, 
         $|R \cap R'| = |(S_1 \setminus S_2) \cap R'| 
                      = |(\cl_{j-1}(S_1) \setminus \cl_{j-1}(S_2))
                      \cap \cl_{j-1}(R')|\footnote{Lemma \ref{lem:setminuscard}}
                      = |\cl_{j}(R) \cap \cl_{j}(R')|\footnote{$\cl_j$
                        definition. Note $R'$ is not a new set}$\\
      Thus Invariant III proven.
    \end{enumerate}
   \item [Case 3:] {\em Invariant IV}
     \begin{enumerate}
       \item [Case 3.2:] {\em Only R is a new set:}\\
         If $R = S_1 \cap S_2$, Consider, $|\cl_{j-1}(S_1) \cap \cl_{j-1}(S_2) \cap \cl_{j-1}(R') \cap \cl_{j-1}(R'')|$. We know from lemma \ref{lem:fourpaths} that the intersection
of these four paths is same as the intersection of three distinct paths among the four.  
Let us call these four paths $P_1,P_2, P_3,P_4$ and without loss of generality, 
let it be that $\displaystyle \cap_{i=1}^4 P_i  = \cap_{i=1}^3 P_i$. Further $|\cap_{i=1}^4 P_i|=| S_1 \cap S_2 \cap R'|$ by the invariant IV of the induction hypothesis.  
Therefore, it follows that $|\cap_{i=1}^4 P_i| \geq |S_1 \cap S_2 \cap R' \cap R''|$.\\
\begin{comment}
 Next, we write $\displaystyle |\cap_{i=1}^4 P_i| = |P_4| + |\displaystyle \cap_{i=1}^3 P_i| - |P_4 \cup \cap_{i=1}^3 P_i|$. Clearly $\displaystyle P_4 \cup \cap_{i=1}^3 P_i = P_4$.  By
 induction hypothesis Invariant I and IV, We can now write 
$\displaystyle |\cap_{i=1}^4 P_i| = |S_4| + |\cap_{i=1}^3 S_i| - |S_4|$.  
Since $\cl_{j-1}$ is an ICPPL in which $S_4$ is mapped to
$P_4$, and $P_4$ contains $\displaystyle \cap_{i=1}^3 P_i$, it follows that 
$|S_4|=|P_4| \geq|\cap_{i=1}^3 P_i| = |S_1 \cap S_2 \cap R'| \geq |S_1 \cap S_2 \cap R' \cap R''|$.  
Therefore, $\displaystyle |\cap_{i=1}^4 P_i| \leq |S_1 \cap S_2 \cap R' \cap R''|$, 
and equality of these two terms follows because we have also proved the inequality in the
opposite direction. It now follows that$|\cl_j(R) \cap \cl_j(R') \cap \cl_j(R'')|=|\cl_{j-1}(S_1) \cap \cl_{j-1}(S_2) \cap \cl_{j-1}(R') \cap \cl_{j-1}(R'')| = |\cap_{i=1}^4 P_i| = |S_1 \cap S_2 \cap R' \cap R''|= |R \cap R' \cap R''|$. This completes induction hypothesis in this case. \\
\end{comment}
      If $R = S_1 \setminus S_2$, a similar argument using Lemma \ref{lem:setminuscard} and the induction hypothesis completes the proof of this case.\\
      Thus Invariant IV proven.
      \end{enumerate}
  \end{enumerate}
\end{proof}

% \begin{lemma}[Observation]
%   \label{lem:fourpaths} Consider four paths in a tree $P_1, P_2, P_3,
%   P_4$ such that they have non-empty pairwise intersection and paths $P_1,
%   P_2$ share a leaf. Then there exists distinct $i, j, k \in
%   \{1,2,3,4\}$ such that, $P_1 \cap P_2 \cap P_3 \cap P_4 = P_i \cap
%   P_j \cap P_k$.
% \end{lemma}


\end{document}





