%%
% Author: Anju Srinivasan Zabil
%%


%\documentclass{../../lib/llncs}
\documentclass{llncs}

\usepackage{fullpage}
\usepackage{latexsym}
\usepackage{amssymb}
\usepackage{amsfonts}
\usepackage{amsmath}
\usepackage{comment}
\usepackage{epsfig}
\usepackage{graphicx}
\usepackage{epstopdf}
\usepackage{algorithm}
\usepackage{algorithmic}
\usepackage{natbib}

% for Natbib
%\bibpunct{(}{)}{;}{a}{,}{,}

\def\Remark{\noindent{\bf Remark:~}}
\long\def\denspar #1\densend {#1}
\def\DEF{\stackrel{\rm def}{=}}

%%% string defs
\def\cA{{\cal A}}
\def\cB{{\cal B}}
\def\cC{{\cal C}}
\def\cD{{\cal D}}
\def\cE{{\cal E}}
\def\cF{{\cal F}}
\def\cG{{\cal G}}
\def\cH{{\cal H}}
\def\cI{{\cal I}}
\def\cJ{{\cal J}}
\def\cK{{\cal K}}
\def\cL{{\cal L}}
\def\cM{{\cal M}}
\def\cN{{\cal N}}
\def\cO{{\cal O}}
\def\cP{{\cal P}}
\def\cQ{{\cal Q}}
\def\cR{{\cal R}}
\def\cS{{\cal S}}
\def\cT{{\cal T}}
\def\cU{{\cal U}}
\def\cV{{\cal V}}
\def\cW{{\cal W}}
\def\cX{{\cal X}}
\def\cY{{\cal Y}}
\def\cZ{{\cal Z}}
\def\hA{{\hat A}}
\def\hB{{\hat B}}
\def\hC{{\hat C}}
\def\hD{{\hat D}}
\def\hE{{\hat E}}
\def\hF{{\hat F}}
\def\hG{{\hat G}}
\def\hH{{\hat H}}
\def\hI{{\hat I}}
\def\hJ{{\hat J}}
\def\hK{{\hat K}}
\def\hL{{\hat L}}
\def\hP{{\hat P}}
\def\hQ{{\hat Q}}
\def\hR{{\hat R}}
\def\hS{{\hat S}}
\def\hT{{\hat T}}
\def\hX{{\hat X}}
\def\hY{{\hat Y}}
\def\hZ{{\hat Z}}
\def\eps{\epsilon}
\def\C{{\mathcal C}}
\def\F{{\mathcal F}}
\def\A{{\mathcal A}}
\def\hd{\hat{\delta}}
\def\Lr{\Leftrightarrow}
\def\If{{\bf if }}
\def\Then{{\bf then }}
\def\Else{{\bf else }}
\def\Do{{\bf do }}
\def\While{{\bf while }}
\def\Continue{{\bf continue }}
\def\Repeat{{\bf repeat }}
\def\Until{{\bf until }}
\def\eqdef{\stackrel {\triangle}{=}}
\def\squarebox#1{\hbox to #1{\hfill\vbox to #1{\vfill}}}


%%% new/renew commands
\renewcommand{\algorithmiccomment}[1]{// #1 } % comment in algorithmic
\newtheorem{observation}{Observation}
\newcommand{\Eqr}[1]{Eq.~(\ref{#1})}
\newcommand{\diff}{\ne}
\newcommand{\OO}[1]{O\left( #1\right)}
\newcommand{\OM}[1]{\Omega\left( #1 \right)}
\newcommand{\Prob}[1]{\Pr\left\{ #1 \right\}}
\newcommand{\Set}[1]{\left\{ #1 \right\}}
\newcommand{\Seq}[1]{\left\langle #1 \right\rangle}
\newcommand{\Range}[1]{\left\{1,\ldots, #1 \right\}}
\newcommand{\ceil}[1]{\left\lceil #1 \right\rceil}
\newcommand{\floor}[1]{\left\lfloor #1 \right\rfloor}
\newcommand{\ignore}[1]{}
\newcommand{\eq}{\equiv}
\newcommand{\abs}[1]{\left| #1\right|}
\newcommand{\set}[1]{\left\{ #1\right\}}
\newcommand{\itoj}{{i \rightarrow j}}
\newcommand{\view}{\mbox{$COMM$}}
\newcommand{\pview}{\mbox{$PView$}}
\newcommand{\vx}{\mbox{${\vec x}$}}
\newcommand{\vy}{\mbox{${\vec y}$}}
\newcommand{\vv}{\mbox{${\vec v}$}}
\newcommand{\vw}{\mbox{${\vec w}$}}
\newcommand{\vb}{\mbox{${\vec b}$}}
\newcommand{\basic}{\mbox{\sc Basic}}
\newcommand{\WR}{\mbox{$\lfloor wr \rfloor$}}
\newcommand{\guarantee}{\mbox{\sc BoundedDT}}
\newcommand{\sq}{{\Delta}}
\newcommand{\Smin}{{S_{0}}}
\newcommand{\outt}{{D^{^+}}}
\newcommand{\outtp}{{\overline{D^{^+}}}}
\newcommand{\inn}{{D^{^-}}}
\newcommand{\innp}{{\overline{D^{^-}}}}
\newcommand{\indexx}{{\gamma}}
\newcommand{\D}{{D}}
\newenvironment{denselist}{
  \begin{list}{(\arabic{enumi})}{\usecounter{enumi}
      \setlength{\topsep}{0pt} \setlength{\partopsep}{0pt}
      \setlength{\itemsep}{0pt} }}{\end{list}}
\newenvironment{denseitemize}{
  \begin{list}{$\bullet$}{ \setlength{\topsep}{0pt}
      \setlength{\partopsep}{0pt} \setlength{\itemsep}{0pt}
    }}{\end{list}}
\newenvironment{subdenselist}{
  \begin{list}{(\arabic{enumi}.\arabic{enumii})}{ \usecounter{enumii}
      \setlength{\topsep}{0pt} \setlength{\partopsep}{0pt}
      \setlength{\itemsep}{0pt} }}{\end{list}}


% D O C U M E N T
\begin{document}
\title{On Extensions of Consecutive Ones Testing}

\author{N.S. Narayanaswamy \inst{1} \and Anju Srinivasan Zabil \inst{2}}

\institute{ %The Institute of Mathematical Sciences, Chennai,
  % \email{arvind@imsc.res.in} \and
  Indian Institute of Technology
  Madras, Chennai, \email{swamy@cse.iitm.ernet.in},\\
  Indian Institute of Technology Madras, Chennai,
  \email{anjuzabil@gmail.com}}


\date{}

\maketitle

\begin{abstract}
  Consecutive ones property (COP) of a binary matrix is a
  combinatorial property of the matrix such that there exists a
  permutation of the rows of the matrix which makes all the
  ones in the columns consecutive. Detecting this property in a
  matrix is polynomial time solvable and there are several algorithms
  known. This problem is also the same as interval assigments to a set
  system. In this paper we extend this idea to tree path assignments
  to a set system and give a characterization for it.
\end{abstract}

\section{Introduction}
%\subsection{Previous Work}

\noindent
Consecutive ones property (COP) of binary matrices is a widely studied
combinatorial problem. The problem is to rearrange rows(columns) of a
binary matrix in such a way that every column(rows) has its $1$s occur
consecutively. If this is possible the matrix is said to have COP.  It
has several practical applications in diverse fields including
scheduling\cite{hl06}, information retrieval\cite{k77} and computational biology\cite{abh98}. Within
theory it is used as a tool in graph theory\cite{mcg04} and integer linear
programming\cite{ht02,hl06}. 
Recognition of COP is polynomial time
solvable by several algorithms. PQ trees\cite{bl76}, variations of PQ
trees\cite{mm09,wlh01,wlh02,mcc04}, ICPIA\cite{nsnrs09} are the main
ones. 
\noindent
The problem of COP can be easily seen as a 
simple constraint satisfaction problem involving a system of sets from
a universe. Every column of the binary matrix can be converted into a
set of integers which are the indices of the rows with $1$s in that
column. When observed in this context, if the matrix has COP, a
reordering of its rows will result in sets that have only consecutive
integers. In other words, the sets are intervals. Indeed now the
problem now becomes finding interval assignments to the given set
system \cite{nsnrs09} with a single permutation of the universe (set
of row
indices) permutes each set to its interval. The idea of this algorithm
is similar to that in \cite{wlh02} but much simpler. 
\noindent
A natural generalization of the interval assignment problem is
feasible tree path assignments to a set system which is the topic of
this paper. The problem is defined as follows - given a set system
from a universe $U$ and a tree $T$, can there be a bijection from the
$U$ to the vertices of $T$ such that each set in the system maps to a
path in $T$. As a special case if $T$ is a path the problem becomes
the interval assignment problem or the ICPIA as it is called in
\cite{nsnrs09}. 
\noindent 
It can be noted that intersection graphs of ICPIA set systems can be
characterized by chordal graphs. Chordal
graphs are of great significance and has several applications. One of
the well known and interesting properties of a chordal graphs is its
connection with intersection graphs\cite{mcg04}. For every chordal
graph, there exists a tree and a family of subtrees of this tree such
that the intersection graph of this family is isomorphic to the
chordal graph\cite{plr70,gav78,bp93}. Certain format of these trees are called clique
trees\cite{apy92} of the graph which is a compact representation of the
chordal graph. There has also been work done on the generalization of
clique trees to clique hypergraphs\cite{km02}. In the chordal graph as
intersection graphs, if all the subtrees that map to the vertices of the
graph in the isomorphism 
are paths on the tree, it is called a path graph\cite{mcg04} or path
chordal graph.
If a set system is an ICPIA or has feasible tree path assignment, its
the intersection graph is an interval graph or a path graph
respectively. This is because every vertex represents a set that can be permuted to  an
interval or a path in a tree. Path graph recognition is polynomial
solvable\cite{gav78,aas93}. If a given set system has an intersection
graph that is isomorphic to a path graph, it is not necessary that the
set system has feasible tree path assignment. It would be interesting
to study if feasible tree path assignments can be used in path graph
isomorphism.


% \noindent
% It has been long known that interval graph recognition is in
% logspace\cite{rei84}. Recently interval graph isomorphism was also
% shown to be logspace decidable using a logspace canonization algorithm
% by \cite{kklv10}. 
% This result is built on top of logspace results of
% undirected graph connectivity \cite{rei08}, logspace tractability
% using a certain logical formalism called FP+C and modular
% decomposition of interval graphs\cite{lau10} etc.
% Interval graphs are closely connected to binary matrices with COP. The
% maximal clique vertex incidence matrix (matrix with rows representing
% maximal cliques and columns representing vertices of a graph) has COP
% on columns iff the graph is an interval graph\cite{fg65}. This follows
% from the interval graph characterization by \cite{gh64}. Due to this
% close relation it is natural to see if consecutive ones property can be
% tested in logspace. \\
% \noindent
% We also explore some extensions of the interval assignment problem in
% \cite{nsnrs09}, namely caterpillar path assignment problem.

% We present a logspace algorithm here that uses the
% ICPIA characterization of binary matrices with COP (set system
% associated with such a matrix)\cite{nsnrs09}.


% \section{Preliminaries}
% A {\em hypergraph} $\cH=(V,E)$  has vertex set $V=\{x_1,x_2, \dots x_n\}$
% and edge set $E \subseteq V$.

% \noindent
% The collection $\F = \{S_i \mid S_i \subseteq U, S_i \ne \O, i \in
% I\}$ is a {\em set system} of a universe $U$.

% \noindent
% Consider a {\em binary matrix} $M$ of order $n \times m$.  The set
% sysetm corresponding to the binary matrix is $\cF_M = \{S_i \mid S_i
% \subseteq U, i \in [m]\}$ where $U = \{x_i \mid i \in [n]\}$ such that
% $x_j \in S_i$ iff $M_{ij} = 1$.

\section{Preliminaries (WIP)}
In this paper, the collection $\F = \{S_i \mid S_i \subseteq U, S_i
\ne \O, i \in I\}$ is a {\em set system} of universe, $U = \{1,
\ldots, n\}$. Moreoever, a set system is assumed to ``cover'' the universe,
i.e. $ \bigcup_{i \in I}S_i = U$. \\

\noindent
The graph $T=(V,E)$ represents a given tree with $|V| = n$. Further,
for simplicity, $V$ is defined as $\{1,\ldots,n\}$. All paths
referred to in this paper are paths on a tree. \\

\noindent
Generalizing the definition of {\em feasibility} in \cite{nsnrs09} to
a set assignment, a set assignment $\A_s$ is defined to be {\em
  feasible} if there exists a bijection defined as follows.
\begin{align}
  \sigma: U \rightarrow Z, \text{ such that }\sigma(S_i) = Q_i \text{
    for all } i \in I, \sigma \text{ is a bijection}
\label{eq:stf}
\end{align}
The assignment $\A_s$ is then called {\em Set Translation Feasible}.\\

\noindent
A {\em path assignment} $\A$ to $\F$ is defined as a set assignment
where second universe is the vertex set $V$ of a given tree $T$ and
every second subset in the ordered pairs is a path in this tree. Formally, the definition is as follows.
\begin{align*}
  \A = \{ (S_i,P_i) \mid  S_i \in \F, P_i \subseteq V \text{ s.t. }T[P_i]
  \text{ is a path, } i \in I \}
\end{align*}
In other words, $P_i$ is the path on the tree $T$ assigned to $S_i$ in
$\A$. As mentioned before for set systems, the paths cover the whole
tree, i.e. $\bigcup_{i \in I}P_i = V$ \\

\noindent
Let $X$ be a partially ordered set. The element $mub(X)$ represents
the maximal upper bound of a poset $X$. A maximal upper bound $X_m$ of
a poset $X$ is such that, $\nexists X_q \mid X_m,X_q \in X, X_m
\preccurlyeq X_q$. $\preccurlyeq$ being the partial order.\\

\noindent
The set $I$ represents the index set $[m]$. If index $i$ is used
without further qualification, it is meant to be $i \in I$. Any
function, if not defined on a domain of sets, when applied on a set is
understood as the function applied to 
each of its elements. i.e. for any function $f$ defined with domain $U$, the abuse of
notation is as follows; $f(S)$ is used instead of $\hat f(S)$ where
$\hat f(S) = \{y \mid y = f(x), x \in S\}$. \\

\noindent
When refering to a tree as $T$ it could be a reference to the tree
itself, or the vertices of the tree. This will be obvious from the
context.\\

\noindent
An in-tree is an oriented tree in which a single vertex is reachable from every other vertex.


\section{Tree path assignment}

A natural extension of the interval assignment problem is assignment
of paths from a tree to a set system. 
Consider a path assignment $\cA = \{(S_i, P_i) \mid S_i \in \cF, P_i
\text{ is a path from $T$}, i \in [m]\}$ to a set system $\cF = \{S_i \mid S_i
\subseteq U, i \in [m]\}$, were $T$ is a given tree, $U$
is the set system's universe and $m$ is the number of sets in $\cF$. We call $\cA$ an {\em Intersection
Cardinality Preserving Path Assignment (ICPPA)} if it has the following properties.

\begin{enumerate}
\item [i.]  $|S_i| = |P_i|$ for all $i \in [m]$
\item [ii.] $|S_i \cap S_j| = |P_i \cap P_j|$ for all $i,j \in [m]$
%\item [iii.] $|\bigcup_{i \in J} S_i| = |\bigcup_{i \in J} P_i|$ for
%  all $J \subseteq I$\footnote{TBD: this is needed now to fix some
%  proofs}
\item [iii.] $|S_i \cap S_j \cap S_k| = |P_i \cap P_j \cap P_k|$ for all $i,j,k \in [m]$
\end{enumerate}

\begin{lemma}
\label{lem:setminuscard}
  If $\cA$ is an ICPPA, and $(S_1, P_1),(S_2, P_2),(S_3, P_3) \in
  \cA$, then $|S_1 \cap (S_2 \setminus S_3)| = |P_1 \cap (P_2 \setminus P_3)|$.
\end{lemma}
\begin{proof}
$|S_1 \cap (S_2 \setminus S_3)| = |(S_1 \cap S_2) \setminus S_3| =
|S_1 \cap S_2| - |S_1 \cap S_2 \cap S_3|$. Due to conditions (ii) and (iii) of
ICPPA, $|S_1 \cap S_2| - |S_1 \cap S_2 \cap S_3| = |P_1 \cap P_2| - |P_1 \cap P_2 \cap
P_3| = |(P_1 \cap P_2) \setminus P_3| =  |P_1 \cap (P_2 \setminus
P_3)|$. Thus lemma is proven. \qed
\end{proof}

\begin{lemma}
  \label{lem:fourpaths} Consider four paths in a tree $Q_1, Q_2, Q_3,
  Q_4$ such that they have nonempty pairwise intersection and $Q_1,
  Q_2$ share a leaf. Then there exists $i, j, k \in \{1,2,3,4\}$ with no two of $i,j,k$
  being equal such that, $Q_1 \cap Q_2 \cap Q_3 \cap Q_4 = Q_i \cap
  Q_j \cap Q_k$.
\end{lemma}
\begin{proof}
  {\em Case 1:} w.l.o.g, consider $Q_3 \cap Q_4$ and let us call it $Q$. This is clearly a
  path (intersection of two paths is a path). 
% Since $Q_1, Q_2$ share a
%   leaf, the following are paths $Q_1 \setminus Q_2$, $Q_2 \setminus
%   Q_1$, $Q_1 \cap Q_2$ and they are mutually disjoint. 
  Suppose $Q$
  does not intersect with $Q_1 \setminus Q_2$, i.e. $Q \cap (Q_1
  \setminus Q_2) = \O$. Then $Q \cap
  Q_1 \cap Q_2 = Q \cap Q_2$. Similarly, if $Q \cap (Q_2 \setminus
  Q_1) = \O$, $Q \cap Q_1 \cap Q_2 = Q \cap Q_1$. Thus it is
  clear that if the intersection of any two paths does not intersect
  with any of the set differences of the remaining two paths, the
  claim in the lemma is true. 
  % Note that $Q_1 \setminus Q_2$ and $Q_2
%   \setminus Q_1$ are paths because $Q_1, Q_2$ share a leaf.\\ 
  {\em Case 2:} Let us consider the compliment of the previous case. i.e. the
  intersection of any two paths intersects with both the set
  differences of the other two. First let us consider $Q \cap (Q_1 \setminus Q_2) \ne
  \O$ and $Q \cap (Q_1 \setminus Q_2) \ne \O$, where $Q = Q_3 \cap
  Q_4$. Since $Q_1$ and
  $Q_2$ share a leaf, there is exactly one vertex at which they branch
  off from the path $Q_1 \cap Q_2$ into two paths $Q_1 \setminus Q_2$
  and $Q_2 \setminus Q_1$. Let this vertex be $v$. It is clear that if
  path $Q_3 \cap Q_4$, must intersect with paths $Q_1 \setminus Q_2$
  and $Q_2 \setminus Q_1$, it must contain $v$ since these are paths
  from a tree. Moreover, $Q_3 \cap Q_4$ intersects with $Q_1 \cap
  Q_2$ at exactly $v$ and only at $v$ which means that $Q_1 \cap Q_2$
  does not intersect with $Q_3 \setminus Q_4$ or $Q_4 \setminus Q_3$
  which contradicts initial condition of this case. Thus this
  case cannot occur and case 1 is the only possible scenario. \\
  Thus lemma is proven \qed
\end{proof}


\noindent
Following are two algorithms which will be used later to prove a
property of the ICPPA.
First we present and then prove the
  correctness of Algorithm \ref{perms}.

\begin{algorithm}[h]
\caption{Permutations from an ICPPA $\{(S_i,P_i) | i \in I\}$}
\label{perms}
\begin{algorithmic}
\STATE Let $\Pi_0=\{(S_i,P_i)| i \in I\}$\\
\STATE $j = 1$;
\label{shareleaf} \WHILE {There is $(P_1,Q_1), (P_2,Q_2) \in \Pi_{j-1}$ with $Q_1$ and
  $Q_2$ having a common leaf}
\STATE $\Pi_j=   \Pi_{j-1} \setminus \{(P_1,Q_1),(P_2,Q_2)\}$;
\label{setbreak}\STATE $\Pi_j = \Pi_j \cup \{(P_1 \cap P_2,Q_1 \cap Q_2), (P_1 \setminus P_2,Q_1 \setminus Q_2), (P_2 \setminus P_1, Q_2 \setminus Q_1)\}$;
\STATE $j = j+1$;
\ENDWHILE
\STATE $\Pi = \Pi_j$;
\STATE Return $\Pi$;
\end{algorithmic}
\end{algorithm}

\begin{lemma}
\label{lem:invar1}
  In Algorithm \ref{perms}, at the end of $j$th iteration, $j \ge 0$, of the while loop of 
  Algorithm \ref{perms}, the following invariants are maintained.
\begin{itemize}
\item {\em Invariant I:} $Q$ is a path in $T$ for each $(P,Q) \in
  \Pi_j$
\item {\em Invariant II:} $|P|=|Q|$ for each $(P,Q) \in \Pi_j$
\item {\em Invariant III:} For any two $(P,Q), (P',Q') \in \Pi_j$,
  $|P' \cap P''|=|Q' \cap Q''|$. 
\item {\em Invariant IV:} For any three, $(P',Q'), (P'',Q''),
  (P, Q) \in \Pi_j$,
  $|P' \cap P'' \cap P|=|Q' \cap Q'' \cap Q|$.
\end{itemize}
\end{lemma}
\begin{proof}
  Proof is by induction on the number of iterations, $j$. In the rest
  of the proof, the term ``new sets'' will refer to the new sets added
  in $j$th iteration as defined in line \ref{setbreak} of Algorithm
  \ref{perms}, i.e. the following three assignment pairs for
  some $(P_1,Q_1), (P_2,Q_2) \in \Pi_{j-1}$ where $Q_1$ and $Q_2$
  intersect and share a leaf: $(P_1 \cap P_2, Q_1 \cap Q_2)$, or $(P_1
  \setminus P_2, Q_1 \setminus Q_2)$, or $(P_2 \setminus P_1, Q_2
  \setminus Q_1)$.\\
  \noindent
  The base
  case, $\Pi_0 = \{(S_i,P_i) \mid i \in [m]\}$, is trivially true since it
  is the input which is an ICPPA.  Assume the lemma is true till the $j-1$
  iteration. Consider $j$th
  iteration:\\
  \noindent
  If $(P,Q)$, $(P',Q')$ and $(P'',Q'')$ are in $\Pi_{j}$ and
  $\Pi_{j-1}$, all the invariants are
  clearly true because they are from $j-1$ iteration.\\
  If $(P,Q)$ is in $\Pi_{j}$ and not in $\Pi_{j-1}$, then it must be
  one of the new sets added in $\Pi_j$. Since $(P_1,Q_1)$ and
  $(P_2,Q_2)$ are from $\Pi_{j-1}$ and $Q_1,Q_2$ intersect and have a
  common leaf, it can be verified that the
  new sets are also paths. \\
  By hypothesis for invariant III, invariant II also holds for $(P,Q)$
  no matter which new set in $\Pi_j$ it
  is.\\
  To prove invariant III, if $(P,Q)$ and $(P',Q')$ are not in
  $\Pi_{j-1}$, then they are both new sets and invariant III holds
  trivially (new sets are disjoint). Next consider $(P,Q), (P',Q') \in
  \Pi_j$ with only one of them, say $(P',Q')$, in $\Pi_{j-1}$. Then
  $(P,Q)$ is one of the new sets added in line \ref{setbreak}. It is
  easy to see that if $(P,Q)$ is $(P_1 \cap P_2, Q_1 \cap Q_2)$, then
  due to invariant IV in hypothesis, invariant III becomes true in
  this iteration. Similarly, using lemma \ref{lem:setminuscard}
  invariant III is proven if $(P, Q)$ is $(P_1 \setminus P_2, Q_1
  \setminus Q_2)$, or $(P_2 \setminus P_1, Q_2
  \setminus Q_1)$.\\
  To prove invariant IV, consider three assignments
  $(P,Q),(P',Q'),(P'',Q'')$. If at least two of these pairs are in not
  $\Pi_{j-1}$, then they are any two of the new sets. Note that these
  new sets are disjoint and hence if $(P',Q'), (P'',Q'')$ are any of
  these sets, $|P \cap P' \cap P''|=|Q \cap Q' \cap Q''|=0$ and
  invariant IV is true. Now we consider the case if at most one of
  $(P,Q),(P',Q'),(P'',Q'')$ is not in $\Pi_{j-1}$. If none of them are
  not in $\Pi_{j-1}$ (i.e. all of them are in $\Pi_{j-1}$), invariant
  IV is clearly true. Consider the case where exactly one of them is
  not in $\Pi_{j-1}$. w.l.o.g let that be $(P,Q)$ and it could be any
  one of the new sets. If $(P,Q)$ is $(P_1 \cap P_2, Q_1 \cap Q_2)$,
  from lemma \ref{lem:fourpaths} and invariant III hypothesis,
  invariant IV is proven. Similarly if $(P,Q)$ is any of the other new
  sets, invariant IV is proven by also using lemma
  \ref{lem:setminuscard}. \qed

\end{proof}

\noindent
It can be observed that the output of algorithm \ref{perms} is such
that every leaf is incident on at most a single path in the new set of
assignments. This is due to the loop condition at line
\ref{shareleaf}. Let $v_1$ be the leaf incident on path $P_i$. Assign
to it any one element from $S_i \setminus \bigcup_{i \ne j}
S_j$. Remove $(S_i, P_i)$ from assignments and add $(\{x_1\},
\{v_1\}), (S_i \setminus \{x_1\}, P_i \setminus \{v_1\})$. Now all
assignments except single leaf assignments are paths from the subtree
$T_0 = T \setminus \{v \mid v \text{ is a leaf in } T\}$.

% exclusively on the spine of the caterpillar. Consider the spine to be an
% interval and run ICPIA algorithm on it to get the required
% permutation. Algorithm \ref{leafasgn} makes this clear.

\begin{algorithm}[h]
\caption{Leaf assignments from an ICPPA $\{(S_i,P_i) | i \in I\}$}
\label{leafasgn}
\begin{algorithmic}
\STATE Let $\Pi_0=\{(S_i,P_i)| i \in [m]\}$. Paths are such that no
two paths $P_i, P_j, i \ne j$ share a leaf.\\
\STATE $j = 1$\\
\WHILE {there is a leaf $v$ and a unique $(S_{i_1}, P_{i_1})$ such that $v \in P_{i_1}$}
\STATE $\Pi_j=   \Pi_{j-1} \setminus \{(S_{i_1}, P_{i_1})\}$\\
\STATE $X = S_{i_1} \setminus \bigcup_{i \ne i_1, i \in
  I}S_i$
\IF{$X$ is empty} 
\STATE exit 
\ENDIF;
\STATE Let $x = $ arbitrary element from $X$ \vspace{2mm}

\STATE $\Pi_j = \Pi_j \cup \{(S_{i_1} \setminus \{x\}),(P_{i_1} \setminus
  \{v\}), (\{x\},\{v\})\}$\\
\STATE $j = j+1$\\
\ENDWHILE
\STATE $\Pi = \Pi_j$\\
\STATE Return $\Pi$\\
\end{algorithmic}
\end{algorithm}

\begin{lemma}
\label{lem:invar3}
  In Algorithm \ref{leafasgn}, at the end of the $j$th iteration, $j \geq 0$, of the while loop of
  Algorithm \ref{leafasgn}, the invariants given in lemma \ref{lem:invar1}
  hold. Moreover, $X$ as defined in the algorithm
  is non empty if this is an ICPPA.
\end{lemma}
\begin{proof}
  First we see that $X = S_{i_1} \setminus \bigcup_{i \ne i_1, i \in
    I}S_i$ is non empty in every iteration for an ICPPA. Suppose $X$
  is empty. We know that $v \in P_{i_1} \setminus \bigcup_{i \ne i_1, i \in
    I}P_i$ since $v$ is in the unique path $P_{i_1}$. Since this is an
  ICPPA $|S_{i_1}| = |P_{i_1}|$. For any $x \in S_{i_1}$ it is
  contained in at least two sets due to our assumption. Let $S_{i_2}$ be a second set that
  contains $x$. We know $v \notin P_{i_2}$. Therefore there cannot
  exist a permutation that maps elements of $S_{i_2}$ to
  $P_{i_2}$. This contradicts our assumption that this is an ICPPA. Thus $X$ cannot be empty.


  We use mathematical induction on the number of iterations for this
  proof. The term ``new sets'' will refer to the sets added in $\Pi_j$
  in the $j$th iteration, i.e. $(P' \setminus
  \{x\},Q' \setminus \{v\})$ and $(\{x\},\{v\})$ for some $(P',Q')$
  in $\Pi_{j-1}$ such that $v$ is a leaf and $Q'$ is the unique path
  incident on it.\\
  For $\Pi_0$ all invariants hold because it is output from
  algorithm \ref{perms} which is an ICPPA. Hence base case is proved.  Assume the
  lemma holds for $\Pi_{j-1}$. Consider $\Pi_j$ and any $(P,Q) \in
  \Pi_j$. If $(P,Q) $ is in $ \Pi_j$ and $\Pi_{j-1}$ invariants I and II are
  true because of induction assumption. If it is only in $\Pi_j$, then it is $\{(P' \setminus
  \{x\}),(Q' \setminus \{v\})$ or $(\{x\},\{v\})$ for some $(P',Q')$
  in $\Pi_{j-1}$. By definition, $x$ is an element in $P'$ (as defined in the
  algorithm) and $v$ is a leaf in $Q'$. If $(P,Q)$ is $\{(P' \setminus \{x\}),(Q' \setminus \{v\})$,
  $Q$ is a path since only a leaf is removed from path $Q'$. We know $|P'| = |Q'|$,
  therefore $|P' \setminus \{x\}| = |Q' \setminus \{v\}|$. Hence in this case invariants
  I and II are obvious. It is easy to see these invariants hold if $(P,Q)$ is $(\{x\},\{v\})$.


  For invariant III consider $(P_1,Q_1),(P_2,Q_2)$ in $\Pi_j$. If both
  of them are also in $\Pi_{j-1}$, claim is proved. If one of them is
  not in $\Pi_{j-1}$ then it has to be $\{(P' \setminus
  \{x\}),(Q' \setminus \{v\})$ or $(\{x\},\{v\})$ for some $(P',Q')$
  in $\Pi_{j-1}$. Since by definition, $Q'$ is the only path with $v$ and $P'$ the
  only set with $x$ in the
  previous iteration, $|P_1 \cap (P' \setminus \{x\})| = |P_1 \cap P'|$
  and $|Q_1 \cap (Q' \setminus \{v\})| = |Q_1 \cap Q'|$ and $|P_1 \cap
  \{x\}| = 0, Q_1 \cap \{v\} = 0$. Thus invariant III is also proven.

  \noindent
  To prove invariant IV, consider $(P_1,Q_1),(P_2,Q_2), (P_3,Q_3)$ in
  $\Pi_j$. If exactly one of them, say $P_3 \notin \Pi_{j-1}$, it is one of the new sets. By the same argument used to
  prove invariant III, $|P_1 \cap P_2 \cap (P'
  \setminus \{x\})| = |P_1 \cap P_2 \cap P'|$ 
  and $|Q_1 \cap Q_2 \cap (Q' \setminus \{x\})| = |Q_1 \cap Q_2 \cap
  Q'|$. Since $P_1, P_2, P'$ are all in $\Pi_{j-1}$, by induction hypothesis
  $|P_1 \cap P_2 \cap P'| = |Q_1 \cap Q_2 \cap Q'|$. Also $|P_1 \cap
  P_2 \cap \{x\}| = 0, Q_1 \cap Q_2 \cap \{v\} = 0$. 
  If two or more of them are not in $\Pi_{j-1}$, then it can
  be verified that $|P_1
  \cap P_2 \cap P_3| = |Q_1 \cap Q_2 \cap Q_3|$ since the new sets
  in $\Pi_j$ are either disjoint or as follows: assuming $P_1, P_2
  \notin \Pi_{j-1}$ and new sets are derived from $(P', Q'), (P'', Q'') \in
  \Pi_{j-1}$ with $x_1, x_2$
  exclusively in $P_1, P_2$, $(\{x_1\},\{v_1\}), (\{x_2\},\{v_2\})
  \in \Pi_j $ thus $v_1, v_2$ are exclusively in $Q_1, Q_2$ resp. it
  follows that
$|P_1
  \cap P_2 \cap P_3| = |(P' \setminus \{x_1\}) \cap (P'' \setminus
  \{x_2\}) \cap P_3| = |P' \cap P'' \cap P_3| = |Q' \cap Q'' \cap Q_3|
  = |(Q' \setminus \{v_1\} \cap Q'' \setminus \{v_2\} \cap Q_3| =
  |Q_1 \cap Q_2 \cap Q_3|$. Thus invariant
  IV is also proven.
  \qed
\end{proof}

\noindent
Using algorithms \ref{perms} and \ref{leafasgn} we prove the following
theorem.

\begin{theorem}
\label{th:perm}
  If $\cA$ is an ICPPA, then there exists a bijection $\sigma : U
\rightarrow V(T)$ such that $\sigma(S_i) = P_i$ for all $i \in I$
\end{theorem}
\begin{proof}
This is a contructive proof. First, the given ICPPA $\cA$ and tree $T$ are given as input to Algorithm
\ref{perms}. This yields a ``filtered'' ICPPA as the output which is
input to Algorithm \ref{leafasgn}.
It can be observed that the output of Algorithm \ref{leafasgn} is a set of interval
assignments to sets and one-to-one assignment of elements of $U$ to
each leaf of $T$. To be precise, it would be of the form $\cB_0 =
\cA_0 \cup \cL_0$. The leaf assignments are defined in $\cL_0
= \{ (x_i,v_i) \mid x_i \in U, v_i \in T, x_i \ne x_j, v_i \ne v_j, i \ne j, i,j \in [k] \}$ where $k$ is the
number of leaves in $T$. The path assignments are defined in $\cA_0
\subseteq \{(S_i',P_i') \mid S_i' \subseteq U_0, P_i' \text{ is a path
  from } T_0\}$ where $T_0$ is the tree obtained by removing all the
leaves in $T$ and $U_0 = U \setminus \{ x \mid x \text{ is assigned to
  a leaf in }\cL_0 \}$. Now we have a subproblem of finding the
permutation for the path assignment $\cA_0$ which has paths from tree
$T_0$ and sets from universe $U_0$. Now we repeat the procedure and the path assignment $\cA_0$ and tree $T_0$
is given as input to Algorithm \ref{perms}. The output of this
algorithm is given to Algorithm \ref{leafasgn} to get a new
union of path and leaf assignments $\cB_1 =
\cA_1 \cup \cL_1$ defined similar to $\cB_0, \cL_0, \cA_0$. In
general, the two algorithms are run on
path assignment $\cA_{i-1}$ with paths from tree $T_{i-1}$ to get a new
subproblem with path assignment $\cA_i$ and tree $T_{i}$. $T_i$ is
the subtree of $T_{i-1}$ obtained by removing all its leaves. More importantly, it gives leaf
assignments $\cL_{i}$ to the leaves in tree $T_{i-1}$. This is
continued until we get a subproblem with path assignment $\cA_{d-1}$ and
tree $T_{d-1}$ for some $d \le n$ which is just a
path. From the last lemma we know that $\cA_{d-1}$ is an
ICPPA. Another observation is that an ICPPA with all its tree paths
being intervals (subpaths from a path) is nothing but an ICPIA\cite{nsnrs09}.
Let $\cA_{d-1}$ be equal to $\{(S_i'',P_i'') \mid S_i'' \subseteq U_{d-1}, P_i'' \text{ is a path
  from } T_{d-1} \}$. It is true that the paths $P_i''$s
may not be precisely an interval in the sense of consecutive integers
because they are some nodes from a tree. However, it is easy to see that
the nodes of $T_{d-1}$ can be ordered from left to right and ranked to get
intervals $I_i$ for every path $P_i''$ as follows. $I_i = \{[l,r]
\mid l = \text{ the lowest rank of the nodes in }P_i'', r = l+|P_i''|-1
\}$. Let asssignment $\cA_d$ be with the renamed paths. $\cA_d = \{ (S_i'', I_i) \mid (S_i'', P_i'') \in \cA_{d-1}
\}$. What has been effectively done is renaming the nodes in $T_{d-1}$
to get a tree $T_d$.
The ICPIA $\cA_d$ is now in the format that the ICPIA algorithm
requires which gives us the permutation $\sigma' : U_{d-1} \rightarrow T_{d-1}$

\noindent
$\sigma'$ along with all the leaf assignments $\cL_i$
gives us the permutation for the original path assignment $\cA$.
More precisely, the permutation for tree path assignment $\cA$ is defined as
follows. $\sigma: U \rightarrow T$ such that the following
is maintained.
\begin{align*}
 \sigma(x) &= \sigma'(x),   \text{ if } x \in U_{d-1} \\
           &= \cL_i(x),     \text{ where $x$ is assigned to a leaf in a
             subproblem $\cA_{i-1}, T_{i-1}$}
\end{align*}

\noindent
To summarize, run algorithm \ref{perms} and
\ref{leafasgn} on $T$. After the leaves have been assigned to specific
elements from $U$, remove all leaves from $T$ to get new tree
$T_0$. The leaf assignments are in $\cL_0$. Since only leaves were removed $T_0$ is indeed a tree. Repeat
the algorithms on $T_0$ to get leaf assignments $\cL_{1}$. Remove the
leaves in $T_0$ to get $T_1$ and so on until the pruned tree $T_d$
is a single path. Now run ICPIA algorithm on $T_d$ to get
permutation $\sigma'$. The relation $\cL_0 \cup \cL_1 \cup .. \cup
\cL_{d} \cup \sigma'$ gives the bijection required in the original problem.\qed
\end{proof}

\section{Finding an assignment of tree paths to a set system}
A set system can be concisely represented by a binary matrix where the
row indices denote the universe of the set system and the column
indices denote each of the sets. Let the binary matrix be $M$ with
order $n \times m$, the set system be $\cF = \{S_i \mid i \in [m]\}$,
universe of set system $U = \{x_1, \dots ,x_n\}$. If $M$ represents $\cF$, $|U| = n, |\cF| =
m$. Thus $(i,j)$th element of $M$, $M_{ij} = 1$ iff $x_i \in S_j$. If $\cF$ has a feasible tree path assignment (ICPPA) $\cA =
\{(S_i,P_i) \mid i \in [m]\}$, then we
say its corresponding matrix $M$ has an ICPPA. Conversly
we say that a matrix $M$ has an ICPPA if there exists an ICPPA $\cA$ as defined
above.\\
\noindent
Consider the strict intersection graph or overlap graph of $\cF$. It
is constructed with its vertices denoting a unique subset in the set
system and an edge is present between vertices of two sets iff the
corresponding sets have a nonempty intersection and none is contained
in the other. Formally, intersection graph is $G_f = (V_f, E_f)$ such
that $V_f = \{v_i \mid S_i \in \cF\}$ and $E_f = \{(v_i, v_j) \mid S_i
\cap S_j \ne \O \text{ and }S_i \nsubseteq S_j, S_j \nsubseteq S_i
\}$.  We use this graph to decompose $M$ as described in \cite{wlh02,nsnrs09}.  
A prime sub-matrix of $M$ is defined as the
matrix formed by a set of columns of $M$ which correspond to a
connected component of the graph $G_f$.  Let us denote the prime
sub-matrices by $M_1,\ldots,M_p$ each corresponding to one of the $p$
components of $G_f$. Clearly, two distinct matrices have
a distinct set of columns.  Let $col(M_i)$ be the set of columns in
the sub-matrix $M_i$.  The support
of a prime sub-matrix $M_i$ is defined as $supp(M_i) = \displaystyle \bigcup_{j \in
  col(M_i)}S_j$. Note that for each $i$, $supp(M_i) \subseteq
U$.  For a set of prime sub-matrices $X$ we define
$supp(X) = \displaystyle \bigcup_{M \in X} supp(M)$. \\


\noindent
Now we define a partial order on the prime submatrices. Consider the relation
$\preccurlyeq$ on the prime sub-matrices $M_1, \ldots, M_p$ defined as
follows: 
\begin{equation} 
\nonumber \{(M_i,M_j) | \mbox{ A set } S \in
  M_i \mbox{ is contained in a set } S' \in M_j\} \cup \{(M_i,M_i) | 1
  \leq i \leq p\} 
\end{equation}

This partial order is the same as that defined in \cite{nsnrs09}. The prime submatrices and the above partial order can be defined for
any set system. We will use this theory of prime submatrices to find
an ICPPA for a set system $\cF$. Recall the following lemmas and theorems.

\begin{lemma}[Lemma 3, \cite{nsnrs09}] \label{lem:containment}
Let $(M_i,M_j) \in \preccurlyeq$.  Then there is a set $S' \in M_j$ such that for each $S \in M_i$, $S \subseteq S'$. 
\end{lemma}
\begin{lemma}[Lemma 4, \cite{nsnrs09}]
For each pair of prime sub-matrices, either $(M_i,M_j) \not\in \preccurlyeq$ or $(M_j,M_i) \not\in \preccurlyeq$.
%If $(M_i,M_j) \in \preccurlyeq$ and $(M_j,M_i) \in \preccurlyeq$, then $i = j$ and $|M_i| = 1$.
\end{lemma}
\begin{lemma}[Lemma 5, \cite{nsnrs09}]
If $(M_i,M_j) \in \preccurlyeq $ and $(M_j,M_k) \in \preccurlyeq$, then $(M_i,M_k) \in \preccurlyeq$.
\end{lemma}
\begin{lemma}[Lemma 6, \cite{nsnrs09}]
If $(M_i,M_j) \in \preccurlyeq$ and $(M_i,M_k) \in \preccurlyeq$, then
either $(M_j,M_k) \in \preccurlyeq$ or $(M_k,M_j) \in \preccurlyeq$. 
\end{lemma}
\begin{theorem}[Theorem 4, \cite{nsnrs09}] \label{thm:partitionold}
  $\preccurlyeq$ is a partial order on the set of prime sub-matrices
  of $M$.  Further, it uniquely partitions the prime sub-matrices of
  $M$ such that on each set in the partition $\preccurlyeq$ induces a
  total order.
\end{theorem}



Now we define another relation $\ll$ on the set of prime
submatrices as follows:\\
\begin{align*}
  \ll = \{ (M_i,M_j) \mid \nexists M_k s.t. M_i \preccurlyeq M_k, M_k \preccurlyeq M_j
  \} \cup \{ (M_i,M_i), i \in [p] \}
\end{align*}

\begin{lemma}
  Relation $\ll$ is a partial order
\end{lemma}
\begin{proof}
Reflexive by definition. Antisymmetric and transitive because
$\preccurlyeq$ is antisymmetric and transitive. 
\end{proof}

- Hasse diagram of $\ll$ is an intree.


\begin{theorem} \label{thm:partition}
Relation $\ll$ is a partial order on the set of prime sub-matrices of $M$.
Further, it uniquely partitions the prime sub-matrices of $M$ such
that on each set in the partition $\ll$ induces intrees. and the
intrees partition into total orders.
\end{theorem}
\begin{proof}
  Relation $\ll$ is relective, antisymmetric and transitive because so
  is $\preccurlyeq$.  Consider the Hasse diagram of $\ll$. By
  definition mubs have outdegree
  0 \\
  Claim 1: every other node has out degree 1.\\
  we wil prove this by induction on the level of the tree. level 1 has
  this true since root(mub) is the only possible parent. assume this
  is true for all nodes till level k-1 . Consider level k. assume
  there is a node, $M_i$ in level k that has out degree 2 going into
  two nodes say. $M_{i1}$ and $M_{i2}$. i.e. $(M_i, M_{i1}$ and $(M_i,
  M_{i2}$. We also know that $(M_{i1}, M_p) \in \ll$. Therefore by
  transitivity, $ (M_i, M_p) \in \ll$. Then $M_i$ should have been
  related in $\ll$ to $M_{i}$ instead of $M_{i1}$. This is a
  contradiction to the assumption above.
%Hence it is acyclic as well since there are no nodes with 

Claim 2: Every node other than the root has a path to the root
proof: tbd.


  % Construct the Hasse diagram of $\preccurlyeq$. Let it be $\cD$. We will layer
%   the submatrices in a particular manner described below. Note that
%   this graph is a dag.  Let the set of matrices $\cE_1 = \{M \mid M
%   \text{ is a maximal upper
%     bound in } \cD \} $. This is layer 1.\\
%   The matrices in layer 2, $\cE_2$ are those that are connected to a
%   matrices in $\cE_1$ by only paths of l
% ength at most 1, $\cE_3$ are those that are connected to a
%   matrices in $\cE_2$ by only paths of length at most 1 and so on. In other
%   words, given a matrix $M$, set of matrices $\cE$, we define
%   functions $N'$ and $N$, $N'(M) = \{ M' \mid M' \preccurlyeq M,
%   \nexists M'' s.t. , M' \preccurlyeq M'', M'' \preccurlyeq M, M'' \ne
%   M, M'' \ne M' \}$,
%   $N(\cE) = \bigcup_{M \in \cE} N'(M)$. Now we have
% $\cE_1 = \{M \mid M \text{ is a maximal upper
%   bound in } \cD \}$, 
% $\cE_2 = N(\cE_1), \cE_3 = N(\cE_2)$ and so on as
% layers of the Hasse graph.\\

% Construct a subgraph from the hasse graph with only edges that
% belong to the longest path between corresponding mub and mlb. \\
% claim: this is a collection of in-trees. one intree per mub.\\
claim: matrices in each level can be partitioned s.t.  each partition
set is a total order.  - suggested construction: matrices of the
level$(level i)$ with common parent
$(level i-1)$ is one partition.\\
claim: these created partitions are disjoint
disjoint. there is total order in the partition.\\

\end{proof}

From Theorem \ref{thm:partition} we can see that the set of prime
submatrices can be partitioned into sets of matrices that form in-trees.\\


Let $X$ be an in-tree and $T$ be a tree. It is said $X$ has ICPPA on
$T$ iff the prime matrix $mub(X)$ has an ICPPA with some subtree $T_1
\subseteq T$ and prime
matrices $X \setminus mub(X)$ have ICPIA.

\begin{lemma}
\label{lem:subicppa}
A matrix $M$ has an ICPPA iff there exists a tree $T$ that has
subtrees $T_1, T_2, \dots T_r \in T, T_i \cap T_j \ne \O, i \ne j, i,j
\in [r]$ such that, intrees $X_1, X_2, \dots X_r$ have an ICPPA in
trees $T_1, T_2, \dots T_r$. $X_1, X_2, \dots X_r$ represent
in-trees of prime submatrices of $M$ due to the partial order
$\ll$.
\end{lemma}

\begin{proof}
  First we prove that if a matrix $M$ has an ICPPA, then there exists
  tree $T$ as described in lemma statement. If $M$ has an ICPPA, then
  there exists a tree $T$ with an assignment $\cA$ of paths from $T$
  to sets in $M$. From Theorem \ref{thm:partition} we know that $M$
  can be decomposed into a partition of prime submatrices $X_1, X_2,
  \dots X_r$.  Let $M_{Xi} = mub(X_i)$. The assignment of sets in
  $M_{Xi}$ are paths by definition and being a prime matrices these sets
  strictly overlap. Thus it is clear that the vertices $\{\cA(x) \mid
  x \in supp(mub(X_i))$ \} for any $i \in [r]$ induce a subtree in
  $T$, let us call it $T_i$. Since $X_i \cap X_j = \O, i \ne j$ and
  $\cA$ is a bijection\footnote{sloppy tbd}, $T_i \cap T_j = \O, i \ne
  j$. Thus
  claim proven.\\

\noindent
Now we need to prove the if condition of lemma. 
We prove by construction of an algorithm to find the ICPPA for a given
matrix $M$ and a tree $T$. 


TO PROVE:
IF
there exists a tree $T$ which has
subtrees $T_1, T_2, \dots T_r \in T, T_i \cap T_j \ne \O, i \ne j, i,j
\in [r]$ such that, the intrees rooted at $mub(X_1), mub(X_2), \dots
mub(X_r)$ have an ICPPA from $T_1, T_2, \dots T_r$ resp. $X_1, X_2,
\dots X_r$ are the partition sets of prime submatrices of $M$ due to the partial
order $\preccurlyeq$. $mub(X)$ denotes the maximal upper bound of a poset
X.
THEN
then sets in M have path assignments that preserve ICPPA conditions

Since $mub(X_i)$ has ICPPA from $T_i$, every set in the matrix mub has a
path in $T_i$ that form an ICPPA. By definition of mub and $X_i$, all other sets
in partition $X_i$ are descendants of $X_i$. there is a unique path from any
node mi in hasse of $X_i$ to mub($X_i$). The matrix adjacent to mub($X_i$) on
this path let it be $M_{xi1}$, and path $P$. $M_{xi1}$ is completely contained in
a set in $mub(X_i)$, say $S_{xi1}$. $P_{xil}$ is a path in $T_i$ assigned to
$S_{xil}$. Effectively $P{xil}$ is an interval. Thus if $M_{xi2}$ is matrix adjacent
to $M_{xi1}$ in the path $p$, to get path assignments to $M_{xi2}$ from $P_{xil}$ is an
ICPIA problem. Similarly $supp(Mxi2) \subseteq S_{xi1}$ a set in
$M_{xi1}$. $P_{xil}'$ is again a path and path assignments to $M_{xi3}$, next in the
path $P$, from $P_{xil}'$ is an ICPIA problem and so on till the mlbs.

to prove: all sets have paths assigned to them. 


  
\end{proof}






The
 following is the outline of the algorithm. Let $\cF$
be the given set system and $T$ be the given tree.  Consider one of
the partitions $X_i$ that is induced by $\preccurlyeq$. By theorem
\ref{thm:partition}, we know that the submatrices in $X_i$ can be
ordered in the following manner- $M_{ik} \preccurlyeq ... \preccurlyeq
M_{i2} \preccurlyeq M_{i1}$ where $\{i_1, i_2, .. i_k\} \subseteq [p]$
and $p$ is the number of prime submatrices, $M_{i_j} \in X_i, j \in
[k]$ . From the definition of $\preccurlyeq$, for each $r$, $2 \leq r
\leq k$, $supp(M_{ir})$ is contained in at least one set in
$M_{i(r-1)}$.  Therefore, it follows that $supp(X_i) = supp(M_{i1})$.
We find a subtree $T_i \subseteq T$, such that $supp(M_{i1}) = |T_i|$,
and associate it with $M_{i1}$. Now we assign paths from $T_i$ to the
sets $col(M_{i1})$ maintaining the ICPPA condition. If we are unable to do so,
we try the next subtree of size $|supp(M_{i1})|$ and so on. Suppose paths
have been successfully assigned to the sets in $M_{i1}$. We know that
$supp(M_{i2}) $ is contained in at least one column/set $S \in
col(M_{i1})$. 
Thus the subtree associated with $supp(M_{i2}) $ is
contained in the path associated with $S$.
Now this becomes an interval assignment problem and ICPIA
algorithms can be used to assign paths to all $S' \in
M_{i2}$. Similarly ICPIA can be used to assign paths to $col(M_{i_j})$
for all $2 \le j \le k$.

For simplicity, we consider $T$ to be a rooted tree with an arbitrary
node as its root.

\begin{algorithm}[h]
\caption{Algorithm to find an ICPPA for a matrix $M$ on tree $T$: $main\_ICPPA(M, T$)}
\label{al:icppa-main}
\begin{algorithmic}
\STATE Identify the prime sub-matrices. This is done by constructing the strict overlap graph and identify connected components.  Each connected component yields a prime sub-matrix.   \\
\STATE Construct the partial order $\preccurlyeq$ on the set of prime sub-matrices.  \\
\STATE Construct the partition $X_1,\ldots,X_l$ of the  prime
sub-matrices induced by $\preccurlyeq$ \\
\STATE  Construct the total order on each set in the partition. Let $M_{k1}, M_{k2}, ... , M_{k{j_k}}$ be the total order of
partition $X_k$ in reverse order. \\

$\cA_M = \O$ 
\FOR {$(k=1; k \leq l; k++)$} % For every partition
\STATE Assign some subtree $T' \subseteq T$ to $M_{k1}$ s.t. $|T'| =
supp(M_{k1})$ \\
\STATE Assign paths from $T'$ to sets in $M_{k1}$. Let this assignment
be $\cA_{k1} = \{(S_{ki},P_{ki}) | S_{ki} \text{ is a set in }M_{k1}, P_{ki} \text{ is a
  path from }T'\}$\\
$\cA_k = \cA_{k1}$\\

\FOR{ $2 \leq r \leq j_k$}
\IF {parent of $M_{kr}$ in total order is $M_{k1}$}
\STATE Find the set in $M_{k1}$ that contains $supp(M_{kr})$. Let it
be $S_{kr}$
 \\
\STATE $\cA_{kr} = ICPPA(M_{kr},P_{kr},k,r)$\\
\IF{$\cA_{kr}$ is invalid}
\STATE  exit
\ENDIF
\STATE $\cA_k = \cA_{k} \cup \cA_{kr}$
\ENDIF
\ENDFOR
\STATE $\cA_M = \cA_M \cup \cA_{kr}$
\ENDFOR\\
return $\cA_M$
\end{algorithmic}
  
\end{algorithm}



%W.l.o.g assume $T$ is an ordered tree with an arbitrary vertex as the root.

\begin{algorithm}[h]
\caption{Recursive algorithm ICPPA($M_{km}, T_x, k, m$), $M$ is the matrix which to which
  paths from $T_x$ must be assigned. $k$ is the partition being
  processed and $m$ is index of the submatrix in the total order for
  $M_{km}$. For simplicity we refer to $M_{km}$ as $M_x$ below}. 
\label{al:icppa-rec}

\begin{algorithmic}
% Handle recursion end case
\IF{$|col(M_x) = 1|$ and $T_x$ is a path}
\STATE return $\{(S, T_x)\}$ where $S$ is the only set in $M_x$\\
\ELSE
\IF{$|col(M_x) = 1|$}
\STATE Report failure and return
\ENDIF
\ENDIF

% Process M_x and its children
\STATE Rank the vertices of $T_x$ to get interval $I_x$.\\
\STATE $\cA' = $ ICPIA ($M_x, I_x$)\\
\STATE Convert intervals back to tree paths. $P_i = \{ v \mid rank(v)
= r, r \in I_i\} \text{ for every interval } I_i \text{ assigned in } \cA' \}$\\
\STATE $\cA = \{(S_i,P_i) \mid \text{ for all } S_i \text{ represented in }M_x\}$\\
\STATE $\cA_f = \cA$\\
%\STATE In the total order of $X_k$, let $M_{k}$
\FOR{$s = m+1 $ to $j_k$}
\IF{$M_x$ is the parent of $M_{ks}$}
\STATE  Find $S_j$ in $M_x$ that contains $supp(M_{ks})$\\
\STATE $\cA_s = ICPPA(M_{ks}, P_j, k, s)$
\STATE $\cA_f = \cA_f \cup \cA_f$
\ENDIF
\ENDFOR \\
return $\cA_f$
\end{algorithmic}  
\end{algorithm}

The ranking in algorithm \ref{al:icppa-rec} of the tree path $T_{x}$
is as follows. Assume one of the ends of path as the left vertex and the other as
right vertex. Rank the vertices of $T_x$ from left to right to get $I_x =
\{r \mid r = rank(v), v \in T_x\}$. i.e. $I_x = [1, |T_x|]$


\section{COP in logspace}

\noindent
% \cite{nsnrs09} introduced a simple set cardinality based
% characterization called ICPIA (Intersection Cardinality Preservation
% Interval Assignment). 
% This characterization uses the fact that COT on
% a binary matrix is equivalent to the problem of testing if feasible
% intervals can be assigned to a set system. In other words, given a set
% system containing subsets of a universe, check if we can assign
% intervals to them such that there exists a bijection from the universe
% of the set system to the intervals' universe, which maps every subset
% to an interval. A simpler COT algorithm is given using ICPIA to find
% the COP ordering for a binary matrix if there exists one. The
% algorithm uses the set system version of the COP problem to solve it.
% It first finds a set $S$ such that any two sets that intersect with
% $S$ also intersect with each other and assigns an arbitrary interval
% (say leftmost) to it. Then it attempts to assign an interval to any
% set $S'$ that intersects with $S$ preserving the pairwise intersection
% cardinality (ICPIA condition). This continues such that at every
% iteration the algorithm selects a set that intersects with already
% interval-assigned sets for the next assignment. This new set is
% assigned an interval making sure ICPIA conditions are still met. If at
% any point no interval assignment is possible that meets ICPIA
% condition, the matrix corresponding to the set system has no COP.
% This is a polynomial time algorithm
\subsection{Inverval hypergraphs}
\cite{kklv10} showed that interval graphs isomorphism can be done in
logspace. Their paper proves that a canon for interval graphs can be
calculated in logspace using an interval hypergraph representation of the
interval graph with each hyperedge being a set to which an interval shall be
assigned by the canonization algorithm. An overlap graph (subgraph of
intersection graph, edges define only strict intersections and no
containment) of the hyperedges of the hypergraph is created and canons
are computed for each overlap component. The overlap components define
a tree like relation due to the fact that two overlap components are
such that either all the hyperedges of one is disjoint from all in the other,
or all of them are contained in one hyperedge in the other. This is
similar to the containment tree defined in \cite{nsnrs09}. Finally the
canon for the whole graph is created using logspace tree canonization
algorithm from \cite{sl92}. The interval labelling done in this
process of canonization is exactly the same as the problem of
assigning feasible intervals to a set system, and thus the problem of
finding a COP ordering in a binary matrix\cite{nsnrs09}.

\begin{theorem}[Theorem 4.7, \cite{kklv10}] 
\label{th:canonlabel}
Given an interval hypergraph $\cH$, a canonical interval labeling $l_H$ 
for $H$ can be computed in FL.
\end{theorem}


Using the following construction it can be seen that COP testing is
indeed in logspace. Given a binary matrix $M$ of order $n \times m$,
let $S_i = \{j \mid M[j,i]=1 \}$. Let $\cF = \{S_i \mid i \in [m] \}$
be this set system. Construct a hypergraph $\cH$ with its vertex set
being $\{1, 2, \dots n\}$. The edge set of $\cH$ is isomorphic to
$\cF$. Thus every edge in $\cH$ represents a set in the given set
system $\cF$. Let this mapping be $\pi: E(\cH) \rightarrow \cF$. It is
easy to see that if $M$ has COP, then $\cH$ is an interval
hypergraph. From theorem \ref{th:canonlabel}, it is clear that the
interval labeling $l_{\cH}: V(\cH) \rightarrow [n]$ can be calculated
in logspace. Construct sets $I_i = \{ \l_{\cH}(x) \mid x \in E, E \in
E(\cH), \pi(E) = S_i\}$, for all $i \in [m]$. Since $\cH$ is an
interval hypergraph, $I_i$ is an interval for all $i \in [m]$, and is
the interval assigned to $S_i$ if $M$ has COP.

Now we have the following corollary.
\begin{corollary}
\label{cor:coplog}
  If a binary matrix $M$ has COP then the interval assignments to each
  of its columns can be calculated in FL.
\end{corollary}


\subsection{Generalized PQ tree}
The first linear time algorithm for testing COP for a binary matrix
was using a data structure called PQ trees invented by
\cite{bl76}. There is a PQ tree for a matrix iff the matrix has COP.
A PQ tree is a ``partly'' ordered tree which is created for a
binary matrix $M$ with COP. 
A PQ tree represents all COP orderings of $M$. 

The PQ tree is constructed by considering one column at a time and at
each iteration, either the algorithm reports $M$ has no COP or changes
the tree to create a PQ tree for the matrix induced by the columns
considered so far. The construction is very elaborate and complicated.

A closely related data structure is the generalized PQ tree in
\cite{mcc04}. 
In generalized PQ tree the P and Q nodes are called prime and linear
nodes. Aside from that, it has a third type of node called degenerate
nodes which is present only if the set system does not have COP.

\begin{theorem}[\cite{mcc04} Theorem 3.6]
If $\cF$ has COP, its generalized PQ tree has no degenerate nodes  
\end{theorem}

Using the idea of generalized PQ tree, this result proves that checking for bipartiteness in
the certain incomparability graph is sufficient to check for COP. 
\cite{mcc04} invented a certificate to confirm when a binary matrix
does not have COP. This is, to the best of our knowledge, the first
result that came up with such a verification strategy. Earlier work
only involved recognition of COP. If a matrix is recognized to have
COP, this is easy to verify because the algorithm outputs the COP
ordering (permutation of rows for COP in columns) and it is only a
matter of applying this ordering for verification.

\cite{mcc04} describes a graph called incompatibility graph of a set
system $\cF$ which has
vertices $(a,b), a \ne b$ for every $a, b \in U$, $U$ being the
universe of the set system. There are edges $((a,b),(b,c))$ and $((b,a),(c,b))$
if there is a set $S \in \cF$ such that $a, c \in S$ and $b \notin
S$. In other words the vertices of an edge in this graph represents
two orderings that cannot occur in a consecutive ones ordering of $\cF$.

\begin{theorem}[Theorem 6.1, \cite{mcc04}]
  Let $\cF$ be an arbitrary set family on domain $V$. Then $\cF$ has
  the consecutive ones property if and only if its incompatibility
  graph is bipartite, and if it does not have the consecutive ones
  property, the incompatibility graph has an odd cycle of length at
  most $n+3$\footnote{original theorem says $n+2$ but this is an error
  and was corrected by \cite{d08phd}}
\end{theorem}

Thus to check for COP of a set system (or matrix), one can check if
its incompatibility graph is bipartite. \cite{rei84} showed that
checking for bipartiteness can be done in logspace. Thus we arrive at
corollary \ref{cor:coplog} using this approach of COP testing as well. 


%\section {Acknowlegements:} 
%TBD

%\bibliographystyle{plainnat}
\bibliographystyle{alpha} %to have only [i] type of citation
\bibliography{cop-variants}

\end{document}
