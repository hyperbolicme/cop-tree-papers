\documentclass{article}


\usepackage{coptreefigures}
\usepackage{tikz}
\usetikzlibrary{backgrounds, positioning, fit,%
                shapes.geometric, shapes.misc%
               }

\def\labelsize{\small}


\begin{document}

% \begin{figure}[htb]
%   \centering

%   \begin{tabular}[t]{c}

%    % \ksubstartplTree\\
%    \ksubstartplMarginalSetOnTree\\

% %    \ksubstartplHypergraph\\
%     \ksubstartplMarginalSetOnHypergraph\\\\

%     \ksubstartplTreeStepIV\\

%   \end{tabular}

%   \caption{$k$-subdivided star tree path labeling example. (a) marginal
%     sets $S_1$ and $S_4$. The latter is super marginal. (b) $P_{17}$
%     is a partial labeled path that gets resolved in Step 4.}
%   \label{fig:ksubstartpl}
% \end{figure}

$k$-subdivided star tree path labeling example. The convention used is, path label $P_i$ maps to set $S_i$. 

\pagestyle{empty}
\begin{figure}[htb]
  \centering

  \begin{tabular}[t]{c}
    \ksubstartplMarginalSetOnTree\\\\
    \hline\\\\
    \ksubstartplMarginalSetOnHypergraph
  \end{tabular}

  \caption{\labelsize \textbf{Step 1:} The sets in shades of red are marginal
    sets: $S_1, S_4, S_6, S_7, S_{12}, S_{10}$. The maximal sets among
    them, $S_4, S_7, S_{10}$, are super-marginal. They are assigned to
  the paths with leaves as shown above, in Step 1. The green
  sets are the ones involved in illustrating the single
  inclusion chain that characterizes marginal sets. Gray sets are not
  involved in this step.}
  \label{fig:ksubstartpl-StepI}
\end{figure}

\begin{figure}[htb]
  \centering

  \begin{tabular}[t]{c}
    \ksubstartplSaturatingRaysOnTree\\\\
    \hline\\\\
    \ksubstartplSaturatingRaysHypergraph
  \end{tabular}

  \caption{\labelsize \textbf{Step 2:} The sets in shades of red are the
    overlapping sets that saturate each of the rays. All sets except
    $S_{14}, S_{15}, S_{17}$ are assigned paths as shown. The ones not
    assigned must proceed with partial labeling in Step 3. Gray sets
    are not involved in this step.}
  \label{fig:ksubstartpl-StepII}
\end{figure}

After Step 2 shown in Figure~\ref{fig:ksubstartpl-StepII}, the rest
of the steps could compute in different ways to arrive at a solution
depending on in what order $X \in \mathcal{O}$ is processed.



\begin{figure}[htb]
  \centering

  \begin{tabular}[t]{c}
    \ksubstartplPartialLabelingOnTreeI\\\\
    \hline\\\\
    \ksubstartplPartialLabelingHypergraph
  \end{tabular}

  \caption{\labelsize \textbf{Step 3:} The red paths are {\em one possibility of} partial labels to
    $S_{14}, S_{15}, S_{17}$. Note that the path's cardinality is
    lesser than the set's cardinality; hence partial labeling.}
  \label{fig:ksubstartpl-StepIIIa}
\end{figure}


\begin{figure}[htb]
  \centering

  \begin{tabular}[t]{c}
    \ksubstartplPartialLabelingOnTreeIStepIV\\\\
    \hline\\\\
    \ksubstartplPartialLabelingHypergraphIStepIV
  \end{tabular}

  \caption{\labelsize \textbf{Step 4:} Following the state of the
    algorithm shown in Fig.~\ref{fig:ksubstartpl-StepIIIa}, the red partial
    path $P_{14}$ grows on to a full path as shown here due to $S_{14}$'s
    overlap with $S_9$. This growth happens onto ray $R_2$ since $P_9$
    is on $R_2$. The gray sets are partial labeling which have not
    been resolved yet.}
  \label{fig:ksubstartpl-StepIVa}
\end{figure}

\begin{figure}[htb]
  \centering

  \begin{tabular}[t]{c}
   \ksubstartplPartialLabelingOnTreeIStepVI\\\\
   \hline\\\\
   \ksubstartplPartialLabelingHypergraphIStepVI
  \end{tabular}

  \caption{\labelsize \textbf{Step 6:} Following the state of the
    algorithm shown in Fig.~\ref{fig:ksubstartpl-StepIVa}, the red partial
    path $P_{17}$ grows on to a full path as shown here due to $S_{17}$'s
    overlap with $S_{15}$. This growth happens onto ray $R_1$ since
    the partial label $P_{15}$
    is on $R_1$. Note that Step 5 does not apply in this instance of
    the problem.}
  \label{fig:ksubstartpl-StepVIa}
\end{figure}

\begin{figure}[htb]
  \centering

  \begin{tabular}[t]{c}
   \ksubstartplPartialLabelingOnTreeIStepVII\\\\
   \hline\\\\
   \ksubstartplPartialLabelingHypergraphIStepVII
  \end{tabular}

  \caption{\labelsize \textbf{Step 7:} Following the state of the
    algorithm shown in Fig.~\ref{fig:ksubstartpl-StepVIa}, the red partial
    path $P_{15}$ grows on to a full path as shown here due to $S_{15}$'s
    overlap with $S_{5}$ and due to $P_5$ being in
    $\mathcal{L}_{1,2}$. Also $S_{15} \subseteq S_{14}$, hence only
    this new $P_{15}$ satisfies pariwise intersection cardinality property.}
  \label{fig:ksubstartpl-StepVIIa}
\end{figure}


\begin{figure}[htb]
  \centering

  \begin{tabular}[t]{c}
   \ksubstartplPartialLabelingOnTreeII\\
%   \hline\\\\
   \ksubstartplPartialLabelingOnTreeIIStepV
  \end{tabular}

  \caption{\labelsize \textbf{Step 5:} To illustrate Step 5, we
    consider a different possible partial assignment in Step 3 (rather
    than the one in Fig.~\ref{fig:ksubstartpl-StepIIIa}). In this instance,  $P_{14}$
    and $P_{15}$ get resolved by Step 5. They were partial labels from
    rays $R_2$ and $R_1$ respectively and by Step 5, it is understood
    that they ``grow into'' each others rays.}
  \label{fig:ksubstartpl-StepVb}
\end{figure}







% \begin{figure}[htb]
%   \centering
%   \begin{tabular}[t]{c}
%     \ksubstartplTreeStepIV
%   \end{tabular}
%   \caption{$k$-subdivided star tree path labeling example: $P_{17}$
%     gets resolved in Step 4. It was partially labeled from ray $R_i$ only
%     and gets resolved due to its overlap with $P_{16}$ which is from
%     $\mathcal{L}$ and labeled from ray $R_k$.}
%   \label{fig:ksubstartpl-stepIV}
% \end{figure}



% \begin{figure}[htb]
%   \centering
%   \begin{tabular}[t]{c}
%     \ksubstartplTreeStepV
%   \end{tabular}

%   \caption{$k$-subdivided star tree path labeling example: $P_{14}$
%     and $P_{15}$ get resolved in Step 5. They were partial labels from
%     rays $R_j$ and $R_i$ respectively and by Step 5, it is understood
%     that they ``grow into'' each others rays.}
%   \label{fig:ksubstartpl}
% \end{figure}


% \begin{figure}[htb]
%   \centering
%   \begin{tabular}[t]{c}
%    \ksubstartplTreeStepVI\\\\
%    \ksubstartplTreeStepVIb
%   \end{tabular}

%   \caption{$k$-subdivided star tree path labeling example: Partial
%     path labeling $P_{14}$ is resolved in Step 6. $X = P_{14}$, $Z =
%     P_{17}$, $R = R_i$. The second diagram shows a different scenario
%     that also gets fixedin Step 6.}
%   \label{fig:ksubstartpl}
% \end{figure}




\begin{figure}[htb]
  \centering

  \begin{tabular}[t]{c}
   \ksubstartplTree\\
   \ksubstartplHypergraph
  \end{tabular}

  \caption{\labelsize \textbf{Step 8:} This solution satisfies the
    properties of ICPPL and hence is a feasible solution.}
  \label{fig:ksubstartpl}
\end{figure}



\end{document}
